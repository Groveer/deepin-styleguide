\chapter{命名约定}

在deepin发行版本中,大约有100个左右的项目是deepin来进行维护的。为了保障项目的统一性,这里对deepin的总体命名进行一个阐述。

\section{通用名词} \label{general-naming-define}

通用名词是指由deepin所持有的或主导的相关名词以及缩写。

通用名词在代码,文件名,文档中有不同的变体。每个名词都会分别说明。

\subsection{deepin}

\textbf{总述}

deepin是指由deepin.org所有发行的发行版本。在指代发行版本时,应该永远使用小写的`deepin`。

\begin{DWarn}
  \DBox{
    在deepin 23以后的版本中,deepin将网站的主体迁移到deepin.org中,这将影响绝大部分的项目,特别是DBus接口部分。deepin承诺在2028年之前保障旧接口还是可以使用的。
  }
\end{DWarn}

\textbf{使用}

在文档,图片中,需要使用全小写的`deepin`,即使是首字母,也应该使用小写。

\begin{cppcode}
  deepin is an opensoucre os.  // 正确

  Deepin is an opensoucre os.  // 错误,即使是段落首字母,也不应该大写
\end{cppcode}

在代码中,需要使用全小写的`deepin`,除非代码风格规定了必须使用全大写,或首字母大小的情况。

\begin{cppcode}
  #define DEEPIN_MACRO XXXX     // 正确,以代码规范为准
  const int kDeepinNumber = 1;  // 正确,以代码规范为准

  // 版权信息中也需要使用小写的deepin
  // * Copyright (c) 2021. deepin All rights reserved.
\end{cppcode}

在文件名中,需要使用全小写的`deepin`。

\begin{cppcode}
  /usr/lib/deepin-daemon/dde-system-daemon  // 正确
  /usr/share/Deepin/msc/res  // 错误,应该为 /usr/share/deepin/msc/res
\end{cppcode}

\textbf{例外}

当deepin和其他名词组成专有名词时,可以使用大小写混合,例如:

\begin{inicode}
  # desktop文件中,deepin-music相关的应用
  [Desktop Entry]
  Name=Deepin Music
\end{inicode}

这里 \iniinline{Deepin Music}是一个专有名词,在任何情况下都不可以拆开使用。

\subsection{DDE}

\textbf{总述}

DDE是 \iniinline{Deepin Desktop Environment} 的缩写。

\DBox{
  \iniinline{Deepin Desktop Environment}也是专有名词,不要拆开使用,也不要写成\iniinline{deepin Desktop Environment},\iniinline{deepin desktop environment}等形式。
}

\textbf{使用}

在文档,图片中,需要使用全大写的`DDE`。

\begin{cppcode}
  The DDE is comprised of the Desktop Environment, deepin Window Manager, Control Center, Launcher and Dock.    // 正确
  Use dde in other os.                  // 错误,文档中只有大写
  Login to Dde.                         // 错误,文档中不允许混合大小写
\end{cppcode}

在代码中,需要使用全大写的`DDE`,除非代码风格规定了必须使用全大写,或首字母大写的情况。

\begin{cppcode}
  #define DDE_MACRO XXXX     // 正确,以代码规范为准
  const int kDdeNumber = 1;  // 正确,以代码规范为准
\end{cppcode}

在文件名中,需要使用全小写的`dde`。

\begin{cppcode}
  /usr/lib/deepin-daemon/dde-system-daemon  // 正确
\end{cppcode}

\section{项目命名} \label{deepin-project-naming}

\textbf{总述}

deepin项目应该使用全小写的命名方式,单词使用\cppinline{-}进行连接。但是如果是应用型的项目,也可以使用倒置域名进行命名。

\textbf{使用}

\begin{cppcode}
  org.deepin.lianliankan // 倒置域名格式,应用必须使用该方式命名
  plymouth-theme-deepin  // 正确
  deepin-font-manager    // 正确

  Robot-Autotest         // 错误,不使用大写
\end{cppcode}

\section{文件命名}

\textbf{总述}

对于安装到系统中的文件,其命名方式和\DFullRef{deepin-project-naming}相同。同时需要满足GNU/Linux的通用命名风格。

\textbf{使用}

\begin{cppcode}
  /usr/bin/dde-dock  // 正确
  /usr/share/polkit-1/actions/com.deepin.pkexec.dde-file-manager.policy  // 正确

  /usr/share/DeepinAIAssistant/translations  // 错误,不使用大写
  /usr/lib/deepin-daemon/logViewerService    // 错误, log-view-service
  /usr/lib/deepin-daemon/backlight_helper    // 错误,backlight-helper
\end{cppcode}

\section{DBus命名}

\textbf{总述}

DBus命名是一个较为模糊的地带,我们根据官方的设计文档\href{https://dbus.freedesktop.org/doc/dbus-api-design.html}{D-Bus API Design Guidelines}来指导DBus的命名规则:

DBus由服务名,路径,接口,方法(包括属性,信号等)四个部分组成。

对于服务名,路径,接口,其应该分解成域名,项目,组件三个部分。例如:

\begin{cppcode}
  org.deepin.Manual1.Search
  /org/deepin/Manual1/Search
  org.deepin.Manual1.Search
\end{cppcode}

其中 \cppinline{org.deepin} 是域名,Manual是项目名称,1是API版本号,Search是组件名称。其中:

\begin{itemize}
  \item 域名:使用倒置域名方法,目前deepin使用的域名为 \cppinline{org.deepin},  \cppinline{org.desktopspec}。
  \item 项目名称:使用大小写混合方式。根据\href{https://dbus.freedesktop.org/doc/dbus-api-design.html}{D-Bus API Design Guidelines},这里需要带上版本号。
  \item 组件名称:如果一个项目提供多个服务,那么这里就需要有组件名称,组件名称使用大小写混合方式。
\end{itemize}

DBus的方法(包括属性,信号等)永远使用大小写混合方式。

\textbf{使用}

注意,这里org.freedesktop.portal是域名,这也是DBus的接口风格中最让人迷惑的地方。

\begin{cppcode}
  org.freedesktop.portal.Desktop      // 正确,但是缺少API版本号
  /org/freedesktop/portal/Desktop     // 正确,但是缺少API版本号
  org.freedesktop.portal.Desktop      // 正确,但是缺少API版本号
\end{cppcode}

\begin{cppcode}
  org.deepin.DDE1.Accounts       // 正确
  /org/deepin/DDE1/Accounts      // 正确
  org.deepin.DDE1.Accounts       // 正确
  com.deepin.daemon.Accounts     // 错误,这是目前的命名方式,其中daemon意义不明确

  org.desktopspec.ConfigManager  // 正确,deepin将通用的标准在desktopsepc组织中进行实现

\end{cppcode}


\textbf{备注}

按照这种命名方式,目前deepin/DDE相关的绝大部分DBus接口需要重新设计。