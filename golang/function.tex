\chapter{函数}
\section{避免参数语义不明确}
函数调用中的意义不明确的参数可能会损害可读性。当参数名称的含义不明显时,请为参数添加 C 样式注释 (\texttt{/* ... */})
\begin{itemize}[leftmargin=4em]
\item 错误用法

  \begin{minted}{go}
    // func printInfo(name string, isLocal, done bool)

    printInfo("foo", true, true)
  \end{minted}
\item 正确用法

  \begin{minted}{go}
    // func printInfo(name string, isLocal, done bool)

    printInfo("foo", true /* isLocal */, true /* done */)
  \end{minted}
\end{itemize}

对于上面的示例代码,还有一种更好的处理方式是将上面的 \texttt{bool} 类型换成自定义类型。
将来,该参数可以支持不仅仅局限于两个状态(\texttt{true/false})。
\begin{minted}[xleftmargin=3.5em]{go}
type Region int

const (
	UnknownRegion Region = iota
	Local
)

type Status int

const (
	StatusReady Status= iota + 1
	StatusDone
	// Maybe we will have a StatusInProgress in the future.
)

func printInfo(name string, region Region, status Status)
\end{minted}

优先使用此种方式。

\section{最小职责原则}
函数实现时应遵循最小职责原则,尽量使函数的功能单一且简洁,避免多种功能揉合。
\begin{itemize}[leftmargin=4em]
\item 错误用法

  \begin{minted}{go}
    // Don't do this
    func main() {
    	fmt.Println(mulOfSums(1, 1))
    }

    func mulOfSums(a, b int) int {
    	return (a + b) * (a + b)
    }
  \end{minted}
\item 正确用法

  \begin{minted}{go}
    // Do this instead
    func main() {
    	fmt.Println(mul(add(1, 1), add(1, 1)))
    }

    func add(a, b int) int {
    	return a + b
    }

    func mul(a, b int) int {
    	return a * b
    }
  \end{minted}
\end{itemize}

\section{避免使用 init()}
尽可能避免使用 \texttt{init()} 。当 \texttt{init()} 是不可避免或可取的,代码应先尝试:
\begin{enumerate}[leftmargin=4em]
\item 无论程序环境或调用如何,都要完全确定;
\item 避免依赖于其它 \texttt{init()} 函数的顺序或副作用。虽然 \texttt{init()} 顺序是明确的,但代码可以更改, 因此 \texttt{init()} 函数之间的关系可能会使代码变得脆弱和容易出错;
\item 避免访问或操作全局或环境状态,如机器信息、环境变量、工作目录、程序参数/输入等;
\item 避免I/O,包括文件系统、网络和系统调用。
\end{enumerate}

不能满足这些要求的代码可能属于要作为 \texttt{main()} 调用的一部分(或程序生命周期中的其它地方), 或者作为 \texttt{main()} 本身的一部分写入。
特别是,打算由其它程序使用的库应该特别注意完全确定性, 而不是执行“init magic”。

\begin{itemize}[leftmargin=4em]
\item 错误用法

  \begin{minted}{go}
    type Foo struct {
    	// ...
    }
    var _defaultFoo Foo
    func init() {
    	_defaultFoo = Foo{
    		// ...
    	}
    }

    type Config struct {
    	// ...
    }
    var _config Config
    func init() {
    	// Bad: 基于当前目录
    	cwd, _ := os.Getwd()
    	// Bad: I/O
    	raw, _ := ioutil.ReadFile(
    		path.Join(cwd, "config", "config.yaml"),
    	)
    	yaml.Unmarshal(raw, &_config)
    }
  \end{minted}
\item 正确用法

  \begin{minted}{go}
    var _defaultFoo = Foo{
    	// ...
    }
    // or, 为了更好的可测试性:
    var _defaultFoo = defaultFoo()
    func defaultFoo() Foo {
    	return Foo{
    		// ...
    	}
    }
    type Config struct {
    	// ...
    }
    func loadConfig() Config {
    	cwd, err := os.Getwd()
    	// handle err
    	raw, err := ioutil.ReadFile(
    		path.Join(cwd, "config", "config.yaml"),
    	)
    	// handle err
    	var config Config
    	yaml.Unmarshal(raw, &config)
    	return config
    }
  \end{minted}
\end{itemize}

考虑到上述情况,在某些情况下, \texttt{init()} 可能更可取或是必要的,可能包括:
\begin{itemize}[leftmargin=4em]
\item 不能表示为单个赋值的复杂表达式;
\item 可插入的钩子,如 \texttt{database/sql} 、编码类型注册表等;
\item 对 \texttt{Google Cloud Functions} 和其它形式的确定性预计算的优化。
\end{itemize}

\section{主函数退出方式(Exit)}
Go 程序使用 \texttt{os.Exit} 或者 \texttt{log.Fatal\*} 立即退出 (使用 \texttt{panic} 不是退出程序的好方法)。

仅在 \texttt{main()} 中调用其中一个 \texttt{os.Exit} 或者 \texttt{log.Fatal\*} 。
所有其它函数应将错误通过返回值返回,并进行处理。
\begin{itemize}[leftmargin=4em]
\item 错误用法

  \begin{minted}{go}
    func main() {
    	body := readFile(path)
    	fmt.Println(body)
    }
    func readFile(path string) string {
    	f, err := os.Open(path)
    	if err != nil {
    		log.Fatal(err)
    	}
    	b, err := ioutil.ReadAll(f)
    	if err != nil {
    		log.Fatal(err)
    	}
    	return string(b)
    }
  \end{minted}
\item 正确用法

  \begin{minted}{go}
    func main() {
    	body, err := readFile(path)
    	if err != nil {
    		log.Fatal(err)
    	}
    	fmt.Println(body)
    }
    func readFile(path string) (string, error) {
    	f, err := os.Open(path)
    	if err != nil {
    		return "", err
    	}
    	b, err := ioutil.ReadAll(f)
    	if err != nil {
    		return "", err
    	}
    	return string(b), nil
    }
  \end{minted}
\end{itemize}

原则上,具有多种功能的程序退出存在一些问题:
\begin{itemize}[leftmargin=4em]
\item 不明显的控制流:任何函数都可以退出程序,因此很难对控制流进行推理;
\item 难以测试:退出程序的函数也将退出调用它的测试。这使得函数很难测试,并引入了跳过 \texttt{go test} 尚未运行的其它测试的风险;
\item 跳过清理:当函数退出程序时,会跳过已经进入 \texttt{defer} 队列里的函数调用。这增加了跳过重要清理任务的风险。
\end{itemize}

一次性退出原则,在 \texttt{main()} 函数中最多一次调用 \texttt{os.Exit} 或者 \texttt{log.Fatal} 。
如果有多个错误场景停止程序执行,请将该逻辑放在单独的函数下并从中返回错误。
这会缩短 \texttt{main()} 函数,并将所有关键业务逻辑放入一个单独的、可测试的函数中。

\begin{itemize}[leftmargin=4em]
\item 错误用法

  \begin{minted}{go}
    package main
    func main() {
    	args := os.Args[1:]
    	if len(args) != 1 {
    		log.Fatal("missing file")
    	}
    	name := args[0]
    	f, err := os.Open(name)
    	if err != nil {
    		log.Fatal(err)
    	}
    	defer f.Close()
    	// 如果我们调用log.Fatal 在这条线之后
    	// f.Close 将会被执行.
    	b, err := ioutil.ReadAll(f)
    	if err != nil {
    		log.Fatal(err)
    	}
    	// ...
    }
  \end{minted}
\item 正确用法

  \begin{minted}{go}
    package main
    func main() {
    	if err := run(); err != nil {
    		log.Fatal(err)
    	}
    }

    func run() error {
    	args := os.Args[1:]
    	if len(args) != 1 {
    		return errors.New("missing file")
    	}
    	name := args[0]
    	f, err := os.Open(name)
    	if err != nil {
    		return err
    	}
    	defer f.Close()
    	b, err := ioutil.ReadAll(f)
    	if err != nil {
    		return err
    	}
    	// ...
    }
  \end{minted}
\end{itemize}
