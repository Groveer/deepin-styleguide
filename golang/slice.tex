\chapter{Slice}
\section{nil 是一个有效的 slice}
\texttt{nil} 是一个有效的长度为 \texttt{0} 的 \texttt{slice} ,这意味着,不应明确返回长度为零的切片。
应该返回 \texttt{nil} 来代替。
\begin{itemize}[leftmargin=4em]
\item 错误用法

  \begin{minted}{go}
    if x == "" {
    	return []int{}
    }
  \end{minted}
\item 正确用法

  \begin{minted}{go}
    if x == "" {
    	return nil
    }
  \end{minted}
\end{itemize}

要检查切片是否为空,请始终使用 \texttt{len(s) == 0} 。而非 \texttt{nil} 。
\begin{itemize}[leftmargin=4em]
\item 错误用法

  \begin{minted}{go}
    func isEmpty(s []string) bool {
    	return s == nil
    }
  \end{minted}
\item 正确用法

  \begin{minted}{go}
    func isEmpty(s []string) bool {
    	return len(s) == 0
    }
  \end{minted}
\end{itemize}

零值切片可立即使用,无需调用 \texttt{make} 创建。
\begin{itemize}[leftmargin=4em]
\item 错误用法

  \begin{minted}{go}
    nums := []int{}
    // or, nums := make([]int)

    if add1 {
    	nums = append(nums, 1)
    }

    if add2 {
    	nums = append(nums, 2)
    }
  \end{minted}
\item 正确用法

  \begin{minted}{go}
    var nums []int

    if add1 {
    	nums = append(nums, 1)
    }

    if add2 {
    	nums = append(nums, 2)
    }
  \end{minted}
\end{itemize}

记住,虽然 \texttt{nil} 切片是有效的切片,但它不等于长度为 \texttt{0} 的切片(一个为 \texttt{nil} ,另一个不是),
并且在不同的情况下(例如序列化),这两个切片的处理方式可能不同。

\section{追加时优先指定切片容量}
在尽可能的情况下,在初始化要追加的切片时为make()提供一个容量值。
\begin{itemize}[leftmargin=4em]
\item 错误用法

  \begin{minted}{go}
    for n := 0; n < b.N; n++ {
    	data := make([]int, 0)
    	for k := 0; k < size; k++{
    		data = append(data, k)
    	}
    }
  \end{minted}
\item 正确用法

  \begin{minted}{go}
    for n := 0; n < b.N; n++ {
    	data := make([]int, 0, size)
    	for k := 0; k < size; k++{
    		data = append(data, k)
    	}
    }
  \end{minted}
\end{itemize}
