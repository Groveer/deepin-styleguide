\chapter{扉页}

\section{前言}
风格一致的代码更合理,学习成本更少,且更容易维护。随着新的约定出现或者出现错误后更容易迁移、更新、修复 \texttt{bug} 。

相反,在一个代码库中包含多个完全不同或冲突的代码风格会导致维护成本增加、不确定性上升和部分认知偏差。所有这些都会直接导致速度降低、代码审查痛苦,并且会增加 \texttt{bug} 数量。

因此需要为代码库制定体套标准,目的是规范 \texttt{Go} 项目的开发,保持代码的一致性,使代码库易于管理和维护。

本规范主要基于 \texttt{Uber GoLang Style Guide} 进行编写,同时结合了工作中的实践,给出了提高性能和安全性的编码技巧。

\section{参考文献}
\begin{itemize}[leftmargin=4em]
  \item \href{https://github.com/uber-go/guide/blob/master/style.md}{Uber GoLang Style Guide}
  \item \href{https://github.com/golang-standards/project-layout}{Standard Go Project Layout}
  \item \href{https://golang.org/doc/effective\_go#mixed-caps}{函数命名规则}
  \item \href{https://github.com/golang/go/wiki/CodeReviewComments}{Go代码审查意见}
  \item \href{https://golang.org/doc/effective\_go}{Effective Go}
  \item \href{https://golang.org/ref/spec}{Go语言规范}
  \item \href{https://yougg.github.io/2017/06/12/go\%E8\%AF\%AD\%E8\%A8\%80\%E5\%AE\%89\%E5\%85\%A8\%E7\%BC\%96\%E7\%A8\%8B\%E8\%A7\%84\%E8\%8C\%83/}{Go语言安全编程规范}
  \item \href{https://gruntwork.io/guides/style\%20guides/golang-style-guide}{Gruntwork Go Style Guide}
\end{itemize}
