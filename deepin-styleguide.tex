\documentclass[UTF8,a4paper,oneside]{ctexbook}

\usepackage{indentfirst}
\setlength{\parindent}{2em}

\usepackage[margin=1in]{geometry}

\usepackage{xcolor}
\definecolor{deepin_blue}{HTML}{0088FD}
\definecolor{deepin_black}{HTML}{000000}
\definecolor{deepin_orange}{RGB}{244, 151, 0}

\usepackage{minted}

\newminted{cpp}{frame=lines,framerule=1pt,breaklines,breakanywhere}
\newmintinline{cpp}{breaklines,breakanywhere}

\newminted{ini}{frame=lines,framerule=1pt,breaklines,breakanywhere}
\newmintinline{ini}{breaklines,breakanywhere}

\usepackage{nameref}
\usepackage[colorlinks=true]{hyperref}

\newcommand*{\DFullRef}[1]{\hyperref[{#1}]{\ref*{#1} \nameref*{#1}}}

\usepackage{titlesec}
\titleformat{\chapter}[hang]
{\huge\bfseries}
{\thechapter\hspace{20pt}\textcolor{deepin_black}{|}\hspace{20pt}}{0pt}
{\huge\bfseries}

\usepackage{enumitem}

\makeatletter
\renewcommand{\section}{
  \@startsection{section}{1}{0mm}
  {0.5\baselineskip}{0.5\baselineskip}{\bf\leftline}
 }
\makeatother

\makeatletter
\newcommand*{\DBox}[1]{
\@makeother\#
\vspace{0.5\baselineskip}
\noindent\fbox{\parbox{\linewidth}{#1}}
}
\makeatother

\makeatletter
\newcommand*{\GWarn}[1]{
\@makeother\#
\noindent\fbox{\parbox{\linewidth}{#1}}
}
\makeatother

\usepackage{framed}
\usepackage{quoting}

\newenvironment{DNote}
{
    \color{deepin_blue}
    \bfseries
    \quoting[leftmargin=0pt, vskip=0pt,noorphans]
}
{
    \endquoting
}

\newenvironment{DWarn}
{
    \color{deepin_orange}
    \bfseries
    \quoting[leftmargin=0pt, vskip=0pt,noorphans]
}
{
    \endquoting
}

\usepackage{verbatim}

% 设置段落间距
\setlength{\parskip}{0.5em}
\makeatletter
\@addtoreset{chapter}{part}
\makeatother
\title{deepin 开源项目风格指南}

\date{\today}

\begin{document}
\maketitle

\textbf{声明}

本项目是在\href{https://github.com/zh-google-styleguide/zh-google-styleguide}{Google 开源项目风格指南——中文版}基础上修改而来。

本项目使用 \LaTeX 构建。

\textbf{Google 开源项目风格指南——中文版声明}

项目原始地址是:\href{https://github.com/zh-google-styleguide/zh-google-styleguide}{Google 开源项目风格指南}

如果你关注的是 Google 官方英文版,请移步 \href{https://github.com/google/styleguide}{Google Style Guide}

每个较大的开源项目都有自己的风格指南:关于如何为该项目编写代码的一系列约定(有时候会比较武断)。当所有代码均保持一致的风格,在理解大型代码库时更为轻松。

“风格”的含义涵盖范围广,从“变量使用驼峰格式(camelCase)”到“决不使用全局变量”再到“决不使用异常”,等等诸如此类。

英文版项目维护的是在 Google 使用的编程风格指南。如果你正在修改的项目源自 Google,你可能会被引导至英文版项目页面,以了解项目所使用的风格。

中文版项目采用 reStructuredText 纯文本标记语法,并使用 Sphinx 生成 HTML / CHM / PDF 等文档格式。

英文版项目还包含 \href{https://github.com/google/styleguide/tree/gh-pages/cpplint}{cpplint} ——一个用来帮助适应风格准则的工具,以及 \href{https://raw.githubusercontent.com/google/styleguide/gh-pages/google-c-style.el}{google-c-style.el},Google 风格的 Emacs 配置文件。

\tableofcontents
\newpage

\part{项目风格}

\chapter{命名约定}

在deepin发行版本中,大约有100个左右的项目是deepin来进行维护的。为了保障项目的统一性,这里对deepin的总体命名进行一个阐述。

\section{通用名词} \label{general-naming-define}

通用名词是指由deepin所持有的或主导的相关名词以及缩写。

通用名词在代码,文件名,文档中有不同的变体。每个名词都会分别说明。

\subsection{deepin}

\textbf{总述}

deepin是指由deepin.org所有发行的发行版本。在指代发行版本时,应该永远使用小写的`deepin`。

\begin{DWarn}
  \DBox{
    在deepin 23以后的版本中,deepin将网站的主体迁移到deepin.org中,这将影响绝大部分的项目,特别是DBus接口部分。deepin承诺在2028年之前保障旧接口还是可以使用的。
  }
\end{DWarn}

\textbf{使用}

在文档,图片中,需要使用全小写的`deepin`,即使是首字母,也应该使用小写。

\begin{cppcode}
  deepin is an opensoucre os.  // 正确

  Deepin is an opensoucre os.  // 错误,即使是段落首字母,也不应该大写
\end{cppcode}

在代码中,需要使用全小写的`deepin`,除非代码风格规定了必须使用全大写,或首字母大小的情况。

\begin{cppcode}
  #define DEEPIN_MACRO XXXX     // 正确,以代码规范为准
  const int kDeepinNumber = 1;  // 正确,以代码规范为准

  // 版权信息中也需要使用小写的deepin
  // * Copyright (c) 2021. deepin All rights reserved.
\end{cppcode}

在文件名中,需要使用全小写的`deepin`。

\begin{cppcode}
  /usr/lib/deepin-daemon/dde-system-daemon  // 正确
  /usr/share/Deepin/msc/res  // 错误,应该为 /usr/share/deepin/msc/res
\end{cppcode}

\textbf{例外}

当deepin和其他名词组成专有名词时,可以使用大小写混合,例如:

\begin{inicode}
  # desktop文件中,deepin-music相关的应用
  [Desktop Entry]
  Name=Deepin Music
\end{inicode}

这里 \iniinline{Deepin Music}是一个专有名词,在任何情况下都不可以拆开使用。

\subsection{DDE}

\textbf{总述}

DDE是 \iniinline{Deepin Desktop Environment} 的缩写。

\DBox{
  \iniinline{Deepin Desktop Environment}也是专有名词,不要拆开使用,也不要写成\iniinline{deepin Desktop Environment},\iniinline{deepin desktop environment}等形式。
}

\textbf{使用}

在文档,图片中,需要使用全大写的`DDE`。

\begin{cppcode}
  The DDE is comprised of the Desktop Environment, deepin Window Manager, Control Center, Launcher and Dock.    // 正确
  Use dde in other os.                  // 错误,文档中只有大写
  Login to Dde.                         // 错误,文档中不允许混合大小写
\end{cppcode}

在代码中,需要使用全大写的`DDE`,除非代码风格规定了必须使用全大写,或首字母大写的情况。

\begin{cppcode}
  #define DDE_MACRO XXXX     // 正确,以代码规范为准
  const int kDdeNumber = 1;  // 正确,以代码规范为准
\end{cppcode}

在文件名中,需要使用全小写的`dde`。

\begin{cppcode}
  /usr/lib/deepin-daemon/dde-system-daemon  // 正确
\end{cppcode}

\section{项目命名} \label{deepin-project-naming}

\textbf{总述}

deepin项目应该使用全小写的命名方式,单词使用\cppinline{-}进行连接。但是如果是应用型的项目,也可以使用倒置域名进行命名。

\textbf{使用}

\begin{cppcode}
  org.deepin.lianliankan // 倒置域名格式,应用必须使用该方式命名
  plymouth-theme-deepin  // 正确
  deepin-font-manager    // 正确

  Robot-Autotest         // 错误,不使用大写
\end{cppcode}

\section{文件命名} \label{deepin-file-naming}

\textbf{总述}

对于安装到系统中的文件,其命名方式和\DFullRef{deepin-project-naming}相同。同时需要满足GNU/Linux的通用命名风格。

\textbf{使用}

\begin{cppcode}
  /usr/bin/dde-dock  // 正确
  /usr/share/polkit-1/actions/com.deepin.pkexec.dde-file-manager.policy  // 正确

  /usr/share/DeepinAIAssistant/translations  // 错误,不使用大写
  /usr/lib/deepin-daemon/logViewerService    // 错误, log-view-service
  /usr/lib/deepin-daemon/backlight_helper    // 错误,backlight-helper
\end{cppcode}

\section{DBus命名}

\textbf{总述}

DBus命名是一个较为模糊的地带,我们根据官方的设计文档\href{https://dbus.freedesktop.org/doc/dbus-api-design.html}{D-Bus API Design Guidelines}来指导DBus的命名规则:

DBus由服务名,路径,接口,方法(包括属性,信号等)四个部分组成。

对于服务名,路径,接口,其应该分解成域名,项目,组件三个部分。例如:

\begin{cppcode}
  org.deepin.Manual1.Search
  /org/deepin/Manual1/Search
  org.deepin.Manual1.Search
\end{cppcode}

其中 \cppinline{org.deepin} 是域名,Manual是项目名称,1是API版本号,Search是组件名称。其中:

\begin{itemize}
  \item 域名:使用倒置域名方法,目前deepin使用的域名为 \cppinline{org.deepin},  \cppinline{org.desktopspec}。
  \item 项目名称:使用大小写混合方式。根据\href{https://dbus.freedesktop.org/doc/dbus-api-design.html}{D-Bus API Design Guidelines},这里需要带上版本号。
  \item 组件名称:如果一个项目提供多个服务,那么这里就需要有组件名称,组件名称使用大小写混合方式。
\end{itemize}

DBus的方法(包括属性,信号等)永远使用大小写混合方式。

\textbf{使用}

注意,这里org.freedesktop.portal是域名,这也是DBus的接口风格中最让人迷惑的地方。

\begin{cppcode}
  org.freedesktop.portal.Desktop      // 正确,但是缺少API版本号
  /org/freedesktop/portal/Desktop     // 正确,但是缺少API版本号
  org.freedesktop.portal.Desktop      // 正确,但是缺少API版本号
\end{cppcode}

\begin{cppcode}
  org.deepin.DDE1.Accounts       // 正确
  /org/deepin/DDE1/Accounts      // 正确
  org.deepin.DDE1.Accounts       // 正确
  com.deepin.daemon.Accounts     // 错误,这是目前的命名方式,其中daemon意义不明确

  org.desktopspec.ConfigManager  // 正确,deepin将通用的标准在desktopsepc组织中进行实现

\end{cppcode}


\textbf{备注}

按照这种命名方式,目前deepin/DDE相关的绝大部分DBus接口需要重新设计。

\chapter{创建}

在deepin发行版本中,大约有100个左右的项目是deepin来进行维护的。为了避免项目冗余,这里对项目以及文件的创建进行一个阐述。

\section{项目创建}

\textbf{总述}

在新创建一个项目之前,应首先考虑deepin发行版已有的项目是否支持功能扩展,不推荐一味的进行项目新建。对于某个新特性必须创建项目时,其命名方式和\DFullRef{deepin-project-naming}相同。

\DBox{
	\iniinline{deepin-default-settings}、\iniinline{uos-config}同属系统通用配置文件管理项目,在后续的版本中,将废弃uos-config项目,使用deepin-default-settings进行统一管理。
}

\DBox{
	\iniinline{dde-wayland-config}、\iniinline{kwayland-data}同属wayland协议配置文件管理项目,推荐使用dde-wayland-config进行统一管理。
}

\textbf{备注}

按照项目创建规则,目前deepin/DDE相关的部分项目需要重新设计。

\section{文件创建}

\textbf{总述}

在新创建一个文件时,其命名方式和\DFullRef{deepin-file-naming}相同,文件路径遵循\href{https://refspecs.linuxfoundation.org/FHS_3.0/fhs-3.0.html}{Filesystem Hierarchy Standard}文件系统层次标准,推荐使用Debian规则中的install文件进行管理,减少使用代码直接进行文件创建。

\textbf{使用}

\begin{cppcode}
  /usr/lib/libudis86.so       // 正确
  /etc/os-version             // 正确
  /usr/bin/apt                // 正确
  /etc/qemu-ifup              // 错误,/etc目录下存放配置文件,不能存放二进制可执行文件
  /usr/share/uos_resources    // 错误,该文件没有归属于任意项目
\end{cppcode}

\part{Qt代码风格}

\chapter{扉页}

\section{前言}

Deepin是一个有着长久Qt开发和使用历史的开源组织。在项目开发过程中,我们采用过很多的代码风格和工具,大多数准则和工具都消散在漫长的项目变动过程中,这导致新的开发者无法开发出符合项目风格的代码。

Qt是一个C++语言的框架,由于Qt项目出现较早,很多C++的特性并没有出现,之后也没有被Qt采用,这也使得Qt与其他风格C++变化越来越大。这份指南将会总结和约定Deepin在开发中如何使用Qt。

同时本指南不包含QML的部分,我们会在另外一份指南中来详细描述如何使用QML。

\section{更新日志}

\textbf{版本}

0.0.1

\textbf{原作者}

Benjy Weinberger

Craig Silverstein

Gregory Eitzmann

Mark Mentovai

Tashana Landray

\textbf{翻译}

\href{http://www.yulefox.com}{YuleFox}

\href{https://github.com/yangyubo}{Yang.Y}

\href{http://acgtyrant.com}{acgtyrant}

\href{http://github.com/lilinsanity}{lilinsanity}

\textbf{Qt部分修订者}

\href{https://blog.iceyer.net}{Iceyer}

\textbf{项目主页}

\href{https://gitlab.deepin.org/styleguide}{Deepin 开源项目风格指南}

\href{https://github.com/zh-google-styleguide/zh-google-styleguide}{Google 开源项目风格指南}

\href{https://github.com/google/styleguide}{Google Style Guide}


\section{《Google 开源项目风格指南——中文版》前言}

Google 经常会发布一些开源项目,意味着会接受来自其他代码贡献者的代码。但是如果代码贡献者的编程风格与 Google 的不一致,会给代码阅读者和其他代码提交者造成不小的困扰。Google 因此发布了这份自己的编程风格指南,使所有提交代码的人都能获知 Google 的编程风格。

翻译初衷:

规则的作用就是避免混乱。但规则本身一定要权威,有说服力,并且是理性的。我们所见过的大部分编程规范,其内容或不够严谨,或阐述过于简单,或带有一定的武断性。Google 保持其一贯的严谨精神,5 万汉字的指南涉及广泛,论证严密。我们翻译该系列指南的主因也正是其严谨性。

严谨意味着指南的价值不仅仅局限于它罗列出的规范,更具参考意义的是它为了列出规范而做的谨慎权衡过程。

指南不仅列出你要怎么做,还告诉你为什么要这么做,哪些情况下可以不这么做,以及如何权衡其利弊。其他团队未必要完全遵照指南亦步亦趋,如前面所说,这份指南是 Google 根据自身实际情况打造的,适用于其主导的开源项目。其他团队可以参照该指南,或从中汲取灵感,建立适合自身实际情况的规范。

我们在翻译的过程中,收获颇多。希望本系列指南中文版对你同样能有所帮助。

我们翻译时也是尽力保持严谨,但水平所限,bug 在所难免。有任何意见或建议,可与我们取得联系。

中文版和英文版一样,使用 ``Artistic License/GPL`` 开源许可。

中文版修订历史:

\begin{itemize}
  \item 2015-08 : 热心的清华大学同学 @lilinsanity 完善了「类」章节以及其它一些小章节。至此,对 Google CPP Style Guide 4.45 的翻译正式竣工。
  \item 2015-07 4.45 : acgtyrant 为了学习 C++ 的规范,顺便重新翻译了本 C++ 风格指南,特别是 C++11 的全新内容。排版大幅度优化,翻译措辞更地道,添加了新译者笔记。Google 总部 C++ 工程师 innocentim,	清华大学不愿意透露姓名的唐马儒先生,大阪大学大学院情报科学研究科计算机科学专攻博士 arseerfc 和其它 Arch Linux 	中文社区众帮了译者不少忙,谢谢他们。因为 C++ Primer 尚未完全入门,暂时没有翻译「类」章节和其它一些小章节。
  \item  2009-06 3.133 : YuleFox 的 1.0 版已经相当完善,但原版在近一年的时间里,其规范也发生了一些变化.Yang.Y 与 YuleFox 一拍即合,以项目的形式来延续中文版 : \href{http://github.com/yangyubo/zh-google-styleguide}{Google 开源项目风格指南 - 中文版项目}。主要变化是同步到 3.133 最新英文版本,做部分勘误和改善可读性方面的修改,并改进排版效果。Yang.Y 重新翻修,YuleFox 做后续评审。
  \item  2008-07 1.0 : 出自 \href{http://www.yulefox.com/?p=207}{YuleFox 的 Blog}` 很多地方摘录的也是该版本。
\end{itemize}

\section{原始背景}

C++ 是 Google 大部分开源项目的主要编程语言。正如每个 C++ 程序员都知道的,C++ 有很多强大的特性,但这种强大不可避免的导致它走向复杂,使代码更容易产生 bug,难以阅读和维护。

本指南的目的是通过详细阐述 C++ 注意事项来驾驭其复杂性。这些规则在保证代码易于管理的同时,也能高效使用 C++ 的语言特性。

*风格*,亦被称作可读性,也就是指导 C++ 编程的约定。使用术语 "风格" 有些用词不当,因为这些习惯远不止源代码文件格式化这么简单。

使代码易于管理的方法之一是加强代码一致性。让任何程序员都可以快速读懂你的代码这点非常重要。保持统一编程风格并遵守约定意味着可以很容易根据"模式匹配" 规则来推断各种标识符的含义。创建通用,必需的习惯用语和模式可以使代码更容易理解。在一些情况下可能有充分的理由改变某些编程风格,但我们还是应该遵循一致性原则,尽量不这么做。

本指南的另一个观点是 C++ 特性的臃肿。C++ 是一门包含大量高级特性的庞大语言。某些情况下,我们会限制甚至禁止使用某些特性。这么做是为了保持代码清爽,避免这些特性可能导致的各种问题。指南中列举了这类特性,并解释为什么这些特性被限制使用。

Google 主导的开源项目均符合本指南的规定。

注意: 本指南并非 C++ 教程,我们假定读者已经对 C++ 非常熟悉。
\chapter{头文件}

通常每一个 \cppinline{.cpp} 文件都有一个对应的 \cppinline{.h} 文件. 也有一些常见例外, 如单元测试代码和只包含 \cppinline{main()} 函数的\cppinline{.cpp} 文件。

正确使用头文件可令代码在可读性、文件大小和性能上大为改观。

下面的规则将引导你规避使用头文件时的各种陷阱。

\section{Self-contained 头文件}

\DBox {
	头文件应该能够自给自足(self-contained,也就是可以作为第一个头文件被引入),以 \cppinline{.h} 结尾。至于用来插入文本的文件,说到底它们并不是头文件,所以应以 \cppinline{.inc} 结尾。不允许分离出 \cppinline{-inl.h} 头文件的做法。
}

所有头文件要能够自给自足。换言之,用户和重构工具不需要为特别场合而包含额外的头文件。详言之,一个头文件要有\DFullRef{pragma-once-guard},统统包含它所需要的其它头文件,也不要求定义任何特别 symbols。

不过有一个例外,即一个文件并不是 self-contained 的,而是作为文本插入到代码某处。或者,文件内容实际上是其它头文件的特定平台(platform-specific)扩展部分。这些文件就要用\cppinline{.inc} 文件扩展名。

如果 \cppinline{.h} 文件声明了一个模板或内联函数,同时也在该文件加以定义。凡是有用到这些的 \cppinline{.cpp} 文件,就得统统包含该头文件,否则程序可能会在构建中链接失败。不要把这些定义放到分离的 \cppinline{-inl.h} 文件里(译者注:过去该规范曾提倡把定义放到 -inl.h 里过)。

有个例外:如果某函数模板为所有相关模板参数显式实例化,或本身就是某类的一个私有成员,那么它就只能定义在实例化该模板的 \cppinline{.cpp} 文件里。

\section{pragma once 保护} \label{pragma-once-guard}

\DBox{
	所有头文件都应该使用 \cppinline{#pragma once} 保护来防止头文件被多重包含。且该语句必须放在第一行。
}

\begin{DWarn}
	这里,Google推荐使用 \cppinline{#pragma once} 形式。 \cppinline{#define} 方式的主要优点是其实现是符合C++/C标准的,但是这带来的宏命名不一致的问题严重降低的代码可读性。并且考虑基于Qt的项目对编译器本来就有一定的筛选,我们认为针对Qt项目使用 \cppinline{#pragma once} 风格是合适的。
\end{DWarn}

\section{前置声明} \label{forward-declarations}

\DBox {
	尽可能地避免使用前置声明。使用 \cppinline{#include} 包含需要的头文件即可。
}

\textbf{定义:}

所谓「前置声明」(forward declaration)是类、函数和模板的纯粹声明,没伴随着其定义.

\textbf{优点:}

\begin{itemize}
	\item 前置声明能够节省编译时间,多余的 \cppinline{#include} 会迫使编译器展开更多的文件,处理更多的输入。
	\item 前置声明能够节省不必要的重新编译的时间。 \cppinline{#include} 使代码因为头文件中无关的改动而被重新编译多次。
\end{itemize}

\textbf{缺点:}

\begin{itemize}
	\item 前置声明隐藏了依赖关系,头文件改动时,用户的代码会跳过必要的重新编译过程。
	\item 前置声明可能会被库的后续更改所破坏。前置声明函数或模板有时会妨碍头文件开发者变动其 API。例如扩大形参类型,加个自带默认参数的模板形参等等。
	\item 前置声明来自命名空间 \cppinline{std::} 的 symbol 时,其行为未定义。
	\item 很难判断什么时候该用前置声明,什么时候该用 \cppinline{#include} 。极端情况下,用前置声明代替 \cppinline{#include} 甚至都会暗暗地改变代码的含义:

\begin{cppcode}
// b.h:
struct B {};
struct D : B {};

// good_user.cpp:
#include "b.h"
void f(B*);
void f(void*);
void test(D* x) { f(x); }  // calls f(B*)
\end{cppcode}

	      如果 \cppinline{#include} 被 \cppinline{B} 和 \cppinline{D} 的前置声明替代, \cppinline{test()} 就会调用 \cppinline{f(void*)} 。	\item 前置声明了不少来自头文件的 symbol 时,就会比单单一行的 \cppinline{include} 冗长。
	\item 仅仅为了能前置声明而重构代码(比如用指针成员代替对象成员)会使代码变得更慢更复杂.
\end{itemize}

\textbf{结论:}

\begin{itemize}
	\item 尽量避免前置声明那些定义在其他项目中的实体。
	\item 函数:总是使用 \cppinline{#include} 。
	\item 类模板:优先使用 \cppinline{#include} 。
\end{itemize}

至于什么时候包含头文件,参见 \DFullRef{name-and-order-of-includes} 。


\section{内联函数} \label{inline-functions}

\DBox {
	只有当函数只有 10 行甚至更少时才将其定义为内联函数。
}

\textbf{定义:}

当函数被声明为内联函数之后, 编译器会将其内联展开, 而不是按通常的函数调用机制进行调用。

\textbf{优点:}

只要内联的函数体较小, 内联该函数可以令目标代码更加高效. 对于存取函数以及其它函数体比较短, 性能关键的函数, 鼓励使用内联.

\textbf{缺点:}

滥用内联将导致程序变得更慢. 内联可能使目标代码量或增或减, 这取决于内联函数的大小. 内联非常短小的存取函数通常会减少代码大小,但内联一个相当大的函数将戏剧性的增加代码大小. 现代处理器由于更好的利用了指令缓存, 小巧的代码往往执行更快。

\textbf{结论:}

一个较为合理的经验准则是, 不要内联超过 10 行的函数. 谨慎对待析构函数, 析构函数往往比其表面看起来要更长,因为有隐含的成员和基类析构函数被调用!

另一个实用的经验准则: 内联那些包含循环或 \cppinline{switch} 语句的函数常常是得不偿失 (除非在大多数情况下, 这些循环或 \cppinline{switch} 语句从不被执行).

有些函数即使声明为内联的也不一定会被编译器内联, 这点很重要; 比如虚函数和递归函数就不会被正常内联. 通常,递归函数不应该声明成内联函数.(YuleFox 注: 递归调用堆栈的展开并不像循环那么简单, 比如递归层数在编译时可能是未知的,大多数编译器都不支持内联递归函数). 虚函数内联的主要原因则是想把它的函数体放在类定义内, 为了图个方便, 抑或是当作文档描述其行为,比如精短的存取函数.

\section{ include 的路径及顺序} \label{name-and-order-of-includes}

\DBox{
	使用标准的头文件包含顺序可增强可读性, 避免隐藏依赖: 相关头文件, C 库, C++ 库, 其他库的 `.h`, 本项目内的 `.h`.
}

项目内头文件应按照项目源代码目录树结构排列, 避免使用 UNIX 特殊的快捷目录 \cppinline{.} (当前目录) 或 \cppinline{..} (上级目录)。

例如, \cppinline{google-awesome-project/src/base/logging.h} 应该按如下方式包含:

\begin{cppcode}
	#include "base/logging.h"
\end{cppcode}

又如, \cppinline{dir/foo.cpp} 或 \cppinline{dir/foo_test.cpp} 的主要作用是实现或测试 \cppinline{dir2/foo2.h}的功能, \cppinline{foo.cpp} 中包含头文件的次序如下:

\begin{enumerate}
	\item \cppinline{dir2/foo2.h} (优先位置, 详情如下)
	\item C 系统文件
	\item C++ 系统文件
	\item 其他库的 \cppinline{.h} 文件
	\item 本项目内 \cppinline{.h} 文件
\end{enumerate}

这种优先的顺序排序保证当 \cppinline{dir2/foo2.h} 遗漏某些必要的库时, \cppinline{dir/foo.cpp} 或 \cppinline{dir/foo_test.cpp} 的构建会立刻中止。因此这一条规则保证维护这些文件的人们首先看到构建中止的消息而不是维护其他包的人们。

\cppinline{dir/foo.cpp} 和 \cppinline{dir2/foo2.h} 通常位于同一目录下,如\cppinline{base/basictypes_unittest.cpp} 和 \cppinline[breaklines]{base/basictypes.h}, 但也可以放在不同目录下.

按字母顺序分别对每种类型的头文件进行二次排序是不错的主意。注意较老的代码可不符合这条规则,要在方便的时候改正它们。

您所依赖的符号 (symbols) 被哪些头文件所定义,您就应该包含(include)哪些头文件,`前置声明`(forward declarations) 情况除外。比如您要用到 \cppinline{bar.h} 中的某个符号, 哪怕您所包含的 \cppinline{foo.h} 已经包含了\cppinline{bar.h}, 也照样得包含 \cppinline{bar.h}, 除非 \cppinline{foo.h} 有明确说明它会自动向您提供 \cppinline{bar.h} 中的symbol. 不过,凡是 cc 文件所对应的「相关头文件」已经包含的,就不用再重复包含进其 cc 文件里面了,就像 \cppinline{foo.cpp}只包含 \cppinline{foo.h} 就够了,不用再管后者所包含的其它内容。

举例来说, \cppinline{google-awesome-project/src/foo/internal/fooserver.cpp} 的包含次序如下:

\begin{cppcode}
	#include "foo/public/fooserver.h" // 优先位置

	#include <sys/types.h>
	#include <unistd.h>

	#include <hash_map>
	#include <vector>

	#include "base/basictypes.h"
	#include "base/commandlineflags.h"
	#include "foo/public/bar.h"
\end{cppcode}

\textbf{例外:}

有时,平台特定(system-specific)代码需要条件编译(conditional includes),这些代码可以放到其它 includes 之后。当然,您的平台特定代码也要够简练且独立,比如:

\begin{cppcode}
	#include "foo/public/fooserver.h"

	#include "base/port.h"  // For LANG_CXX11.

	#ifdef LANG_CXX11
	#include <initializer_list>
	#endif  // LANG_CXX11
\end{cppcode}


\section{注解}

\subsection{译者 (YuleFox) 笔记}

\begin{itemize}
	\item  避免多重包含是学编程时最基本的要求;
	\item  前置声明是为了降低编译依赖,防止修改一个头文件引发多米诺效应;
	\item  内联函数的合理使用可提高代码执行效率;
	\item  \cppinline{-inl.h} 可提高代码可读性 (一般用不到吧:D);
	\item  标准化函数参数顺序可以提高可读性和易维护性 (对函数参数的堆栈空间有轻微影响, 我以前大多是相同类型放在一起);
	\item  包含文件的名称使用 \cppinline{.} 和 \cppinline{..} 虽然方便却易混乱, 使用比较完整的项目路径看上去很清晰, 很条理,包含文件的次序除了美观之外, 最重要的是可以减少隐藏依赖, 使每个头文件在 "最需要编译" (对应源文件处 :D) 的地方编译,有人提出库文件放在最后, 这样出错先是项目内的文件, 头文件都放在对应源文件的最前面, 这一点足以保证内部错误的及时发现了.
\end{itemize}

\subsection{译者(acgtyrant)笔记}

\begin{itemize}
	\item  原来还真有项目用 \cppinline{#include} 来插入文本,且其文件扩展名 \cppinline{.inc} 看上去也很科学。
	\item  Google 已经不再提倡 \cppinline{-inl.h} 用法。
	\item  注意,前置声明的类是不完全类型(incomplete type),我们只能定义指向该类型的指针或引用,或者声明(但不能定义)以不完全类型作为参数或者返回类型的函数。毕竟编译器不知道不完全类型的定义,我们不能创建其类的任何对象,也不能声明成类内部的数据成员。
	\item  类内部的函数一般会自动内联。所以某函数一旦不需要内联,其定义就不要再放在头文件里,而是放到对应的 \cppinline{.cpp} 文件里。这样可以保持头文件的类相当精炼,也很好地贯彻了声明与定义分离的原则。
	\item  在 \cppinline{#include} 中插入空行以分割相关头文件, C 库, C++ 库, 其他库的 \cppinline{.h} 和本项目内的 \cppinline{.h} 是个好习惯。
\end{itemize}


\chapter{作用域}

\section{命名空间} \label{namespace}

\DBox{
  鼓励在 ``.cc`` 文件内使用匿名命名空间或 ``static`` 声明. 使用具名的命名空间时, 其名称可基于项目名或相对路径.
  禁止使用 using 指示(using-directive)。禁止使用内联命名空间(inline namespace)。
}

\textbf{定义:}

命名空间将全局作用域细分为独立的, 具名的作用域, 可有效防止全局作用域的命名冲突。

\textbf{优点:}

虽然类已经提供了(可嵌套的)命名轴线 (YuleFox 注: 将命名分割在不同类的作用域内), 命名空间在这基础上又封装了一层。

举例来说, 两个不同项目的全局作用域都有一个类 ``Foo``, 这样在编译或运行时造成冲突. 如果每个项目将代码置于不同命名空间中,
``project1::Foo`` 和 ``project2::Foo`` 作为不同符号自然不会冲突.

内联命名空间会自动把内部的标识符放到外层作用域,比如:

\begin{cppcode}
  namespace X {
      inline namespace Y {
          void foo();
        }  // namespace Y
    }  // namespace X
\end{cppcode}

``X::Y::foo()`` 与 ``X::foo()`` 彼此可代替。内联命名空间主要用来保持跨版本的 ABI 兼容性。

\textbf{缺点:}

命名空间具有迷惑性, 因为它们使得区分两个相同命名所指代的定义更加困难。

内联命名空间很容易令人迷惑,毕竟其内部的成员不再受其声明所在命名空间的限制。内联命名空间只在大型版本控制里有用。

有时候不得不多次引用某个定义在许多嵌套命名空间里的实体,使用完整的命名空间会导致代码的冗长。

在头文件中使用匿名空间导致违背 C++ 的唯一定义原则 (One Definition Rule (ODR))。

\textbf{结论:}

根据下文将要提到的策略合理使用命名空间。

\begin{itemize}
  \item 遵守 `命名空间命名 <naming.html#namespace-names>` 中的规则。
  \item 像之前的几个例子中一样,在命名空间的最后注释出命名空间的名字。
  \item 用命名空间把文件包含, `gflags <https://gflags.github.io/gflags/>` 的声明/定义, 以及类的前置声明以外的整个源文件封装起来, 以区别于其它命名空间:

        % \begin{noindent}
\begin{cppcode}
  // .h 文件
  namespace mynamespace {

      // 所有声明都置于命名空间中
      // 注意不要使用缩进
  class MyClass {
  public:
  ...
  void Foo();
  };
} // namespace mynamespace
\end{cppcode}
%\end{noindent}

        %\begin{noindent}
\begin{cppcode}
// .cc 文件
namespace mynamespace {

  // 函数定义都置于命名空间中
  void MyClass::Foo() {
  ...
  }

} // namespace mynamespace
\end{cppcode}
%\end{noindent}

        更复杂的 ``.cc`` 文件包含更多, 更复杂的细节, 比如 gflags 或 using 声明。

        %\begin{noindent}
\begin{cppcode}
  #include "a.h"

  DEFINE_FLAG(bool, someflag, false, "dummy flag");

  namespace a {

      ...code for a...// 左对齐

    } // namespace a
\end{cppcode}
% \end{noindent}

  \item 不要在命名空间 ``std`` 内声明任何东西, 包括标准库的类前置声明. 在 ``std`` 命名空间声明实体是未定义的行为,会导致如不可移植. 声明标准库下的实体, 需要包含对应的头文件。

  \item 不应该使用 *using 指示* 引入整个命名空间的标识符号。

        %\begin{noindent}
\begin{cppcode}
  // 禁止 —— 污染命名空间
  using namespace foo;
\end{cppcode}
% \end{noindent}

  \item  不要在头文件中使用 *命名空间别名* 除非显式标记内部命名空间使用。因为任何在头文件中引入的命名空间都会成为公开API的一部分。

        %\begin{noindent}
\begin{cppcode}
  // 在 .cc 中使用别名缩短常用的命名空间
  namespace baz = ::foo::bar::baz;
\end{cppcode}

\begin{cppcode}
// 在 .h 中使用别名缩短常用的命名空间
namespace librarian {
  namespace impl {  // 仅限内部使用
      namespace sidetable = ::pipeline_diagnostics::sidetable;
    }  // namespace impl

  inline void my_inline_function() {
    // 限制在一个函数中的命名空间别名
    namespace baz = ::foo::bar::baz;
    ...
  }
}  // namespace librarian
\end{cppcode}
% \end{noindent}

  \item  禁止用内联命名空间
\end{itemize}

\section{匿名命名空间和静态变量} \label{unnamed-namespace-and-static-variables}

\DBox {
  在 ``.cc`` 文件中定义一个不需要被外部引用的变量时,可以将它们放在匿名命名空间或声明为 ``static`` 。但是不要在
  ``.h`` 文件中这么做。
}

\textbf{定义:}

所有置于匿名命名空间的声明都具有内部链接性,函数和变量可以经由声明为 ``static`` 拥有内部链接性,这意味着你在这个文件中声明的这些标识符都不能在另一个文件中被访问。即使两个文件声明了完全一样名字的标识符,它们所指向的实体实际上是完全不同的。

\textbf{结论:}

推荐、鼓励在 ``.cc`` 中对于不需要在其他地方引用的标识符使用内部链接性声明,但是不要在 ``.h`` 中使用。

匿名命名空间的声明和具名的格式相同,在最后注释上 ``namespace`` :

%\begin{noindent}
\begin{cppcode}
      namespace {
      ...
      }  // namespace
\end{cppcode}
% \end{noindent}
\chapter{类}

类是 C++ 中代码的基本单元. 显然, 它们被广泛使用. 本节列举了在写一个类时的主要注意事项。

\section{构造函数的职责}

\textbf{总述:}

不要在构造函数中调用虚函数, 也不要在无法报出错误时进行可能失败的初始化.

\textbf{定义:}

在构造函数中可以进行各种初始化操作.

\textbf{优点:}

\begin{itemize}
  \item 无需考虑类是否被初始化
  \item 经过构造函数完全初始化后的对象可以为 ``const`` 类型, 也能更方便地被标准容器或算法使用
\end{itemize}

\textbf{缺点:}

\begin{itemize}
  \item 如果在构造函数内调用了自身的虚函数, 这类调用是不会重定向到子类的虚函数实现. 即使当前没有子类化实现, 将来仍是隐患。
  \item 在没有使程序崩溃 (因为并不是一个始终合适的方法) 或者使用异常 (因为已经被 :ref:`禁用 <exceptions>` 了) 等方法的条件下, 构造函数很难上报错误
  \item 如果执行失败, 会得到一个初始化失败的对象, 这个对象有可能进入不正常的状态, 必须使用 ``bool IsValid()`` 或类似这样的机制才能检查出来, 然而这是一个十分容易被疏忽的方法.
  \item 构造函数的地址是无法被取得的, 因此, 举例来说, 由构造函数完成的工作是无法以简单的方式交给其他线程的.
\end{itemize}

\textbf{结论:}

构造函数不允许调用虚函数. 如果代码允许, 直接终止程序是一个合适的处理错误的方式. 否则, 考虑用 ``Init()`` 方法或工厂函数.

构造函数不得调用虚函数, 或尝试报告一个非致命错误. 如果对象需要进行有意义的 (non-trivial) 初始化, 考虑使用明确的 Init() 方法或使用工厂模式. Avoid ``Init()`` methods on objects with no other states that affect which public methods may be called (此类形式的半构造对象有时无法正确工作).

\section{隐式类型转换} \label{implicit-conversions}

\textbf{总述:}

不要定义隐式类型转换. 对于转换运算符和单参数构造函数, 请使用 ``explicit`` 关键字.

\textbf{定义:}

隐式类型转换允许一个某种类型 (称作 *源类型*) 的对象被用于需要另一种类型 (称作 *目的类型*) 的位置, 例如, 将一个 ``int`` 类型的参数传递给需要 ``double`` 类型的函数.

除了语言所定义的隐式类型转换, 用户还可以通过在类定义中添加合适的成员定义自己需要的转换. 在源类型中定义隐式类型转换, 可以通过目的类型名的类型转换运算符实现 (例如 ``operator bool()``). 在目的类型中定义隐式类型转换, 则通过以源类型作为其唯一参数 (或唯一无默认值的参数) 的构造函数实现.

``explicit`` 关键字可以用于构造函数或 (在 C++11 引入) 类型转换运算符, 以保证只有当目的类型在调用点被显式写明时才能进行类型转换, 例如使用 ``cast``. 这不仅作用于隐式类型转换, 还能作用于 C++11 的列表初始化语法:

% \begin{noindent}
\begin{cppcode}
class Foo {
  explicit Foo(int x, double y);
  ...
};

void Func(Foo f);
\end{cppcode}
% \end{noindent}

此时下面的代码是不允许的:

% \begin{noindent}
  \begin{cppcode}
Func({42, 3.14});  // Error
\end{cppcode}
% \end{noindent}

这一代码从技术上说并非隐式类型转换, 但是语言标准认为这是 ``explicit`` 应当限制的行为.

\textbf{优点:}

\begin{itemize}
  \item 有时目的类型名是一目了然的, 通过避免显式地写出类型名, 隐式类型转换可以让一个类型的可用性和表达性更强.
  \item 隐式类型转换可以简单地取代函数重载.
  \item 在初始化对象时, 列表初始化语法是一种简洁明了的写法.
\end{itemize}

\textbf{缺点:}

\begin{itemize}
  \item 隐式类型转换会隐藏类型不匹配的错误. 有时, 目的类型并不符合用户的期望, 甚至用户根本没有意识到发生了类型转换.
  \item 隐式类型转换会让代码难以阅读, 尤其是在有函数重载的时候, 因为这时很难判断到底是哪个函数被调用.
  \item 单参数构造函数有可能会被无意地用作隐式类型转换.
  \item 如果单参数构造函数没有加上 ``explicit`` 关键字, 读者无法判断这一函数究竟是要作为隐式类型转换, 还是作者忘了加上 ``explicit`` 标记.
  \item 并没有明确的方法用来判断哪个类应该提供类型转换, 这会使得代码变得含糊不清.
  \item 如果目的类型是隐式指定的, 那么列表初始化会出现和隐式类型转换一样的问题, 尤其是在列表中只有一个元素的时候.
\end{itemize}

\textbf{结论:}

在类型定义中, 类型转换运算符和单参数构造函数都应当用 ``explicit`` 进行标记. 一个例外是, 拷贝和移动构造函数不应当被标记为 ``explicit``, 因为它们并不执行类型转换. 对于设计目的就是用于对其他类型进行透明包装的类来说, 隐式类型转换有时是必要且合适的. 这时应当联系项目组长并说明特殊情况.

不能以一个参数进行调用的构造函数不应当加上 ``explicit``. 接受一个 ``std::initializer\_list`` 作为参数的构造函数也应当省略 ``explicit``, 以便支持拷贝初始化 (例如 ``MyType m = {1, 2};``).

\section{可拷贝类型和可移动类型} \label{copyable-and-movable-types}

\textbf{总述:}

如果你的类型需要, 就让它们支持拷贝 / 移动. 否则, 就把隐式产生的拷贝和移动函数禁用.

\textbf{定义:}

可拷贝类型允许对象在初始化时得到来自相同类型的另一对象的值, 或在赋值时被赋予相同类型的另一对象的值, 同时不改变源对象的值. 对于用户定义的类型, 拷贝操作一般通过拷贝构造函数与拷贝赋值操作符定义. ``string`` 类型就是一个可拷贝类型的例子.

可移动类型允许对象在初始化时得到来自相同类型的临时对象的值, 或在赋值时被赋予相同类型的临时对象的值 (因此所有可拷贝对象也是可移动的). \cppinline{std::unique_ptr<int>} 就是一个可移动但不可复制的对象的例子. 对于用户定义的类型, 移动操作一般是通过移动构造函数和移动赋值操作符实现的.

拷贝 / 移动构造函数在某些情况下会被编译器隐式调用. 例如, 通过传值的方式传递对象.

\textbf{优点:}

可移动及可拷贝类型的对象可以通过传值的方式进行传递或者返回, 这使得 API 更简单, 更安全也更通用. 与传指针和引用不同, 这样的传递不会造成所有权, 生命周期, 可变性等方面的混乱, 也就没必要在协议中予以明确. 这同时也防止了客户端与实现在非作用域内的交互, 使得它们更容易被理解与维护. 这样的对象可以和需要传值操作的通用 API 一起使用, 例如大多数容器.

拷贝 / 移动构造函数与赋值操作一般来说要比它们的各种替代方案, 比如 ``Clone()``, ``CopyFrom()`` or ``Swap()``, 更容易定义, 因为它们能通过编译器产生, 无论是隐式的还是通过 ``= default``. 这种方式很简洁, 也保证所有数据成员都会被复制. 拷贝与移动构造函数一般也更高效, 因为它们不需要堆的分配或者是单独的初始化和赋值步骤, 同时, 对于类似\href{http://en.cppreference.com/w/cpp/language/copy_elision}{省略不必要的拷贝}这样的优化它们也更加合适.

移动操作允许隐式且高效地将源数据转移出右值对象. 这有时能让代码风格更加清晰.

\textbf{缺点:}

许多类型都不需要拷贝, 为它们提供拷贝操作会让人迷惑, 也显得荒谬而不合理. 单件类型 (``Registerer``), 与特定的作用域相关的类型 (``Cleanup``), 与其他对象实体紧耦合的类型 (``Mutex``) 从逻辑上来说都不应该提供拷贝操作. 为基类提供拷贝 / 赋值操作是有害的, 因为在使用它们时会造成 \href{https://en.wikipedia.org/wiki/Object_slicing}{对象切割} . 默认的或者随意的拷贝操作实现可能是不正确的, 这往往导致令人困惑并且难以诊断出的错误.

拷贝构造函数是隐式调用的, 也就是说, 这些调用很容易被忽略. 这会让人迷惑, 尤其是对那些所用的语言约定或强制要求传引用的程序员来说更是如此. 同时, 这从一定程度上说会鼓励过度拷贝, 从而导致性能上的问题.

\textbf{结论:}

如果需要就让你的类型可拷贝 / 可移动. 作为一个经验法则, 如果对于你的用户来说这个拷贝操作不是一眼就能看出来的, 那就不要把类型设置为可拷贝. 如果让类型可拷贝, 一定要同时给出拷贝构造函数和赋值操作的定义, 反之亦然. 如果让类型可移动, 同时移动操作的效率高于拷贝操作, 那么就把移动的两个操作 (移动构造函数和赋值操作) 也给出定义. 如果类型不可拷贝, 但是移动操作的正确性对用户显然可见, 那么把这个类型设置为只可移动并定义移动的两个操作.

如果定义了拷贝/移动操作, 则要保证这些操作的默认实现是正确的. 记得时刻检查默认操作的正确性, 并且在文档中说明类是可拷贝的且/或可移动的.

\begin{cppcode}
  class Foo {
      public:
      Foo(Foo&& other) : field_(other.field) {}
      // 差, 只定义了移动构造函数, 而没有定义对应的赋值运算符.

      private:
      Field field_;
    };
\end{cppcode}

由于存在对象切割的风险, 不要为任何有可能有派生类的对象提供赋值操作或者拷贝 / 移动构造函数 (当然也不要继承有这样的成员函数的类). 如果你的基类需要可复制属性, 请提供一个 ``public virtual Clone()`` 和一个 ``protected`` 的拷贝构造函数以供派生类实现.

如果你的类不需要拷贝 / 移动操作, 请显式地通过在 ``public`` 域中使用 ``= delete`` 或其他手段禁用之.

\begin{cppcode}
  // MyClass is neither copyable nor movable.
  MyClass(const MyClass&) = delete;
  MyClass& operator=(const MyClass&) = delete;
\end{cppcode}

\section{结构体 VS. 类} \label{structs-vs-classes}

\textbf{总述:}

仅当只有数据成员时使用 ``struct``, 其它一概使用 ``class``.

\textbf{说明:}

在 C++ 中 ``struct`` 和 ``class`` 关键字几乎含义一样. 我们为这两个关键字添加我们自己的语义理解, 以便为定义的数据类型选择合适的关键字.

``struct`` 用来定义包含数据的被动式对象, 也可以包含相关的常量, 但除了存取数据成员之外, 没有别的函数功能. 并且存取功能是通过直接访问位域, 而非函数调用. 除了构造函数, 析构函数, ``Initialize()``, ``Reset()``, ``Validate()`` 等类似的用于设定数据成员的函数外, 不能提供其它功能的函数.

如果需要更多的函数功能, ``class`` 更适合. 如果拿不准, 就用 ``class``.

为了和 STL 保持一致, 对于仿函数等特性可以不用 ``class`` 而是使用 ``struct``.

注意: 类和结构体的成员变量使用不同的 :ref:`命名规则 <variable-names>`.

\section{继承} \label{inheritance}

\textbf{总述:}

使用组合 (YuleFox 注: 这一点也是 GoF 在 <<Design Patterns>> 里反复强调的) 常常比使用继承更合理. 如果使用继承的话, 定义为 ``public`` 继承.

\textbf{定义:}

当子类继承基类时, 子类包含了父基类所有数据及操作的定义. C++ 实践中, 继承主要用于两种场合: 实现继承, 子类继承父类的实现代码; :ref:`接口继承 <interface>`, 子类仅继承父类的方法名称.

\textbf{优点:}

实现继承通过原封不动的复用基类代码减少了代码量. 由于继承是在编译时声明, 程序员和编译器都可以理解相应操作并发现错误. 从编程角度而言, 接口继承是用来强制类输出特定的 API. 在类没有实现 API 中某个必须的方法时, 编译器同样会发现并报告错误.

\textbf{缺点:}

对于实现继承, 由于子类的实现代码散布在父类和子类间之间, 要理解其实现变得更加困难. 子类不能重写父类的非虚函数, 当然也就不能修改其实现. 基类也可能定义了一些数据成员, 因此还必须区分基类的实际布局.

\textbf{结论:}

所有继承必须是 ``public`` 的. 如果你想使用私有继承, 你应该替换成把基类的实例作为成员对象的方式.

不要过度使用实现继承. 组合常常更合适一些. 尽量做到只在 "是一个" ("is-a", YuleFox 注: 其他 "has-a" 情况下请使用组合) 的情况下使用继承: 如果 ``Bar`` 的确 "是一种" ``Foo``, ``Bar`` 才能继承 ``Foo``.

必要的话, 析构函数声明为 ``virtual``. 如果你的类有虚函数, 则析构函数也应该为虚函数.

对于可能被子类访问的成员函数, 不要过度使用 ``protected`` 关键字. 注意, 数据成员都必须是 :ref:`私有的 <access-control>`.

对于重载的虚函数或虚析构函数, 使用 ``override``, 或 (较不常用的) ``final`` 关键字显式地进行标记. 较早 (早于 C++11) 的代码可能会使用 ``virtual`` 关键字作为不得已的选项. 因此, 在声明重载时, 请使用 ``override``, ``final`` 或 ``virtual`` 的其中之一进行标记. 标记为 ``override`` 或 ``final`` 的析构函数如果不是对基类虚函数的重载的话, 编译会报错, 这有助于捕获常见的错误. 这些标记起到了文档的作用, 因为如果省略这些关键字, 代码阅读者不得不检查所有父类, 以判断该函数是否是虚函数.

\section{多重继承} \label{multiple-inheritance}

\textbf{总述:}

真正需要用到多重实现继承的情况少之又少. 只在以下情况我们才允许多重继承: 最多只有一个基类是非抽象类; 其它基类都是以 ``Interface`` 为后缀的 :ref:`纯接口类 <interface>`.

\textbf{定义:}

多重继承允许子类拥有多个基类. 要将作为 *纯接口* 的基类和具有 *实现* 的基类区别开来.

\textbf{优点:}

相比单继承 (见 :ref:`继承 <inheritance>`), 多重实现继承可以复用更多的代码.

\textbf{缺点:}

真正需要用到多重 *实现* 继承的情况少之又少. 有时多重实现继承看上去是不错的解决方案, 但这时你通常也可以找到一个更明确, 更清晰的不同解决方案.

\textbf{结论:}

只有当所有父类除第一个外都是 :ref:`纯接口类 <interface>` 时, 才允许使用多重继承. 为确保它们是纯接口, 这些类必须以 ``Interface`` 为后缀.

\textbf{注意:}

关于该规则, Windows 下有个 :ref:`特例 <windows-code>`.

\section{接口} \label{interface}

\textbf{总述:}

接口是指满足特定条件的类, 这些类以 ``Interface`` 为后缀 (不强制).

\textbf{定义:}

当一个类满足以下要求时, 称之为纯接口:

- 只有纯虚函数 ("``=0``") 和静态函数 (除了下文提到的析构函数).

- 没有非静态数据成员.

- 没有定义任何构造函数. 如果有, 也不能带有参数, 并且必须为 ``protected``.

- 如果它是一个子类, 也只能从满足上述条件并以 ``Interface`` 为后缀的类继承.

接口类不能被直接实例化, 因为它声明了纯虚函数. 为确保接口类的所有实现可被正确销毁, 必须为之声明虚析构函数 (作为上述第 1 条规则的特例, 析构函数不能是纯虚函数). 具体细节可参考 Stroustrup 的 *The C++ Programming Language, 3rd edition* 第 12.4 节.

\textbf{优点:}

以 ``Interface`` 为后缀可以提醒其他人不要为该接口类增加函数实现或非静态数据成员. 这一点对于 :ref:`多重继承 <multiple-inheritance>` 尤其重要. 另外, 对于 Java 程序员来说, 接口的概念已是深入人心.

\textbf{缺点:}

``Interface`` 后缀增加了类名长度, 为阅读和理解带来不便. 同时, 接口属性作为实现细节不应暴露给用户.

\textbf{结论:}

只有在满足上述条件时, 类才以 ``Interface`` 结尾, 但反过来, 满足上述需要的类未必一定以 ``Interface`` 结尾.

\section{运算符重载}

\textbf{总述:}

除少数特定环境外, 不要重载运算符. 也不要创建用户定义字面量.

\textbf{定义:}

C++ 允许用户通过使用 ``operator`` 关键字 \href{http://en.cppreference.com/w/cpp/language/operators}{对内建运算符进行重载定义 } , 只要其中一个参数是用户定义的类型. ``operator`` 关键字还允许用户使用 ``operator""`` 定义新的字面运算符, 并且定义类型转换函数, 例如 ``operator bool()``.

\textbf{优点:}

重载运算符可以让代码更简洁易懂, 也使得用户定义的类型和内建类型拥有相似的行为. 重载运算符对于某些运算来说是符合语言习惯的名称 (例如 ``==``, ``<``, ``=``, ``<<``), 遵循这些语言约定可以让用户定义的类型更易读, 也能更好地和需要这些重载运算符的函数库进行交互操作.

对于创建用户定义的类型的对象来说, 用户定义字面量是一种非常简洁的标记.

\textbf{缺点:}

- 要提供正确, 一致, 不出现异常行为的操作符运算需要花费不少精力, 而且如果达不到这些要求的话, 会导致令人迷惑的 Bug.

- 过度使用运算符会带来难以理解的代码, 尤其是在重载的操作符的语义与通常的约定不符合时.

- 函数重载有多少弊端, 运算符重载就至少有多少.

- 运算符重载会混淆视听, 让你误以为一些耗时的操作和操作内建类型一样轻巧.

- 对重载运算符的调用点的查找需要的可就不仅仅是像 grep 那样的程序了, 这时需要能够理解 C++ 语法的搜索工具.

- 如果重载运算符的参数写错, 此时得到的可能是一个完全不同的重载而非编译错误. 例如: ``foo < bar`` 执行的是一个行为, 而 ``\&foo < \&bar`` 执行的就是完全不同的另一个行为了.

- 重载某些运算符本身就是有害的. 例如, 重载一元运算符 ``\&`` 会导致同样的代码有完全不同的含义, 这取决于重载的声明对某段代码而言是否是可见的. 重载诸如 ``\&\&``, ``||`` 和 ``,`` 会导致运算顺序和内建运算的顺序不一致.

- 运算符从通常定义在类的外部, 所以对于同一运算, 可能出现不同的文件引入了不同的定义的风险. 如果两种定义都链接到同一二进制文件, 就会导致未定义的行为, 有可能表现为难以发现的运行时错误.

- 用户定义字面量所创建的语义形式对于某些有经验的 C++ 程序员来说都是很陌生的.

\textbf{结论:}

只有在意义明显, 不会出现奇怪的行为并且与对应的内建运算符的行为一致时才定义重载运算符. 例如, ``|`` 要作为位或或逻辑或来使用, 而不是作为 shell 中的管道.

只有对用户自己定义的类型重载运算符. 更准确地说, 将它们和它们所操作的类型定义在同一个头文件中, ``.cc`` 中和命名空间中. 这样做无论类型在哪里都能够使用定义的运算符, 并且最大程度上避免了多重定义的风险. 如果可能的话, 请避免将运算符定义为模板, 因为此时它们必须对任何模板参数都能够作用. 如果你定义了一个运算符, 请将其相关且有意义的运算符都进行定义, 并且保证这些定义的语义是一致的. 例如, 如果你重载了 ``<``, 那么请将所有的比较运算符都进行重载, 并且保证对于同一组参数, ``<`` 和 ``>`` 不会同时返回 ``true``.

建议不要将不进行修改的二元运算符定义为成员函数. 如果一个二元运算符被定义为类成员, 这时隐式转换会作用域右侧的参数却不会作用于左侧. 这时会出现 ``a < b`` 能够通过编译而 ``b < a`` 不能的情况, 这是很让人迷惑的.

不要为了避免重载操作符而走极端. 比如说, 应当定义 ``==``, ``=``, 和 ``<<`` 而不是 ``Equals()``, ``CopyFrom()`` 和 ``PrintTo()``. 反过来说, 不要只是为了满足函数库需要而去定义运算符重载. 比如说, 如果你的类型没有自然顺序, 而你要将它们存入 ``std::set`` 中, 最好还是定义一个自定义的比较运算符而不是重载 ``<``.

不要重载 ``\&\&``, ``||``, ``,`` 或一元运算符 ``\&``. 不要重载 ``operator""``, 也就是说, 不要引入用户定义字面量.

类型转换运算符在 :ref:`隐式类型转换 <implicit-conversions>` 一节有提及. ``=`` 运算符在 :ref:`可拷贝类型和可移动类型 <copyable-and-movable-types>` 一节有提及. 运算符 ``<<`` 在 :ref:`流 <streams>` 一节有提及. 同时请参见 :ref:`函数重载 <function-overloading>` 一节, 其中提到的的规则对运算符重载同样适用.

\section{存取控制} \label{access-control}

\textbf{总述:}

将 *所有* 数据成员声明为 ``private``, 除非是 ``static const`` 类型成员 (遵循 :ref:`常量命名规则 <constant-names>`). 出于技术上的原因, 在使用 \href{https://github.com/google/googletest}{Google Test} 时我们允许测试固件类中的数据成员为 ``protected``.

\section{声明顺序} \label{declaration-order}

\textbf{总述:}

将相似的声明放在一起, 将 ``public`` 部分放在最前.

\textbf{说明:}

类定义一般应以 ``public:`` 开始, 后跟 ``protected:``, 最后是 ``private:``. 省略空部分.

在各个部分中, 建议将类似的声明放在一起, 并且建议以如下的顺序: 类型 (包括 ``typedef``, ``using`` 和嵌套的结构体与类), 常量, 工厂函数, 构造函数, 赋值运算符, 析构函数, 其它函数, 数据成员.

不要将大段的函数定义内联在类定义中. 通常,只有那些普通的, 或性能关键且短小的函数可以内联在类定义中. 参见 :ref:`内联函数 <inline-functions>` 一节.

\section{注解}

\subsection{ 译者 (YuleFox) 笔记}

\begin{itemize}
  \item 不在构造函数中做太多逻辑相关的初始化;
  \item 编译器提供的默认构造函数不会对变量进行初始化, 如果定义了其他构造函数, 编译器不再提供, 需要编码者自行提供默认构造函数;
  \item 为避免隐式转换, 需将单参数构造函数声明为 ``explicit``;
  \item 为避免拷贝构造函数, 赋值操作的滥用和编译器自动生成, 可将其声明为 ``private`` 且无需实现;
  \item 仅在作为数据集合时使用 ``struct``;
  \item 组合 > 实现继承 > 接口继承 > 私有继承, 子类重载的虚函数也要声明 ``virtual`` 关键字, 虽然编译器允许不这样做;
  \item 避免使用多重继承, 使用时, 除一个基类含有实现外, 其他基类均为纯接口;
  \item 接口类类名以 ``Interface`` 为后缀, 除提供带实现的虚析构函数, 静态成员函数外, 其他均为纯虚函数, 不定义非静态数据成员, 不提供构造函数, 提供的话, 声明为 ``protected``;
  \item 为降低复杂性, 尽量不重载操作符, 模板, 标准类中使用时提供文档说明;
  \item 存取函数一般内联在头文件中;
  \item 声明次序: ``public`` -> ``protected`` -> ``private``;
  \item 函数体尽量短小, 紧凑, 功能单一;
\end{itemize}
\chapter{函数}

\section{参数顺序}

\textbf{总述}

函数的参数顺序为: 输入参数在先, 后跟输出参数。

\textbf{说明}

C/C++ 中的函数参数或者是函数的输入, 或者是函数的输出, 或兼而有之。 输入参数通常是值参或 \cppinline{const} 引用, 输出参数或输入/输出参数则一般为非 \cppinline{const} 指针。 在排列参数顺序时, 将所有的输入参数置于输出参数之前。 特别要注意, 在加入新参数时不要因为它们是新参数就置于参数列表最后, 而是仍然要按照前述的规则, 即将新的输入参数也置于输出参数之前。

这并非一个硬性规定。 输入/输出参数 (通常是类或结构体) 让这个问题变得复杂。 并且, 有时候为了其他函数保持一致, 你可能不得不有所变通。

\section{编写简短函数}

\textbf{总述}

我们倾向于编写简短, 凝练的函数。

\textbf{说明}

我们承认长函数有时是合理的, 因此并不硬性限制函数的长度。 如果函数超过 40 行, 可以思索一下能不能在不影响程序结构的前提下对其进行分割。

即使一个长函数现在工作的非常好, 一旦有人对其修改, 有可能出现新的问题, 甚至导致难以发现的 bug。 使函数尽量简短, 以便于他人阅读和修改代码。

在处理代码时, 你可能会发现复杂的长函数。 不要害怕修改现有代码: 如果证实这些代码使用 / 调试起来很困难, 或者你只需要使用其中的一小段代码, 考虑将其分割为更加简短并易于管理的若干函数。

\section{引用参数}

\textbf{总述}

所有按引用传递的参数必须加上 \cppinline{const}。

\textbf{定义}

在 C 语言中, 如果函数需要修改变量的值, 参数必须为指针, 如 \cppinline{int foo(int *pval)}。 在 C++ 中, 函数还可以声明为引用参数: \cppinline{int foo(int &val)}。

\textbf{优点}

定义引用参数可以防止出现 \cppinline{(*pval)++} 这样丑陋的代码。 引用参数对于拷贝构造函数这样的应用也是必需的。 同时也更明确地不接受空指针。

\textbf{缺点}

容易引起误解, 因为引用在语法上是值变量却拥有指针的语义。

\textbf{结论}

函数参数列表中, 所有引用参数都必须是 \cppinline{const}:

\begin{cppcode}
  void Foo(const string &in, string *out);
\end{cppcode}

事实上这在 Google Code 是一个硬性约定: 输入参数是值参或 \cppinline{const} 引用, 输出参数为指针。 输入参数可以是 \cppinline{const} 指针, 但决不能是非 \cppinline{const} 的引用参数, 除非特殊要求, 比如 \cppinline{swap()}。

有时候, 在输入形参中用 \cppinline{const T*} 指针比 \cppinline{const T&} 更明智。 比如:

* 可能会传递空指针。

* 函数要把指针或对地址的引用赋值给输入形参。

总而言之, 大多时候输入形参往往是 \cppinline{const T&}。 若用 \cppinline{const T*} 则说明输入另有处理。 所以若要使用 \cppinline{const T*}, 则应给出相应的理由, 否则会使得读者感到迷惑。

\section{函数重载} \label{function-overloading}

\textbf{总述}

若要使用函数重载, 则必须能让读者一看调用点就胸有成竹, 而不用花心思猜测调用的重载函数到底是哪一种。 这一规则也适用于构造函数。

\textbf{定义}

你可以编写一个参数类型为 \cppinline{const string&} 的函数, 然后用另一个参数类型为 \cppinline{const char*} 的函数对其进行重载:


\begin{cppcode}
  class MyClass {
      public:
      void Analyze(const string &text);
      void Analyze(const char *text, size_t textlen);
    };
\end{cppcode}

\textbf{优点}

通过重载参数不同的同名函数, 可以令代码更加直观。 模板化代码需要重载, 这同时也能为使用者带来便利。

\textbf{缺点}

如果函数单靠不同的参数类型而重载 (acgtyrant 注:这意味着参数数量不变), 读者就得十分熟悉 C++ 五花八门的匹配规则, 以了解匹配过程具体到底如何。 另外, 如果派生类只重载了某个函数的部分变体, 继承语义就容易令人困惑。

\textbf{结论}

如果打算重载一个函数, 可以试试改在函数名里加上参数信息。 例如, 用 \cppinline{AppendString()} 和 \cppinline{AppendInt()} 等, 而不是一口气重载多个 \cppinline{Append()}。 如果重载函数的目的是为了支持不同数量的同一类型参数, 则优先考虑使用 \cppinline{std::vector} 以便使用者可以用 \DFullRef{braced-initializer-list} 指定参数。

\section{缺省参数}

\textbf{总述}

只允许在非虚函数中使用缺省参数, 且必须保证缺省参数的值始终一致。 缺省参数与 \DFullRef{function-overloading} 遵循同样的规则。 一般情况下建议使用函数重载, 尤其是在缺省函数带来的可读性提升不能弥补下文中所提到的缺点的情况下。

\textbf{优点}

有些函数一般情况下使用默认参数, 但有时需要又使用非默认的参数。 缺省参数为这样的情形提供了便利, 使程序员不需要为了极少的例外情况编写大量的函数。 和函数重载相比, 缺省参数的语法更简洁明了, 减少了大量的样板代码, 也更好地区别了 "必要参数" 和 "可选参数"。

\textbf{缺点}

缺省参数实际上是函数重载语义的另一种实现方式, 因此所有 \DFullRef{function-overloading} 也都适用于缺省参数。

虚函数调用的缺省参数取决于目标对象的静态类型, 此时无法保证给定函数的所有重载声明的都是同样的缺省参数。

缺省参数是在每个调用点都要进行重新求值的, 这会造成生成的代码迅速膨胀。 作为读者, 一般来说也更希望缺省的参数在声明时就已经被固定了, 而不是在每次调用时都可能会有不同的取值。

缺省参数会干扰函数指针, 导致函数签名与调用点的签名不一致。 而函数重载不会导致这样的问题。

\textbf{结论}

对于虚函数, 不允许使用缺省参数, 因为在虚函数中缺省参数不一定能正常工作。 如果在每个调用点缺省参数的值都有可能不同, 在这种情况下缺省函数也不允许使用。 (例如, 不要写像 \cppinline{void f(int n = counter++);} 这样的代码。)

在其他情况下, 如果缺省参数对可读性的提升远远超过了以上提及的缺点的话, 可以使用缺省参数。 如果仍有疑惑, 就使用函数重载。

\section{函数返回类型后置语法}

\textbf{总述}

只有在常规写法 (返回类型前置) 不便于书写或不便于阅读时使用返回类型后置语法。

\textbf{定义}

C++ 现在允许两种不同的函数声明方式。 以往的写法是将返回类型置于函数名之前。 例如:

\begin{cppcode}
int foo(int x);
\end{cppcode}

C++11 引入了这一新的形式。 现在可以在函数名前使用 \cppinline{auto} 关键字, 在参数列表之后后置返回类型。 例如:

\begin{cppcode}
auto foo(int x) -> int;
\end{cppcode}

后置返回类型为函数作用域。 对于像 \cppinline{int} 这样简单的类型, 两种写法没有区别。 但对于复杂的情况, 例如类域中的类型声明或者以函数参数的形式书写的类型, 写法的不同会造成区别。

\textbf{优点}

后置返回类型是显式地指定 \DFullRef{lambda-expressions} 的返回值的唯一方式。 某些情况下, 编译器可以自动推导出 Lambda 表达式的返回类型, 但并不是在所有的情况下都能实现。 即使编译器能够自动推导, 显式地指定返回类型也能让读者更明了。

有时在已经出现了的函数参数列表之后指定返回类型, 能够让书写更简单, 也更易读, 尤其是在返回类型依赖于模板参数时。 例如:

\begin{cppcode}
  template <class T, class U> auto add(T t, U u) -> decltype(t + u);
\end{cppcode}

对比下面的例子:

\begin{cppcode}
  template <class T, class U> decltype(declval<T&>() + declval<U&>()) add(T t, U u);
\end{cppcode}

\textbf{缺点}

后置返回类型相对来说是非常新的语法, 而且在 C 和 Java 中都没有相似的写法, 因此可能对读者来说比较陌生。

在已有的代码中有大量的函数声明, 你不可能把它们都用新的语法重写一遍。 因此实际的做法只能是使用旧的语法或者新旧混用。 在这种情况下, 只使用一种版本是相对来说更规整的形式。

\textbf{结论}

在大部分情况下, 应当继续使用以往的函数声明写法, 即将返回类型置于函数名前。 只有在必需的时候 (如 Lambda 表达式) 或者使用后置语法能够简化书写并且提高易读性的时候才使用新的返回类型后置语法。 但是后一种情况一般来说是很少见的, 大部分时候都出现在相当复杂的模板代码中, 而多数情况下不鼓励写这样 \DFullRef{template-metaprogramming}。

\chapter{来自 Google 的奇技}

Google 用了很多自己实现的技巧 / 工具使 C++ 代码更加健壮, 我们使用 C++ 的方式可能和你在其它地方见到的有所不同。

\section{所有权与智能指针}

\textbf{ 总述}

动态分配出的对象最好有单一且固定的所有主, 并通过智能指针传递所有权。

\textbf{ 定义}

所有权是一种登记/管理动态内存和其它资源的技术。 动态分配对象的所有主是一个对象或函数, 后者负责确保当前者无用时就自动销毁前者。 所有权有时可以共享, 此时就由最后一个所有主来负责销毁它。 甚至也可以不用共享, 在代码中直接把所有权传递给其它对象。

智能指针是一个通过重载 \cppinline{*} 和 \cppinline{->} 运算符以表现得如指针一样的类。 智能指针类型被用来自动化所有权的登记工作, 来确保执行销毁义务到位。
\href{http://en.cppreference.com/w/cpp/memory/unique_ptr}{std::unique\_ptr} 是 C++11 新推出的一种智能指针类型, 用来表示动态分配出的对象的独一无二的所有权; 当 \cppinline{std::unique_ptr} 离开作用域时, 对象就会被销毁。 \cppinline{std::unique_ptr} 不能被复制, 但可以把它移动(move)给新所有主。\href{http://en.cppreference.com/w/cpp/memory/shared_ptr}{std::shared\_ptr} 同样表示动态分配对象的所有权, 但可以被共享, 也可以被复制; 对象的所有权由所有复制者共同拥有, 最后一个复制者被销毁时, 对象也会随着被销毁。

\textbf{ 优点}

* 如果没有清晰、逻辑条理的所有权安排, 不可能管理好动态分配的内存。

* 传递对象的所有权, 开销比复制来得小, 如果可以复制的话。

* 传递所有权也比"借用"指针或引用来得简单, 毕竟它大大省去了两个用户一起协调对象生命周期的工作。

* 如果所有权逻辑条理, 有文档且不紊乱的话, 可读性会有很大提升。

* 可以不用手动完成所有权的登记工作, 大大简化了代码, 也免去了一大波错误之恼。

* 对于 const 对象来说, 智能指针简单易用, 也比深度复制高效。

\textbf{ 缺点}

* 不得不用指针(不管是智能的还是原生的)来表示和传递所有权。 指针语义可要比值语义复杂得许多了, 特别是在 API 里:这时不光要操心所有权, 还要顾及别名, 生命周期, 可变性以及其它大大小小的问题。

* 其实值语义的开销经常被高估, 所以所有权传递带来的性能提升不一定能弥补可读性和复杂度的损失。

* 如果 API 依赖所有权的传递, 就会害得客户端不得不用单一的内存管理模型。

* 如果使用智能指针, 那么资源释放发生的位置就会变得不那么明显。

* \cppinline{std::unique_ptr} 的所有权传递原理是 C++11 的 move 语法, 后者毕竟是刚刚推出的, 容易迷惑程序员。

* 如果原本的所有权设计已经够完善了, 那么若要引入所有权共享机制, 可能不得不重构整个系统。

* 所有权共享机制的登记工作在运行时进行, 开销可能相当大。

* 某些极端情况下 (例如循环引用), 所有权被共享的对象永远不会被销毁。

* 智能指针并不能够完全代替原生指针。

\textbf{ 结论}

如果必须使用动态分配, 那么更倾向于将所有权保持在分配者手中。 如果其他地方要使用这个对象, 最好传递它的拷贝, 或者传递一个不用改变所有权的指针或引用。 倾向于使用 \cppinline{std::unique_ptr} 来明确所有权传递, 例如:

\begin{cppcode}
  std::unique_ptr<Foo> FooFactory();
  void FooConsumer(std::unique_ptr<Foo> ptr);
\end{cppcode}

如果没有很好的理由, 则不要使用共享所有权。 这里的理由可以是为了避免开销昂贵的拷贝操作, 但是只有当性能提升非常明显, 并且操作的对象是不可变的(比如说 \cppinline{std::shared_ptr<const Foo>} )时候, 才能这么做。 如果确实要使用共享所有权, 建议于使用 \cppinline{std::shared_ptr} 。

不要使用 \cppinline{std::auto_ptr}, 使用 \cppinline{std::unique_ptr} 代替它。

\section{Cpplint}

\textbf{ 总述}

使用 \cppinline{cpplint.py} 检查风格错误。

\textbf{ 说明}

\cppinline{cpplint.py} 是一个用来分析源文件, 能检查出多种风格错误的工具。 它不并完美, 甚至还会漏报和误报, 但它仍然是一个非常有用的工具。 在行尾加 \cppinline{// NOLINT}, 或在上一行加 \cppinline{// NOLINTNEXTLINE}, 可以忽略报错。

某些项目会指导你如何使用他们的项目工具运行 \cppinline{cpplint.py}。 如果你参与的项目没有提供, 你可以单独下载 \href{http://github.com/google/styleguide/blob/gh-pages/cpplint/cpplint.py}{cpplint.py}。

\section{注解}

\subsection{译者(acgtyrant)笔记}

\begin{itemize}
  \item 把智能指针当成对象来看待的话, 就很好领会它与所指对象之间的关系了。
  \item 原来 Rust 的 Ownership 思想是受到了 C++ 智能指针的很大启发啊。
  \item \cppinline{scoped_ptr} 和 \cppinline{auto_ptr} 已过时。  现在是 \cppinline{shared_ptr} 和 \cppinline{uniqued_ptr} 的天下了。
  \item 按本文来说, 似乎除了智能指针, 还有其它所有权机制, 值得留意。
  \item Arch Linux 用户注意了, AUR 有对 cpplint 打包。
\end{itemize}

\input{qt/others.tex}
\chapter{命名约定}

最重要的一致性规则是命名管理。命名的风格能让我们在不需要去查找类型声明的条件下快速地了解某个名字代表的含义: 类型,变量,函数,常量,宏,等等,甚至 我们大脑中的模式匹配引擎非常依赖这些命名规则。\textbf{命名规则具有一定随意性,但相比按个人喜好命名,一致性更重要,所以无论你认为它们是否重要,规则总归是规则。}

\section{通用命名规则} \label{general-naming-rules}

\textbf{总述}

函数命名,变量命名,文件命名要有描述性; 少用缩写。

\textbf{说明}

尽可能使用描述性的命名,别心疼空间,毕竟相比之下让代码易于新读者理解更重要。不要用只有项目开发者能理解的缩写,也不要通过砍掉几个字母来缩写单词.

\begin{cppcode}
  int priceCountReader;     // 无缩写
  int numErrors;            // "num" 是一个常见的写法
  int numDnsConnections;    // 人人都知道 "DNS" 是什么
\end{cppcode}

\begin{cppcode}
  int n;                     // 毫无意义.
  int nerr;                  // 含糊不清的缩写.
  int nCompConns;            // 含糊不清的缩写.
  int wgcConnections;        // 只有贵团队知道是什么意思.
  int pcReader;              // "pc" 有太多可能的解释了.
  int cstmrID;               // 删减了若干字母.
\end{cppcode}

注意,一些特定的广为人知的缩写是允许的,例如用 \cppinline{i} 表示迭代变量和用 \cppinline{T} 表示模板参数。

\begin{DWarn}
  在D-Pointer风格中,\cppinline{d_ptr,q_ptr,dd_ptr,qq_ptr}都是保留的名称。
\end{DWarn}

模板参数的命名应当遵循对应的分类: 类型模板参数应当遵循 \DFullRef{type-names} 的规则,而非类型模板应当遵循  \DFullRef{variable-names} 的规则.

\section{文件命名}

\textbf{总述}

文件名要全部小写,可以包含下划线 (\cppinline{_}) 。如果存在无法\cppinline{_}的情况,可以考虑使用连字符\cppinline{-},否则不允许例外。

\begin{DWarn}
  Qt默认情况下不使用任何连接符合,这使得文件名非常难以看懂,我们不接受这种风格。deepin的Qt项目统一使用下划线(\cppinline{_})作为文件名连接符合。
\end{DWarn}

\textbf{说明}

可接受的文件命名示例:

\begin{cppcode}
  my_useful_class.cpp
  myusefulclass_test.cpp // \cppinline{_unittest} 和 \cppinline{_regtest} 已弃用.
\end{cppcode}

不接受的文件命名示例:

\begin{cppcode}
  my-useful-class.cpp  // 不接受,除非无法使用_
  myusefulclass.cpp    // 不接受,难以看懂
\end{cppcode}

C++ 文件要以 \cppinline{.cpp} 结尾,头文件以 \cppinline{.h} 结尾。专门插入文本的文件则以 \cppinline{.inc} 结尾,参见 \DFullRef{self-contained-headers}。

不要使用已经存在于 \cppinline{/usr/include} 下的文件名 (Yang.Y 注: 即编译器搜索系统头文件的路径),如 \cppinline{db.h}。

通常应尽量让文件名更加明确。\cppinline{http_server_logs.h} 就比 \cppinline{logs.h} 要好。定义类时文件名一般成对出现,如 \cppinline{foo_bar.h} 和 \cppinline{foo_bar.cpp},对应于类 \cppinline{FooBar}。

内联函数必须放在 \cppinline{.h} 文件中。如果内联函数比较短,就直接放在 \cppinline{.h} 中.

\section{类型命名} \label{type-names}

\textbf{总述}

类型名称的每个单词首字母均大写,不包含下划线: \cppinline{MyExcitingClass},\cppinline{MyExcitingEnum}。

\textbf{说明}

所有类型命名 —— 类,结构体,类型定义 (\cppinline{typedef}),枚举,类型模板参数 —— 均使用相同约定,即以大写字母开始,每个单词首字母均大写,不包含下划线。例如:

\begin{cppcode}
  // 类和结构体
  class UrlTable { ...
  class UrlTableTester { ...
  struct UrlTableProperties { ...

  // 类型定义
  typedef hash_map<UrlTableProperties *, string> PropertiesMap;

  // using 别名
  using PropertiesMap = hash_map<UrlTableProperties *, string>;

  // 枚举
  enum UrlTableErrors { ...
\end{cppcode}

\section{变量命名} \label{variable-names}

\textbf{总述}

\begin{DWarn}
  变量 (包括函数参数) 和数据成员名一律使用驼峰命名。

  在D-Pointer的Private类中,成员变量不加任何修饰。

  在一般的类中,使用\cppinline{m_}开头来标记成员变量。
\end{DWarn}

\textbf{说明}

\subsection{普通变量命名}

举例:

\begin{cppcode}
  QString tableName;   // 接受,驼峰命名

  QString table_name;  // 不接受 - 用下划线.
  QString tablename;   // 不接受 - 全小写.
\end{cppcode}

\subsection{类数据成员}

不管是静态的还是非静态的,类数据成员都可以和普通变量一样,但是需要使用\cppinline{m_}前缀来修饰。

\begin{DWarn}
  为了实现较好的封装,Qt中大量使用D-Pointer技术,在这种情况下,一般通过\cppinline{d->localValue}的方式访问Private类的变量,这时候就不需要使用\cppinline{m_}来修饰成员变量。
  对于非D-Pointer的Private类,使用\cppinline{m_}前缀来修饰成员变量。
\end{DWarn}


\begin{cppcode}
  class TableInfo {
      ...
      private:
      QString m_tableName;               // 好
      static Pool<TableInfo>* m_pool;    // 好
    };

  class TableInfoPrivate {
      ...
      public:
      QString tableName;               // 好,Private类不需要任何修饰
      static Pool<TableInfo>* pool;    // 好,Private类不需要任何修饰
    };
\end{cppcode}

\subsection{结构体变量}

不管是静态的还是非静态的,结构体数据成员都可以和普通变量一样,不用像类那样接下划线:

\begin{cppcode}
  struct TableInfoData {
      QString tableName;               // 好,命名风格和Private保持一致
      static Pool<TableInfo>* pool;    // 好,命名风格和Private保持一致
    }
\end{cppcode}

结构体与类的使用讨论,参考 \DFullRef{structs-vs-classes}。

\section{常量命名} \label{constant-names}

\textbf{总述}

声明为 \cppinline{constexpr} 或 \cppinline{const} 的变量,或在程序运行期间其值始终保持不变的,命名时以 "k" 开头,大小写混合。例如:

\begin{cppcode}
  const int kDaysInAWeek = 7;
\end{cppcode}

\textbf{说明}

所有具有静态存储类型的变量 (例如静态变量或全局变量,参见 \href{http://en.cppreference.com/w/cpp/language/storage_duration#Storage_duration}{存储类型}) 都应当以此方式命名。对于其他存储类型的变量,如自动变量等,这条规则是可选的。如果不采用这条规则,就按照一般的变量命名规则。

\section{函数命名} \label{function-names}

\textbf{总述}

常规函数使用大小写混合,取值和设值函数则要求与变量名匹配: \cppinline{MyExcitingFunction()},\cppinline{MyExcitingMethod()},\cppinline{my_exciting_member_variable()},\cppinline{set_my_exciting_member_variable()}。

\textbf{说明}

一般来说,函数名的每个单词首字母大写 (即 "驼峰变量名" 或 "帕斯卡变量名"),没有下划线。对于首字母缩写的单词,更倾向于将它们视作一个单词进行首字母大写 (例如,写作 \cppinline{StartRpc()} 而非 \cppinline{StartRPC()})。

\begin{cppcode}
  AddTableEntry()
  DeleteUrl()
  OpenFileOrDie()
\end{cppcode}

\begin{DWarn}
  对于DBus接口函数,属性,信号,确保首字母大写,这也适用于一些其他风格的IPC/RPC接口或代码生成器生成的接口,包括dbus/protobuf/thrift等。
\end{DWarn}

同样的命名规则同时适用于类作用域与命名空间作用域的常量,因为它们是作为 API 的一部分暴露对外的,因此应当让它们看起来像是一个函数,因为在这时,它们实际上是一个对象而非函数的这一事实对外不过是一个无关紧要的实现细节。

取值和设值函数的命名与变量一致。一般来说它们的名称与实际的成员变量对应,但并不强制要求。例如 \cppinline{int getCount()} 与 \cppinline{void setCount(int count)}。

\section{命名空间命名}

\textbf{总述}
\begin{DWarn}
  命名空间以大写字母命名。最高级命名空间的名字取决于项目名称。要注意避免嵌套命名空间的名字之间和常见的顶级命名空间的名字之间发生冲突.
\end{DWarn}

顶级命名空间的名称应当是项目名或者是该命名空间中的代码所属的团队的名字。命名空间中的代码,应当存放于和命名空间的名字匹配的文件夹或其子文件夹中.

注意 \DFullRef{general-naming-rules} 的规则同样适用于命名空间。命名空间中的代码极少需要涉及命名空间的名称,因此没有必要在命名空间中使用缩写.

要避免嵌套的命名空间与常见的顶级命名空间发生名称冲突。由于名称查找规则的存在,命名空间之间的冲突完全有可能导致编译失败。尤其是,不要创建嵌套的 \cppinline{std} 命名空间。建议使用更独特的项目标识符 (\cppinline{WebSearch::Index},\cppinline{WebSearch::IndexUtil}) 而非常见的极易发生冲突的名称 (比如 \cppinline{WebSearch::Util}).

对于 \cppinline{Internal} 命名空间,要当心加入到同一 \cppinline{internal} 命名空间的代码之间发生冲突 (由于内部维护人员通常来自同一团队,因此常有可能导致冲突)。在这种情况下,请使用文件名以使得内部名称独一无二 (例如对于 \cppinline{frobber.h},使用 \cppinline{WebSearch::Index::FrobberInternal})。

\section{枚举命名}

\textbf{总述}

枚举的命名应当和 \DFullRef{type-names} 一致: \cppinline{EnumName} 。

\textbf{说明}

单独的枚举值使用首字母大写的大小写混合命名方式。枚举名 \cppinline{UrlTableErrors} (以及 \cppinline{AlternateUrlTableErrors}) 是类型,所以要用大小写混合的方式.

\begin{cppcode}
  enum UrlTableErrors {
      OK = 0,
      ErrorOutOfMemory,
      ErrorMalformedInput,
    };
\end{cppcode}

2009 年 1 月之前,Google一直建议采用 \DFullRef{macro-names} 的方式命名枚举值。由于枚举值和宏之间的命名冲突,直接导致了很多问题。由此,这里改为优先选择常量风格的命名方式。新代码应该尽可能优先使用常量风格。但是老代码没必要切换到常量风格,除非宏风格确实会产生编译期问题.

\section{宏命名} \label{macro-names}

\textbf{总述}

你并不打算 \DFullRef{preprocessor-macros},对吧? 如果你一定要用,像这样命名:

\cppinline{MY_MACRO_THAT_SCARES_SMALL_CHILDREN}.

\textbf{说明}

参考 \DFullRef{preprocessor-macros}; 通常 *不应该* 使用宏。如果不得不用,其命名像枚举命名一样全部大写,使用下划线:

\begin{cppcode}
  #define ROUND(x) ...
  #define PI_ROUNDED 3.0
\end{cppcode}

\section{命名规则的特例}

\textbf{总述}

如果你命名的实体与已有 C/C++ 实体相似,可参考现有命名策略。

\begin{DWarn}
  如果是为了扩展STL的接口,或继承其他底层库的函数,则可以不受命名规则限制,以避免功能错误。
\end{DWarn}

\cppinline{bigopen()}: 函数名,参照 \cppinline{open()} 的形式

\cppinline{uint}: \cppinline{typedef}

\cppinline{bigpos}: \cppinline{struct} 或 \cppinline{class},参照 \cppinline{pos} 的形式

\cppinline{sparse_hash_map}: STL 型实体; 参照 STL 命名约定

\cppinline{LONGLONG_MAX}: 常量,如同 \cppinline{INT_MAX}

\section{注解}

\subsection{ 译者 (YuleFox) 笔记}

感觉 Google 的命名约定很高明,比如写了简单的类 QueryResult,接着又可以直接定义一个变量 \cppinline{query_result},区分度很好; 再次,类内变量以下划线结尾,那么就可以直接传入同名的形参,比如 \cppinline{TextQuery::TextQuery(std::string word) : word_(word) {}} ,其中 \cppinline{word_} 自然是类内私有成员.

\subsection{ deepin风格注解 }

对Qt风格同理:

比如写了简单的类 \cppinline{QueryResult},接着又可以直接定义一个变量 \cppinline{queryResult},区分度很好; 再次,类内变量以下划线结尾,那么就可以直接传入同名的形参,比如 \cppinline{TextQuery::TextQuery(QString word) : m_word(word) {}} ,其中 \cppinline{m_word} 自然是类内私有成员.

\chapter{注释}

注释虽然写起来很痛苦, 但对保证代码可读性至关重要。下面的规则描述了如何注释以及在哪儿注释。当然也要记住: 注释固然很重要, 但最好的代码应当本身就是文档。有意义的类型名和变量名, 要远胜过要用注释解释的含糊不清的名字.

你写的注释是给代码读者看的, 也就是下一个需要理解你的代码的人。所以慷慨些吧, 下一个读者可能就是你!

\section{注释风格}

\textbf{总述}

使用 \cppinline{//} 或 \cppinline{/* */}, 统一就好.

\textbf{说明}

\cppinline{//} 或 \cppinline{/* */} 都可以; 但 \cppinline{//} *更* 常用。要在如何注释及注释风格上确保统一。

\section{文件注释}

\textbf{总述}

在每一个文件开头加入版权公告.

文件注释描述了该文件的内容。如果一个文件只声明, 或实现, 或测试了一个对象, 并且这个对象已经在它的声明处进行了详细的注释, 那么就没必要再加上文件注释。除此之外的其他文件都需要文件注释.

\textbf{说明}

\subsection{法律公告和作者信息}

请使用SPDX格式的许可信息,而不是复制许可文本,即如下的\cppinline{SPDX-License-Identifier: GPL-3.0-or-later}:

% \begin{noindent}
\begin{cppcode}
/*
* Copyright (c) 2021. deepin All rights reserved.
*
* Author:     Iceyer <me@iceyer.net>
*
* Maintainer: Iceyer <me@iceyer.net>
*
* SPDX-License-Identifier: GPL-3.0-or-later
*/
\end{cppcode}
% \end{noindent}

每个文件都应该包含许可证引用。为项目选择合适的许可证版本(比如, Apache 2.0, BSD, LGPL, GPL)。

如果你对原始作者的文件做了重大修改, 请考虑删除原作者信息。

\subsection{文件内容}

如果一个 \cppinline{.h} 文件声明了多个概念, 则文件注释应当对文件的内容做一个大致的说明, 同时说明各概念之间的联系。一个一到两行的文件注释就足够了, 对于每个概念的详细文档应当放在各个概念中, 而不是文件注释中。

不要在 \cppinline{.h} 和 \cppinline{.cpp} 之间复制注释, 这样的注释偏离了注释的实际意义。

\label{class-comments}

\section{类注释}

\textbf{总述}

每个类的定义都要附带一份注释, 描述类的功能和用法, 除非它的功能相当明显.

% \begin{noindent}
\begin{cppcode}
// Iterates over the contents of a GargantuanTable.
// Example:
//    GargantuanTableIterator* iter = table->NewIterator();
//    for (iter->Seek("foo"); !iter->done(); iter->Next()) {
//      process(iter->key(), iter->value());
//    }
//    delete iter;
class GargantuanTableIterator {
  ...
};
\end{cppcode}
% \end{noindent}

\textbf{说明}

类注释应当为读者理解如何使用与何时使用类提供足够的信息, 同时应当提醒读者在正确使用此类时应当考虑的因素。如果类有任何同步前提, 请用文档说明。如果该类的实例可被多线程访问, 要特别注意文档说明多线程环境下相关的规则和常量使用。

如果你想用一小段代码演示这个类的基本用法或通常用法, 放在类注释里也非常合适。

如果类的声明和定义分开了(例如分别放在了 \cppinline{.h} 和 \cppinline{.cc} 文件中), 此时, 描述类用法的注释应当和接口定义放在一起, 描述类的操作和实现的注释应当和实现放在一起。

\section{函数注释}

\textbf{总述}

函数声明处的注释描述函数功能; 定义处的注释描述函数实现.

\textbf{说明}

\subsection{函数声明}

基本上每个函数声明处前都应当加上注释, 描述函数的功能和用途。只有在函数的功能简单而明显时才能省略这些注释(例如, 简单的取值和设值函数)。注释使用叙述式 ("Opens the file") 而非指令式 ("Open the file"); 注释只是为了描述函数, 而不是命令函数做什么。通常, 注释不会描述函数如何工作。那是函数定义部分的事情.

函数声明处注释的内容:

- 函数的输入输出.

- 对类成员函数而言: 函数调用期间对象是否需要保持引用参数, 是否会释放这些参数.

- 函数是否分配了必须由调用者释放的空间.

- 参数是否可以为空指针.

- 是否存在函数使用上的性能隐患.

- 如果函数是可重入的, 其同步前提是什么?

举例如下:

% \begin{noindent}
\begin{cppcode}
// Returns an iterator for this table.  It is the client's
// responsibility to delete the iterator when it is done with it,
// and it must not use the iterator once the GargantuanTable object
// on which the iterator was created has been deleted.
//
// The iterator is initially positioned at the beginning of the table.
//
// This method is equivalent to:
//    Iterator* iter = table->NewIterator();
//    iter->Seek("");
//    return iter;
// If you are going to immediately seek to another place in the
// returned iterator, it will be faster to use NewIterator()
// and avoid the extra seek.
  Iterator* GetIterator() const;
\end{cppcode}
% \end{noindent}

但也要避免罗罗嗦嗦, 或者对显而易见的内容进行说明. 下面的注释就没有必要加上 "否则返回 false", 因为已经暗含其中了:

% \begin{noindent}
\begin{cppcode}
// Returns true if the table cannot hold any more entries.
bool IsTableFull();
\end{cppcode}
% \end{noindent}

注释函数重载时, 注释的重点应该是函数中被重载的部分, 而不是简单的重复被重载,的函数的注释。多数情况下, 函数重载不需要额外的文档, 因此也没有必要加上注释。

注释构造/析构函数时, 切记读代码的人知道构造/析构函数的功能, 所以 "销毁这一对象" 这样的注释是没有意义的。你应当注明的是注明构造函数对参数做了什么 (例如, 是否取得指针所有权) 以及析构函数清理了什么。如果都是些无关紧要的内容, 直接省掉注释。析构函数前没有注释是很正常的。

\subsection{函数定义}

如果函数的实现过程中用到了很巧妙的方式, 那么在函数定义处应当加上解释性的注释。例如, 你所使用的编程技巧, 实现的大致步骤, 或解释如此实现的理由。举个例子, 你可以说明为什么函数的前半部分要加锁而后半部分不需要。

\textbf{不要} 从 \cppinline{.h} 文件或其他地方的函数声明处直接复制注释。简要重述函数功能是可以的, 但注释重点要放在如何实现上。

\section{变量注释}

\textbf{总述}

通常变量名本身足以很好说明变量用途。某些情况下, 也需要额外的注释说明.

\textbf{说明}

\subsection{类数据成员}

每个类数据成员 (也叫实例变量或成员变量) 都应该用注释说明用途。如果有非变量的参数(例如特殊值, 数据成员之间的关系, 生命周期等)不能够用类型与变量名明确表达, 则应当加上注释。然而, 如果变量类型与变量名已经足以描述一个变量, 那么就不再需要加上注释。

特别地, 如果变量可以接受 \cppinline{NULL} 或 \cppinline{-1} 等警戒值, 须加以说明。比如:

% \begin{noindent}
\begin{cppcode}
private:
// Used to bounds-check table accesses. -1 means
// that we don't yet know how many entries the table has.
int num_total_entries_;
\end{cppcode}
% \end{noindent}

\subsection{全局变量}

和数据成员一样, 所有全局变量也要注释说明含义及用途, 以及作为全局变量的原因。比如:

% \begin{noindent}
\begin{cppcode}
// The total number of tests cases that we run through in this regression test.
const int kNumTestCases = 6;
\end{cppcode}
% \end{noindent}

\section{实现注释}

\textbf{总述}

对于代码中巧妙的, 晦涩的, 有趣的, 重要的地方加以注释.

\textbf{说明}

\subsection{代码前注释}

巧妙或复杂的代码段前要加注释。比如:

% \begin{noindent}
\begin{cppcode}
// Divide result by two, taking into account that x
// contains the carry from the add.
for (int i = 0; i < result->size(); i++) {
    x = (x << 8) + (*result)[i];
    (*result)[i] = x >> 1;
    x &= 1;
  }
\end{cppcode}
% \end{noindent}

\subsection{行注释}

比较隐晦的地方要在行尾加入注释。在行尾空两格进行注释。比如:

% \begin{noindent}
\begin{cppcode}
// If we have enough memory, mmap the data portion too.
mmap_budget = max<int64>(0, mmap_budget - index_->length());
if (mmap_budget >= data_size_ && !MmapData(mmap_chunk_bytes, mlock))
return;  // Error already logged.
\end{cppcode}
% \end{noindent}

注意, 这里用了两段注释分别描述这段代码的作用, 和提示函数返回时错误已经被记入日志。

如果你需要连续进行多行注释,可以使之对齐获得更好的可读性:

% \begin{noindent}
\begin{cppcode}
DoSomething();                  // Comment here so the comments line up.
DoSomethingElseThatIsLonger();  // Two spaces between the code and the comment.
  { // One space before comment when opening a new scope is allowed,
    // thus the comment lines up with the following comments and code.
    DoSomethingElse();  // Two spaces before line comments normally.
  }
std::vector<string> list{
    // Comments in braced lists describe the next element...
    "First item",
    // .. and should be aligned appropriately.
    "Second item"};
DoSomething(); /* For trailing block comments, one space is fine. */
\end{cppcode}
% \end{noindent}

\subsection{函数参数注释}

如果函数参数的意义不明显, 考虑用下面的方式进行弥补:

\begin{itemize}
  \item 如果参数是一个字面常量, 并且这一常量在多处函数调用中被使用, 用以推断它们一致, 你应当用一个常量名让这一约定变得更明显, 并且保证这一约定不会被打破。
  \item 考虑更改函数的签名, 让某个 \cppinline{bool} 类型的参数变为 \cppinline{enum} 类型, 这样可以让这个参数的值表达其意义。
  \item 如果某个函数有多个配置选项, 你可以考虑定义一个类或结构体以保存所有的选项, 并传入类或结构体的实例。这样的方法有许多优点, 例如这样的选项可以在调用处用变量名引用, 这样就能清晰地表明其意义。同时也减少了函数参数的数量, 使得函数调用更易读也易写。除此之外, 以这样的方式, 如果你使用其他的选项, 就无需对调用点进行更改.
  \item 用具名变量代替大段而复杂的嵌套表达式.
  \item 万不得已时, 才考虑在调用点用注释阐明参数的意义.
\end{itemize}

比如下面的示例的对比:

% \begin{noindent}
\begin{cppcode}
// What are these arguments?
const DecimalNumber product = CalculateProduct(values, 7, false, nullptr);
\end{cppcode}
% \end{noindent}

和

% \begin{noindent}
\begin{cppcode}
ProductOptions options;
options.set_precision_decimals(7);
options.set_use_cache(ProductOptions::kDontUseCache);
const DecimalNumber product =
CalculateProduct(values, options, /*completion_callback=*/nullptr);
\end{cppcode}
% \end{noindent}

哪个更清晰一目了然。

\subsection{不允许的行为}

不要描述显而易见的现象,\textbf{永远不要}用自然语言翻译代码作为注释, 除非即使对深入理解 C++ 的读者来说代码的行为都是不明显的。要假设读代码的人 C++ 水平比你高, 即便他/她可能不知道你的用意:

你所提供的注释应当解释代码\textbf{为什么}要这么做和代码的目的, 或者最好是让代码自文档化。

比较这样的注释:

% \begin{noindent}
\begin{cppcode}
  // Find the element in the vector.  <-- 差: 这太明显了!
  auto iter = std::find(v.begin(), v.end(), element);
  if (iter != v.end()) {
      Process(element);
    }
\end{cppcode}
% \end{noindent}

和这样的注释:

% \begin{noindent}
\begin{cppcode}
  // Process "element" unless it was already processed.
  auto iter = std::find(v.begin(), v.end(), element);
  if (iter != v.end()) {
      Process(element);
    }
\end{cppcode}
% \end{noindent}

自文档化的代码根本就不需要注释。上面例子中的注释对下面的代码来说就是毫无必要的:


% \begin{noindent}
\begin{cppcode}
  if (!IsAlreadyProcessed(element)) {
      Process(element);
    }
\end{cppcode}
% \end{noindent}

\section{标点, 拼写和语法}

\textbf{总述}

注意标点, 拼写和语法; 写的好的注释比差的要易读的多。

\textbf{说明}

注释的通常写法是包含正确大小写和结尾句号的完整叙述性语句。大多数情况下, 完整的句子比句子片段可读性更高。短一点的注释, 比如代码行尾注释, 可以随意点, 但依然要注意风格的一致性。

虽然被别人指出该用分号时却用了逗号多少有些尴尬, 但清晰易读的代码还是很重要的。正确的标点, 拼写和语法对此会有很大帮助。

\section{TODO 注释}

\textbf{总述}

对那些临时的, 短期的解决方案, 或已经够好但仍不完美的代码使用 \cppinline{TODO} 注释。

\cppinline{TODO} 注释要使用全大写的字符串 \cppinline{TODO}, 在随后的圆括号里写上你的名字, 邮件地址, bug ID, 或其它身份标识和与这一 \cppinline{TODO} 相关的 issue。主要目的是让添加注释的人 (也是可以请求提供更多细节的人) 可根据规范的 \cppinline{TODO} 格式进行查找。添加 \cppinline{TODO} 注释并不意味着你要自己来修正, 因此当你加上带有姓名的 \cppinline{TODO} 时, 一般都是写上自己的名字.

% \begin{noindent}
\begin{cppcode}
  // TODO(kl@gmail.com): Use a "*" here for concatenation operator.
  // TODO(Zeke) change this to use relations.
  // TODO(bug 12345): remove the "Last visitors" feature
\end{cppcode}
% \end{noindent}

如果加 \cppinline{TODO} 是为了在 "将来某一天做某事", 可以附上一个非常明确的时间 "Fix by November 2005"), 或者一个明确的事项 ("Remove this code when all clients can handle XML responses.").

\section{弃用注释}

\textbf{总述}

通过弃用注释(\cppinline{DEPRECATED} comments)以标记某接口点已弃用.

您可以写上包含全大写的 \cppinline{DEPRECATED} 的注释, 以标记某接口为弃用状态。注释可以放在接口声明前, 或者同一行.

在 \cppinline{DEPRECATED} 一词后, 在括号中留下您的名字, 邮箱地址以及其他身份标识.

弃用注释应当包涵简短而清晰的指引, 以帮助其他人修复其调用点。在 C++ 中, 你可以将一个弃用函数改造成一个内联函数, 这一函数将调用新的接口.

仅仅标记接口为 \cppinline{DEPRECATED} 并不会让大家不约而同地弃用, 您还得亲自主动修正调用点(callsites), 或是找个帮手.

修正好的代码应该不会再涉及弃用接口点了, 着实改用新接口点。如果您不知从何下手, 可以找标记弃用注释的当事人一起商量。

\section{注解}

\subsection{译者 (YuleFox) 笔记}

\begin{itemize}
  \item 关于注释风格, 很多 C++ 的 coders 更喜欢行注释, C coders 或许对块注释依然情有独钟, 或者在文件头大段大段的注释时使用块注释;
  \item 文件注释可以炫耀你的成就, 也是为了捅了篓子别人可以找你;
  \item 注释要言简意赅, 不要拖沓冗余, 复杂的东西简单化和简单的东西复杂化都是要被鄙视的;
  \item 对于 Chinese coders 来说, 用英文注释还是用中文注释, it is a problem, 但不管怎样, 注释是为了让别人看懂, 难道是为了炫耀编程语言之外的你的母语或外语水平吗;
  \item 注释不要太乱, 适当的缩进才会让人乐意看。但也没有必要规定注释从第几列开始 (我自己写代码的时候总喜欢这样), UNIX/LINUX 下还可以约定是使用 tab 还是 space, 个人倾向于 space。
  \item TODO 很不错, 有时候, 注释确实是为了标记一些未完成的或完成的不尽如人意的地方, 这样一搜索, 就知道还有哪些活要干, 日志都省了。
\end{itemize}

% TODO(iceyer): use clang-format
% 9. 格式
------------

每个人都可能有自己的代码风格和格式, 但如果一个项目中的所有人都遵循同一风格的话, 这个项目就能更顺利地进行. 每个人未必能同意下述的每一处格式规则, 而且其中的不少规则需要一定时间的适应, 但整个项目服从统一的编程风格是很重要的, 只有这样才能让所有人轻松地阅读和理解代码.

为了帮助你正确的格式化代码, 我们写了一个 `emacs 配置文件 <https://raw.githubusercontent.com/google/styleguide/gh-pages/google-c-style.el>`_.

.. _line-length:

9.1. 行长度
~~~~~~~~~~~~~~~~~~~~

**总述**

每一行代码字符数不超过 80.

我们也认识到这条规则是有争议的, 但很多已有代码都遵照这一规则, 因此我们感觉一致性更重要.

**优点**

提倡该原则的人认为强迫他们调整编辑器窗口大小是很野蛮的行为. 很多人同时并排开几个代码窗口, 根本没有多余的空间拉伸窗口. 大家都把窗口最大尺寸加以限定, 并且 80 列宽是传统标准. 那么为什么要改变呢?

**缺点**

反对该原则的人则认为更宽的代码行更易阅读. 80 列的限制是上个世纪 60 年代的大型机的古板缺陷; 现代设备具有更宽的显示屏, 可以很轻松地显示更多代码.

**结论**

80 个字符是最大值.

如果无法在不伤害易读性的条件下进行断行, 那么注释行可以超过 80 个字符, 这样可以方便复制粘贴. 例如, 带有命令示例或 URL 的行可以超过 80 个字符.

包含长路径的 ``#include`` 语句可以超出80列.

:ref:`头文件保护 <define-guard>` 可以无视该原则.

9.2. 非 ASCII 字符
~~~~~~~~~~~~~~~~~~~~~~~~~~~~~~~~

**总述**

尽量不使用非 ASCII 字符, 使用时必须使用 UTF-8 编码.

**说明**

即使是英文, 也不应将用户界面的文本硬编码到源代码中, 因此非 ASCII 字符应当很少被用到. 特殊情况下可以适当包含此类字符. 例如, 代码分析外部数据文件时, 可以适当硬编码数据文件中作为分隔符的非 ASCII 字符串; 更常见的是 (不需要本地化的) 单元测试代码可能包含非 ASCII 字符串. 此类情况下, 应使用 UTF-8 编码, 因为很多工具都可以理解和处理 UTF-8 编码.

十六进制编码也可以, 能增强可读性的情况下尤其鼓励 —— 比如 ``"\xEF\xBB\xBF"``, 或者更简洁地写作 ``u8"\uFEFF"``, 在 Unicode 中是 *零宽度 无间断* 的间隔符号, 如果不用十六进制直接放在 UTF-8 格式的源文件中, 是看不到的.

(Yang.Y 注: ``"\xEF\xBB\xBF"`` 通常用作 UTF-8 with BOM 编码标记)

使用 ``u8`` 前缀把带 ``uXXXX`` 转义序列的字符串字面值编码成 UTF-8. 不要用在本身就带 UTF-8 字符的字符串字面值上, 因为如果编译器不把源代码识别成 UTF-8, 输出就会出错.

别用 C++11 的 ``char16_t`` 和 ``char32_t``, 它们和 UTF-8 文本没有关系, ``wchar_t`` 同理, 除非你写的代码要调用 Windows API, 后者广泛使用了 ``wchar_t``.

9.3. 空格还是制表位
~~~~~~~~~~~~~~~~~~~~~~~~~~~~~~~~

**总述**

只使用空格, 每次缩进 2 个空格.

**说明**

我们使用空格缩进. 不要在代码中使用制表符. 你应该设置编辑器将制表符转为空格.

9.4. 函数声明与定义
~~~~~~~~~~~~~~~~~~~~~~~~~~~~~~~~

**总述**

返回类型和函数名在同一行, 参数也尽量放在同一行, 如果放不下就对形参分行, 分行方式与 :ref:`函数调用 <function-calls>` 一致.

**说明**

函数看上去像这样:

.. code-block:: c++

    ReturnType ClassName::FunctionName(Type par_name1, Type par_name2) {
      DoSomething();
      ...
    }

如果同一行文本太多, 放不下所有参数:

.. code-block:: c++

    ReturnType ClassName::ReallyLongFunctionName(Type par_name1, Type par_name2,
                                                 Type par_name3) {
      DoSomething();
      ...
    }

甚至连第一个参数都放不下:

.. code-block:: c++

    ReturnType LongClassName::ReallyReallyReallyLongFunctionName(
        Type par_name1,  // 4 space indent
        Type par_name2,
        Type par_name3) {
      DoSomething();  // 2 space indent
      ...
    }

注意以下几点:

- 使用好的参数名.

- 只有在参数未被使用或者其用途非常明显时, 才能省略参数名.

- 如果返回类型和函数名在一行放不下, 分行.

- 如果返回类型与函数声明或定义分行了, 不要缩进.

- 左圆括号总是和函数名在同一行.

- 函数名和左圆括号间永远没有空格.

- 圆括号与参数间没有空格.

- 左大括号总在最后一个参数同一行的末尾处, 不另起新行.

- 右大括号总是单独位于函数最后一行, 或者与左大括号同一行.

- 右圆括号和左大括号间总是有一个空格.

- 所有形参应尽可能对齐.

- 缺省缩进为 2 个空格.

- 换行后的参数保持 4 个空格的缩进.

未被使用的参数, 或者根据上下文很容易看出其用途的参数, 可以省略参数名:

.. code-block:: c++

    class Foo {
     public:
      Foo(Foo&&);
      Foo(const Foo&);
      Foo& operator=(Foo&&);
      Foo& operator=(const Foo&);
    };

未被使用的参数如果其用途不明显的话, 在函数定义处将参数名注释起来:

.. code-block:: c++

    class Shape {
     public:
      virtual void Rotate(double radians) = 0;
    };

    class Circle : public Shape {
     public:
      void Rotate(double radians) override;
    };

    void Circle::Rotate(double /*radians*/) {}

.. code-block:: c++

    // 差 - 如果将来有人要实现, 很难猜出变量的作用.
    void Circle::Rotate(double) {}

属性, 和展开为属性的宏, 写在函数声明或定义的最前面, 即返回类型之前:

.. code-block:: c++

    MUST_USE_RESULT bool IsOK();

9.5. Lambda 表达式
~~~~~~~~~~~~~~~~~~~~~~~~~~~~~~~~~~

**总述**

Lambda 表达式对形参和函数体的格式化和其他函数一致; 捕获列表同理, 表项用逗号隔开.

**说明**

若用引用捕获, 在变量名和 ``&`` 之间不留空格.

.. code-block:: c++

    int x = 0;
    auto add_to_x = [&x](int n) { x += n; };

短 lambda 就写得和内联函数一样.

.. code-block:: c++

    std::set<int> blacklist = {7, 8, 9};
    std::vector<int> digits = {3, 9, 1, 8, 4, 7, 1};
    digits.erase(std::remove_if(digits.begin(), digits.end(), [&blacklist](int i) {
                   return blacklist.find(i) != blacklist.end();
                 }),
                 digits.end());

.. _function-calls:

9.6. 函数调用
~~~~~~~~~~~~~~~~~~~~~~

**总述**

要么一行写完函数调用, 要么在圆括号里对参数分行, 要么参数另起一行且缩进四格. 如果没有其它顾虑的话, 尽可能精简行数, 比如把多个参数适当地放在同一行里.

**说明**

函数调用遵循如下形式:

.. code-block:: c++

    bool retval = DoSomething(argument1, argument2, argument3);

如果同一行放不下, 可断为多行, 后面每一行都和第一个实参对齐, 左圆括号后和右圆括号前不要留空格:

.. code-block:: c++

    bool retval = DoSomething(averyveryveryverylongargument1,
                              argument2, argument3);

参数也可以放在次行, 缩进四格:

.. code-block:: c++

    if (...) {
      ...
      ...
      if (...) {
        DoSomething(
            argument1, argument2,  // 4 空格缩进
            argument3, argument4);
      }

把多个参数放在同一行以减少函数调用所需的行数, 除非影响到可读性. 有人认为把每个参数都独立成行, 不仅更好读, 而且方便编辑参数. 不过, 比起所谓的参数编辑, 我们更看重可读性, 且后者比较好办:

如果一些参数本身就是略复杂的表达式, 且降低了可读性, 那么可以直接创建临时变量描述该表达式, 并传递给函数:

.. code-block:: c++

    int my_heuristic = scores[x] * y + bases[x];
    bool retval = DoSomething(my_heuristic, x, y, z);

或者放着不管, 补充上注释:

.. code-block:: c++

    bool retval = DoSomething(scores[x] * y + bases[x],  // Score heuristic.
                              x, y, z);

如果某参数独立成行, 对可读性更有帮助的话, 那也可以如此做. 参数的格式处理应当以可读性而非其他作为最重要的原则.

此外, 如果一系列参数本身就有一定的结构, 可以酌情地按其结构来决定参数格式:

.. code-block:: c++

    // 通过 3x3 矩阵转换 widget.
    my_widget.Transform(x1, x2, x3,
                        y1, y2, y3,
                        z1, z2, z3);

.. _braced-initializer-list-format:

9.7. 列表初始化格式
~~~~~~~~~~~~~~~~~~~~~~~~~~~~~~~~

**总述**

您平时怎么格式化函数调用, 就怎么格式化 :ref:`列表初始化 <braced-initializer-list>`.

**说明**

如果列表初始化伴随着名字, 比如类型或变量名, 格式化时将将名字视作函数调用名, `{}` 视作函数调用的括号. 如果没有名字, 就视作名字长度为零.

.. code-block:: c++

    // 一行列表初始化示范.
    return {foo, bar};
    functioncall({foo, bar});
    pair<int, int> p{foo, bar};

    // 当不得不断行时.
    SomeFunction(
        {"assume a zero-length name before {"},  // 假设在 { 前有长度为零的名字.
        some_other_function_parameter);
    SomeType variable{
        some, other, values,
        {"assume a zero-length name before {"},  // 假设在 { 前有长度为零的名字.
        SomeOtherType{
            "Very long string requiring the surrounding breaks.",  // 非常长的字符串, 前后都需要断行.
            some, other values},
        SomeOtherType{"Slightly shorter string",  // 稍短的字符串.
                      some, other, values}};
    SomeType variable{
        "This is too long to fit all in one line"};  // 字符串过长, 因此无法放在同一行.
    MyType m = {  // 注意了, 您可以在 { 前断行.
        superlongvariablename1,
        superlongvariablename2,
        {short, interior, list},
        {interiorwrappinglist,
         interiorwrappinglist2}};

9.8. 条件语句
~~~~~~~~~~~~~~~~~~~~~~

**总述**

倾向于不在圆括号内使用空格. 关键字 ``if`` 和 ``else`` 另起一行.

**说明**

对基本条件语句有两种可以接受的格式. 一种在圆括号和条件之间有空格, 另一种没有.

最常见的是没有空格的格式. 哪一种都可以, 最重要的是 *保持一致*. 如果你是在修改一个文件, 参考当前已有格式. 如果是写新的代码, 参考目录下或项目中其它文件. 还在犹豫的话, 就不要加空格了.

.. code-block:: c++

    if (condition) {  // 圆括号里没有空格.
      ...  // 2 空格缩进.
    } else if (...) {  // else 与 if 的右括号同一行.
      ...
    } else {
      ...
    }

如果你更喜欢在圆括号内部加空格:

.. code-block:: c++

    if ( condition ) {  // 圆括号与空格紧邻 - 不常见
      ...  // 2 空格缩进.
    } else {  // else 与 if 的右括号同一行.
      ...
    }

注意所有情况下 ``if`` 和左圆括号间都有个空格. 右圆括号和左大括号之间也要有个空格:

.. code-block:: c++

    if(condition)     // 差 - IF 后面没空格.
    if (condition){   // 差 - { 前面没空格.
    if(condition){    // 变本加厉地差.

.. code-block:: c++

    if (condition) {  // 好 - IF 和 { 都与空格紧邻.

如果能增强可读性, 简短的条件语句允许写在同一行. 只有当语句简单并且没有使用 ``else`` 子句时使用:

.. code-block:: c++

    if (x == kFoo) return new Foo();
    if (x == kBar) return new Bar();

如果语句有 ``else`` 分支则不允许:

.. code-block:: c++

    // 不允许 - 当有 ELSE 分支时 IF 块却写在同一行
    if (x) DoThis();
    else DoThat();

通常, 单行语句不需要使用大括号, 如果你喜欢用也没问题; 复杂的条件或循环语句用大括号可读性会更好. 也有一些项目要求 ``if`` 必须总是使用大括号:

.. code-block:: c++

    if (condition)
      DoSomething();  // 2 空格缩进.

    if (condition) {
      DoSomething();  // 2 空格缩进.
    }

但如果语句中某个 ``if-else`` 分支使用了大括号的话, 其它分支也必须使用:

.. code-block:: c++

    // 不可以这样子 - IF 有大括号 ELSE 却没有.
    if (condition) {
      foo;
    } else
      bar;

    // 不可以这样子 - ELSE 有大括号 IF 却没有.
    if (condition)
      foo;
    else {
      bar;
    }


.. code-block:: c++

    // 只要其中一个分支用了大括号, 两个分支都要用上大括号.
    if (condition) {
      foo;
    } else {
      bar;
    }

9.9. 循环和开关选择语句
~~~~~~~~~~~~~~~~~~~~~~~~~~~~~~~~~~~~~~

**总述**

``switch`` 语句可以使用大括号分段, 以表明 cases 之间不是连在一起的. 在单语句循环里, 括号可用可不用. 空循环体应使用 ``{}`` 或 ``continue``.

**说明**

``switch`` 语句中的 ``case`` 块可以使用大括号也可以不用, 取决于你的个人喜好. 如果用的话, 要按照下文所述的方法.

如果有不满足 ``case`` 条件的枚举值, ``switch`` 应该总是包含一个 ``default`` 匹配 (如果有输入值没有 case 去处理, 编译器将给出 warning). 如果 ``default`` 应该永远执行不到, 简单的加条 ``assert``:

.. code-block:: c++

    switch (var) {
      case 0: {  // 2 空格缩进
        ...      // 4 空格缩进
        break;
      }
      case 1: {
        ...
        break;
      }
      default: {
        assert(false);
      }
    }

在单语句循环里, 括号可用可不用:

.. code-block:: c++

    for (int i = 0; i < kSomeNumber; ++i)
      printf("I love you\n");

    for (int i = 0; i < kSomeNumber; ++i) {
      printf("I take it back\n");
    }

空循环体应使用 ``{}`` 或 ``continue``, 而不是一个简单的分号.

.. code-block:: c++

    while (condition) {
      // 反复循环直到条件失效.
    }
    for (int i = 0; i < kSomeNumber; ++i) {}  // 可 - 空循环体.
    while (condition) continue;  // 可 - contunue 表明没有逻辑.

.. code-block:: c++

    while (condition);  // 差 - 看起来仅仅只是 while/loop 的部分之一.

9.10. 指针和引用表达式
~~~~~~~~~~~~~~~~~~~~~~~~~~~~~~~~~~~~~~

**总述**

句点或箭头前后不要有空格. 指针/地址操作符 (``*, &``) 之后不能有空格.

**说明**

下面是指针和引用表达式的正确使用范例:

.. code-block:: c++

    x = *p;
    p = &x;
    x = r.y;
    x = r->y;

注意:

- 在访问成员时, 句点或箭头前后没有空格.

- 指针操作符 ``*`` 或 ``&`` 后没有空格.

在声明指针变量或参数时, 星号与类型或变量名紧挨都可以:

.. code-block:: c++

    // 好, 空格前置.
    char *c;
    const string &str;

    // 好, 空格后置.
    char* c;
    const string& str;

.. code-block:: c++

    int x, *y;  // 不允许 - 在多重声明中不能使用 & 或 *
    char * c;  // 差 - * 两边都有空格
    const string & str;  // 差 - & 两边都有空格.

在单个文件内要保持风格一致, 所以, 如果是修改现有文件, 要遵照该文件的风格.

9.11. 布尔表达式
~~~~~~~~~~~~~~~~~~~~~~~~~~~~

**总述**

如果一个布尔表达式超过 :ref:`标准行宽 <line-length>`, 断行方式要统一一下.

**说明**

下例中, 逻辑与 (``&&``) 操作符总位于行尾:

.. code-block:: c++

    if (this_one_thing > this_other_thing &&
        a_third_thing == a_fourth_thing &&
        yet_another && last_one) {
      ...
    }

注意, 上例的逻辑与 (``&&``) 操作符均位于行尾. 这个格式在 Google 里很常见, 虽然把所有操作符放在开头也可以. 可以考虑额外插入圆括号, 合理使用的话对增强可读性是很有帮助的. 此外, 直接用符号形式的操作符, 比如 ``&&`` 和 ``~``, 不要用词语形式的 ``and`` 和 ``compl``.

9.12. 函数返回值
~~~~~~~~~~~~~~~~~~~~~~~~~~~~

**总述**

不要在 ``return`` 表达式里加上非必须的圆括号.

**说明**

只有在写 ``x = expr`` 要加上括号的时候才在 ``return expr;`` 里使用括号.

.. code-block:: c++

    return result;                  // 返回值很简单, 没有圆括号.
    // 可以用圆括号把复杂表达式圈起来, 改善可读性.
    return (some_long_condition &&
            another_condition);

.. code-block:: c++

    return (value);                // 毕竟您从来不会写 var = (value);
    return(result);                // return 可不是函数!

9.13. 变量及数组初始化
~~~~~~~~~~~~~~~~~~~~~~~~~~~~~~~~~~~~~~

**总述**

用 ``=``, ``()`` 和 ``{}`` 均可.

**说明**

您可以用 ``=``, ``()`` 和 ``{}``, 以下的例子都是正确的:

.. code-block:: c++

    int x = 3;
    int x(3);
    int x{3};
    string name("Some Name");
    string name = "Some Name";
    string name{"Some Name"};

请务必小心列表初始化 ``{...}`` 用 ``std::initializer_list`` 构造函数初始化出的类型. 非空列表初始化就会优先调用 ``std::initializer_list``, 不过空列表初始化除外, 后者原则上会调用默认构造函数. 为了强制禁用 ``std::initializer_list`` 构造函数, 请改用括号.

.. code-block:: c++

    vector<int> v(100, 1);  // 内容为 100 个 1 的向量.
    vector<int> v{100, 1};  // 内容为 100 和 1 的向量.

此外, 列表初始化不允许整型类型的四舍五入, 这可以用来避免一些类型上的编程失误. 

.. code-block:: c++

    int pi(3.14);  // 好 - pi == 3.
    int pi{3.14};  // 编译错误: 缩窄转换.

9.14. 预处理指令
~~~~~~~~~~~~~~~~~~~~~~~~~~~~

**总述**

预处理指令不要缩进, 从行首开始.

**说明**

即使预处理指令位于缩进代码块中, 指令也应从行首开始.

.. code-block:: c++

    // 好 - 指令从行首开始
      if (lopsided_score) {
    #if DISASTER_PENDING      // 正确 - 从行首开始
        DropEverything();
    # if NOTIFY               // 非必要 - # 后跟空格
        NotifyClient();
    # endif
    #endif
        BackToNormal();
      }

.. code-block:: c++

    // 差 - 指令缩进
      if (lopsided_score) {
        #if DISASTER_PENDING  // 差 - "#if" 应该放在行开头
        DropEverything();
        #endif                // 差 - "#endif" 不要缩进
        BackToNormal();
      }

9.15. 类格式
~~~~~~~~~~~~~~~~~~~~~~

**总述**

访问控制块的声明依次序是 ``public:``, ``protected:``, ``private:``, 每个都缩进 1 个空格.

**说明**

类声明 (下面的代码中缺少注释, 参考 :ref:`类注释 <class-comments>`) 的基本格式如下:

.. code-block:: c++

    class MyClass : public OtherClass {
     public:      // 注意有一个空格的缩进
      MyClass();  // 标准的两空格缩进
      explicit MyClass(int var);
      ~MyClass() {}

      void SomeFunction();
      void SomeFunctionThatDoesNothing() {
      }

      void set_some_var(int var) { some_var_ = var; }
      int some_var() const { return some_var_; }

     private:
      bool SomeInternalFunction();

      int some_var_;
      int some_other_var_;
    };

注意事项:

- 所有基类名应在 80 列限制下尽量与子类名放在同一行.

- 关键词 ``public:``, ``protected:``, ``private:`` 要缩进 1 个空格.

- 除第一个关键词 (一般是 ``public``) 外, 其他关键词前要空一行. 如果类比较小的话也可以不空.

- 这些关键词后不要保留空行.

- ``public`` 放在最前面, 然后是 ``protected``, 最后是 ``private``.

- 关于声明顺序的规则请参考 :ref:`声明顺序 <declaration-order>` 一节.

9.16. 构造函数初始值列表
~~~~~~~~~~~~~~~~~~~~~~~~~~~~

**总述**

构造函数初始化列表放在同一行或按四格缩进并排多行.

**说明**

下面两种初始值列表方式都可以接受:

.. code-block:: c++

    // 如果所有变量能放在同一行:
    MyClass::MyClass(int var) : some_var_(var) {
      DoSomething();
    }

    // 如果不能放在同一行,
    // 必须置于冒号后, 并缩进 4 个空格
    MyClass::MyClass(int var)
        : some_var_(var), some_other_var_(var + 1) {
      DoSomething();
    }

    // 如果初始化列表需要置于多行, 将每一个成员放在单独的一行
    // 并逐行对齐
    MyClass::MyClass(int var)
        : some_var_(var),             // 4 space indent
          some_other_var_(var + 1) {  // lined up
      DoSomething();
    }

    // 右大括号 } 可以和左大括号 { 放在同一行
    // 如果这样做合适的话
    MyClass::MyClass(int var)
        : some_var_(var) {}

9.17. 命名空间格式化
~~~~~~~~~~~~~~~~~~~~~~~~~~~~~~~~~~

**总述**

命名空间内容不缩进.

**说明**

:ref:`命名空间 <namespaces>` 不要增加额外的缩进层次, 例如:

.. code-block:: c++

    namespace {

    void foo() {  // 正确. 命名空间内没有额外的缩进.
      ...
    }

    }  // namespace

不要在命名空间内缩进:

.. code-block:: c++

    namespace {

      // 错, 缩进多余了.
      void foo() {
        ...
      }

    }  // namespace

声明嵌套命名空间时, 每个命名空间都独立成行.

.. code-block:: c++

    namespace foo {
    namespace bar {

9.19. 水平留白
~~~~~~~~~~~~~~~~~~~~~~~~

**总述**

水平留白的使用根据在代码中的位置决定. 永远不要在行尾添加没意义的留白.

**说明**

通用
=============================

.. code-block:: c++

    void f(bool b) {  // 左大括号前总是有空格.
      ...
    int i = 0;  // 分号前不加空格.
    // 列表初始化中大括号内的空格是可选的.
    // 如果加了空格, 那么两边都要加上.
    int x[] = { 0 };
    int x[] = {0};

    // 继承与初始化列表中的冒号前后恒有空格.
    class Foo : public Bar {
     public:
      // 对于单行函数的实现, 在大括号内加上空格
      // 然后是函数实现
      Foo(int b) : Bar(), baz_(b) {}  // 大括号里面是空的话, 不加空格.
      void Reset() { baz_ = 0; }  // 用空格把大括号与实现分开.
      ...

添加冗余的留白会给其他人编辑时造成额外负担. 因此, 行尾不要留空格. 如果确定一行代码已经修改完毕, 将多余的空格去掉; 或者在专门清理空格时去掉(尤其是在没有其他人在处理这件事的时候). (Yang.Y 注: 现在大部分代码编辑器稍加设置后, 都支持自动删除行首/行尾空格, 如果不支持, 考虑换一款编辑器或 IDE)

循环和条件语句
=============================

.. code-block:: c++

    if (b) {          // if 条件语句和循环语句关键字后均有空格.
    } else {          // else 前后有空格.
    }
    while (test) {}   // 圆括号内部不紧邻空格.
    switch (i) {
    for (int i = 0; i < 5; ++i) {
    switch ( i ) {    // 循环和条件语句的圆括号里可以与空格紧邻.
    if ( test ) {     // 圆括号, 但这很少见. 总之要一致.
    for ( int i = 0; i < 5; ++i ) {
    for ( ; i < 5 ; ++i) {  // 循环里内 ; 后恒有空格, ;  前可以加个空格.
    switch (i) {
      case 1:         // switch case 的冒号前无空格.
        ...
      case 2: break;  // 如果冒号有代码, 加个空格.

操作符
=============================

.. code-block:: c++

    // 赋值运算符前后总是有空格.
    x = 0;

    // 其它二元操作符也前后恒有空格, 不过对于表达式的子式可以不加空格.
    // 圆括号内部没有紧邻空格.
    v = w * x + y / z;
    v = w*x + y/z;
    v = w * (x + z);

    // 在参数和一元操作符之间不加空格.
    x = -5;
    ++x;
    if (x && !y)
      ...

模板和转换
=============================

.. code-block:: c++

    // 尖括号(< and >) 不与空格紧邻, < 前没有空格, > 和 ( 之间也没有.
    vector<string> x;
    y = static_cast<char*>(x);

    // 在类型与指针操作符之间留空格也可以, 但要保持一致.
    vector<char *> x;

9.19. 垂直留白
~~~~~~~~~~~~~~~~~~~~~~~~

**总述**

垂直留白越少越好.

**说明**

这不仅仅是规则而是原则问题了: 不在万不得已, 不要使用空行. 尤其是: 两个函数定义之间的空行不要超过 2 行, 函数体首尾不要留空行, 函数体中也不要随意添加空行.

基本原则是: 同一屏可以显示的代码越多, 越容易理解程序的控制流. 当然, 过于密集的代码块和过于疏松的代码块同样难看, 这取决于你的判断. 但通常是垂直留白越少越好.

下面的规则可以让加入的空行更有效:

- 函数体内开头或结尾的空行可读性微乎其微.

- 在多重 if-else 块里加空行或许有点可读性.

译者 (YuleFox) 笔记
~~~~~~~~~~~~~~~~~~~~~~~~~~~~~~~~~~~~

#. 对于代码格式, 因人, 系统而异各有优缺点, 但同一个项目中遵循同一标准还是有必要的;
#. 行宽原则上不超过 80 列, 把 22 寸的显示屏都占完, 怎么也说不过去;
#. 尽量不使用非 ASCII 字符, 如果使用的话, 参考 UTF-8 格式 (尤其是 UNIX/Linux 下, Windows 下可以考虑宽字符), 尽量不将字符串常量耦合到代码中, 比如独立出资源文件, 这不仅仅是风格问题了;
#. UNIX/Linux 下无条件使用空格, MSVC 的话使用 Tab 也无可厚非;
#. 函数参数, 逻辑条件, 初始化列表: 要么所有参数和函数名放在同一行, 要么所有参数并排分行;
#. 除函数定义的左大括号可以置于行首外, 包括函数/类/结构体/枚举声明, 各种语句的左大括号置于行尾, 所有右大括号独立成行;
#. ``.``/``->`` 操作符前后不留空格, ``*``/``&`` 不要前后都留, 一个就可, 靠左靠右依各人喜好;
#. 预处理指令/命名空间不使用额外缩进, 类/结构体/枚举/函数/语句使用缩进;
#. 初始化用 ``=`` 还是 ``()`` 依个人喜好, 统一就好;
#. ``return`` 不要加 ``()``;
#. 水平/垂直留白不要滥用, 怎么易读怎么来.
#. 关于 UNIX/Linux 风格为什么要把左大括号置于行尾 (``.cc`` 文件的函数实现处, 左大括号位于行首), 我的理解是代码看上去比较简约, 想想行首除了函数体被一对大括号封在一起之外, 只有右大括号的代码看上去确实也舒服; Windows 风格将左大括号置于行首的优点是匹配情况一目了然.

译者(acgtyrant)笔记
~~~~~~~~~~~~~~~~~~~~~~~~~~~~~~~~~~~~~~

#. 80 行限制事实上有助于避免代码可读性失控, 比如超多重嵌套块, 超多重函数调用等等. 
#. Linux 上设置好了 Locale 就几乎一劳永逸设置好所有开发环境的编码, 不像奇葩的 Windows.
#. Google 强调有一对 if-else 时, 不论有没有嵌套, 都要有大括号. Apple 正好 `有栽过跟头 <http://coolshell.cn/articles/11112.html>`_ .
#. 其实我主张指针/地址操作符与变量名紧邻, ``int* a, b`` vs ``int *a, b``, 新手会误以为前者的 ``b`` 是 ``int *`` 变量, 但后者就不一样了, 高下立判. 
#. 在这风格指南里我才刚知道 C++ 原来还有所谓的 `Alternative operator representations <http://en.cppreference.com/w/cpp/language/operator_alternative>`_, 大概没人用吧. 
#. 注意构造函数初始值列表(Constructer Initializer List)与列表初始化(Initializer List)是两码事, 我就差点混淆了它们的翻译. 
#. 事实上, 如果您熟悉英语本身的书写规则, 就会发现该风格指南在格式上的规定与英语语法相当一脉相承. 比如普通标点符号和单词后面还有文本的话, 总会留一个空格; 特殊符号与单词之间就不用留了, 比如 ``if (true)`` 中的圆括号与 ``true``.
#. 本风格指南没有明确规定 void 函数里要不要用 return 语句, 不过就 Google 开源项目 leveldb 并没有写; 此外从 `Is a blank return statement at the end of a function whos return type is void necessary? <http://stackoverflow.com/questions/9316717/is-a-blank-return-statement-at-the-end-of-a-function-whos-return-type-is-void-ne>`_ 来看, ``return;`` 比 ``return ;`` 更约定俗成(事实上 cpplint 会对后者报错, 指出分号前有多余的空格), 且可用来提前跳出函数栈. 

\chapter{规则特例}

前面说明的编程习惯基本都是强制性的。 但所有优秀的规则都允许例外, 这里就是探讨这些特例。

\section{现有不合规范的代码}

\textbf{总述}

对于现有不符合既定编程风格的代码可以网开一面。

\textbf{说明}

当你修改使用其他风格的代码时, 为了与代码原有风格保持一致可以不使用本指南约定。 如果不放心, 可以与代码原作者或现在的负责人员商讨。 记住, *一致性* 也包括原有的一致性。

\section{Windows 代码} \label{windows-code}

\textbf{总述}

Windows 程序员有自己的编程习惯, 主要源于 Windows 头文件和其它 Microsoft 代码。 我们希望任何人都可以顺利读懂你的代码, 所以针对所有平台的 C++ 编程只给出一个单独的指南。

\textbf{说明}

如果你习惯使用 Windows 编码风格, 这儿有必要重申一下某些你可能会忘记的指南:

% TODO(iceyer): naming and pragma once need to be discuss
\begin{itemize}
  \item  不要使用匈牙利命名法 (比如把整型变量命名成 \cppinline{iNum})。 使用 Google 命名约定, 包括对源文件使用 \cppinline{.cc} 扩展名。
  \item Windows 定义了很多原生类型的同义词 (YuleFox 注: 这一点, 我也很反感), 如 \cppinline{DWORD}, \cppinline{HANDLE} 等等。 在调用 Windows API 时这是完全可以接受甚至鼓励的。 即使如此, 还是尽量使用原有的 C++ 类型, 例如使用 \cppinline{const TCHAR *} 而不是 \cppinline{LPCTSTR}。
  \item 使用 Microsoft Visual C++ 进行编译时, 将警告级别设置为 3 或更高, 并将所有警告(warnings)当作错误(errors)处理。
  \item 不要使用 \cppinline{#pragma once}; 而应该使用 Google 的头文件保护规则。 头文件保护的路径应该相对于项目根目录 (Yang.Y 注: 如 \cppinline{#ifndef SRC_DIR_BAR_H_}, 参考 \DFullRef{define-guard} 一节)。
  \item 除非万不得已, 不要使用任何非标准的扩展, 如 \cppinline{#pragma} 和 \cppinline{__declspec}。 使用 \cppinline{__declspec(dllimport)} 和 \cppinline{__declspec(dllexport)} 是允许的, 但必须通过宏来使用, 比如 \cppinline{DLLIMPORT} 和 \cppinline{DLLEXPORT}, 这样其他人在分享使用这些代码时可以很容易地禁用这些扩展。
\end{itemize}

然而, 在 Windows 上仍然有一些我们偶尔需要违反的规则:

\begin{itemize}
  \item 通常我们 \DFullRef{multiple-inheritance}, 但在使用 COM 和 ATL/WTL 类时可以使用多重继承。 为了实现 COM 或 ATL/WTL 类/接口, 你可能不得不使用多重实现继承。
  \item  虽然代码中不应该使用异常, 但是在 ATL 和部分 STL(包括 Visual C++ 的 STL) 中异常被广泛使用。 使用 ATL 时, 应定义 \cppinline{_ATL_NO_EXCEPTIONS} 以禁用异常。 你需要研究一下是否能够禁用 STL 的异常, 如果无法禁用, 可以启用编译器异常。 (注意这只是为了编译 STL, 自己的代码里仍然不应当包含异常处理)。
  \item  通常为了利用头文件预编译, 每个每个源文件的开头都会包含一个名为 \cppinline{StdAfx.h} 或 \cppinline{precompile.h} 的文件。 为了使代码方便与其他项目共享, 请避免显式包含此文件 (除了在 \cppinline{precompile.cc} 中), 使用 \cppinline{/FI} 编译器选项以自动包含该文件。
  \item  资源头文件通常命名为 \cppinline{resource.h} 且只包含宏, 这一文件不需要遵守本风格指南。
\end{itemize}
\chapter{结束语}

运用常识和判断力, 并且 \emph{保持一致} 。

编辑代码时, 花点时间看看项目中的其它代码, 并熟悉其风格。 如果其它代码中 \cppinline{if} 语句使用空格, 那么你也要使用。 如果其中的注释用星号 (*) 围成一个盒子状, 那么你同样要这么做。

风格指南的重点在于提供一个通用的编程规范, 这样大家可以把精力集中在实现内容而不是表现形式上。 我们展示的是一个总体的的风格规范, 但局部风格也很重要, 如果你在一个文件中新加的代码和原有代码风格相去甚远, 这就破坏了文件本身的整体美观, 也让打乱读者在阅读代码时的节奏, 所以要尽量避免。

好了, 关于编码风格写的够多了; 代码本身才更有趣。 尽情享受吧!


\part{GoLang代码风格}

\chapter{扉页}

\section{前言}
风格一致的代码更合理,学习成本更少,且更容易维护。随着新的约定出现或者出现错误后更容易迁移、更新、修复 \texttt{bug} 。

相反,在一个代码库中包含多个完全不同或冲突的代码风格会导致维护成本增加、不确定性上升和部分认知偏差。所有这些都会直接导致速度降低、代码审查痛苦,并且会增加 \texttt{bug} 数量。

因此需要为代码库制定体套标准,目的是规范 \texttt{Go} 项目的开发,保持代码的一致性,使代码库易于管理和维护。

本规范主要基于 \texttt{Uber GoLang Style Guide} 进行编写,同时结合了工作中的实践,给出了提高性能和安全性的编码技巧。

\section{参考文献}
\begin{itemize}[leftmargin=4em]
  \item \href{https://github.com/uber-go/guide/blob/master/style.md}{Uber GoLang Style Guide}
  \item \href{https://github.com/golang-standards/project-layout}{Standard Go Project Layout}
  \item \href{https://golang.org/doc/effective\_go#mixed-caps}{函数命名规则}
  \item \href{https://github.com/golang/go/wiki/CodeReviewComments}{Go代码审查意见}
  \item \href{https://golang.org/doc/effective\_go}{Effective Go}
  \item \href{https://golang.org/ref/spec}{Go语言规范}
  \item \href{https://yougg.github.io/2017/06/12/go\%E8\%AF\%AD\%E8\%A8\%80\%E5\%AE\%89\%E5\%85\%A8\%E7\%BC\%96\%E7\%A8\%8B\%E8\%A7\%84\%E8\%8C\%83/}{Go语言安全编程规范}
  \item \href{https://gruntwork.io/guides/style\%20guides/golang-style-guide}{Gruntwork Go Style Guide}
\end{itemize}

\chapter{缩进}
\texttt{Go} 提供了 \texttt{gofmt} 命令用于格式化代码,因此所有代码文件都必须经过 \texttt{gofmt} 格式化。

建议在开发工具中配置文件保存时自动执行 \texttt{gofmt} ,以自动修复代码中的格式问题。

同时自动执行 \texttt{golint} 和 \texttt{go vet} ,以及时发现代码中的错误并修复。

\chapter{分组}
Go 语言支持将相似的声明放在一个组内,适用于常量、变量和类型声明,不要将不相关的声明放在一组。

\begin{itemize}[leftmargin=4em]
\item 错误用法

  \begin{minted}{go}
    const _STATE_SUCESS = 1
    const _STATE_FAILED = 2
  \end{minted}
\item 正确用法

  \begin{minted}{go}
    const (
    	_STATE_SUCESS = iota + 1
    	_STATE_FAILED
    )
  \end{minted}
\end{itemize}

分组使用的位置没有限制,例如:你可以在函数内部使用它们。
\begin{itemize}[leftmargin=4em]
\item 错误用法

  \begin{minted}{go}
    func calcCost() {
    	var total int
    	idx := 0
    }
  \end{minted}
\item 正确用法

  \begin{minted}{go}
    func calcCost() {
    	var (
    		total int
    		idx int
    	)
    }
  \end{minted}
\end{itemize}

\section{Import}
分组对 \texttt{import} 也同样适用,但标准库和第三方库需要隔开,例如:
\begin{itemize}[leftmargin=4em]
\item 错误用法

  \begin{minted}{go}
    import (
    	"os"
    	"os/exec"
    	"example.com/client-go"
    	"example.com/http"
    )
  \end{minted}
\item 正确用法

  \begin{minted}{go}
    import (
    	"os"
    	"os/exec"

    	"example.com/client-go"
    	"example.com/http"
    )
  \end{minted}
\end{itemize}

推荐使用 \texttt{goimports} 导入包,其会进行自动分组。

\section{函数}
函数在编写时,也应按照以下原则进行分组:
\begin{itemize}[leftmargin=4em]
\item 函数应在 \texttt{struct、const、var} 等定义的后面;
\item 导出的函数应先出现在文件中;
\item 相同接受者的函数应在一起;
\item 普通工具函数应在文件末尾;
\item 函数应按调用顺序排序。
\end{itemize}

如:
\begin{itemize}[leftmargin=4em]
\item 错误用法

  \begin{minted}{go}
    func (s *something) Cost() {
    	return calcCost(s.weights)
    }

    type something struct{ ... }

    func calcCost(n []int) int {...}

    func (s *something) Stop() {...}

    func newSomething() *something {
    	return &something{}
    }
  \end{minted}
\item 正确用法

  \begin{minted}{go}
    type something struct{ ... }

    func newSomething() *something {
    	return &something{}
    }

    func (s *something) Cost() {
    	return calcCost(s.weights)
    }

    func (s *something) Stop() {...}

    func calcCost(n []int) int {...}
  \end{minted}
\end{itemize}

\chapter{命名}
变量名和函数使用驼峰命名法,首字母小写,但如果需要导出,首字母则应该大写。名称应能准确表达其含义,且容易记忆。

代码中应避免使用全局变量,若要使用全局变量,并且无需对外导出时,则在变量名前添加 \texttt{'\_'} 。

\begin{itemize}[leftmargin=4em]
\item 错误用法

  \begin{minted}{go}
    var count int

    func calc_cost() {
    	var Total int
    }
  \end{minted}
\item 正确用法

  \begin{minted}{go}
    var _count int

    func calcCost() {
    	var total int
    }
  \end{minted}
\end{itemize}

\section{避免使用内置名称}
Go 语言规范概述了几个内置的,不应在 Go 项目中使用的名称标识 \texttt{predeclared identifiers} 。

根据上下文的不同,将这些标识符作为名称重复使用, 将在当前作用域(或任何嵌套作用域)中隐藏原始标识符,或者混淆代码。
在最好的情况下,编译器会报错;在最坏的情况下,这样的代码可能会引入潜在的、难以恢复的错误。
\begin{itemize}[leftmargin=4em]
\item 错误用法

  \begin{minted}[breaklines]{go}
    var error string
    // `error` 作用域隐式覆盖
    // or

    func handleErrorMessage(error string) {
    	// `error` 作用域隐式覆盖
    }

    type Foo struct {
    	// 虽然这些字段在技术上不构成阴影,但`error`或`string`字符串的重映射现在是不明确的。
    	error  error
    	string string
    }

    func (f Foo) Error() error {
    	// `error` 和 `f.error` 在视觉上是相似的
    	return f.error
    }

    func (f Foo) String() string {
    	// `string` and `f.string` 在视觉上是相似的
    	return f.string
    }
  \end{minted}
\item 正确用法

  \begin{minted}{go}
    var errorMessage string
    // `error` 指向内置的非覆盖
    // or

    func handleErrorMessage(msg string) {
    	// `error` 指向内置的非覆盖
    }
    type Foo struct {
    	// `error` and `string` 现在是明确的。
    	err error
    	str string
    }

    func (f Foo) Error() error {
    	return f.err
    }

    func (f Foo) String() string {
    	return f.str
    }
  \end{minted}
\end{itemize}

注意,编译器在使用预先分配的标识符时不会生成错误,但是诸如 \texttt{go vet} 之类的工具会正确地指出这些和其它情况下的隐式问题。

\section{常量}
常量因使用蛇形命名法,全局常量需要全部大写。无需对外导出的全局常量,须在常量名前添加 \texttt{'\_'} 。

\begin{itemize}[leftmargin=4em]
\item 错误用法

  \begin{minted}{go}
    const maxBufSize = 1024
    const Version = "1.0"
  \end{minted}
\item 正确用法

  \begin{minted}{go}
    const _MAX_BUF_SIZE = 1024
    const VERSION = "1.0"
  \end{minted}
\end{itemize}

\section{包名}
包名需满足以下规则:
\begin{itemize}[leftmargin=4em]
\item 全部小写,没有大写或下划线及连接符;
\item 包名应避免重复;
\item 简短且简洁,容易记忆;
\item 不用复数;
\item 避免使用 \texttt{common/util/shared/lib} 这类模糊不清的包名。
\end{itemize}

如:
\begin{itemize}[leftmargin=4em]
\item 错误用法

  \begin{minted}{go}
    package URL
    package urls
  \end{minted}
\item 正确用法

  \begin{minted}{go}
    package url
  \end{minted}
\end{itemize}

\chapter{注释}
使用 \texttt{'//'} 的语法添加注释,允许多行。注释推荐使用单独的行,在对应语句之上,避免使用行内注释。

所有对外导出的常量、变量、结构体、方法、接口等都应添加注释。

\begin{itemize}[leftmargin=4em]
\item 错误用法

  \begin{minted}[breaklines]{go}
    const (
    	STATE_SUCCESS = iota + 1 // Indicates that the task has been completed and successfully
    	STATE_FAILED
    )
  \end{minted}
\item 正确用法

  \begin{minted}{go}
    const (
    	// Indicates that the task has been completed and successfully.
    	STATE_SUCCESS = iota + 1
    	// Indicates that the task has been completed, but failure.
    	STATE_FAILED
    )
  \end{minted}
\end{itemize}

\section{TODO 注释}
具体参见 \textbf{Qt} 中的 \textbf{TODO 注释}规则。

\chapter{Import}
包导入时应避免使用别名,但以下情况必须使用别名:
\begin{itemize}[leftmargin=4em]
\item 包名称与导入路径的最后一个元素不匹配;
\item 包名称与已有的包重复。
\end{itemize}

如:
\begin{minted}[xleftmargin=3.5em]{go}
  import (
  	"net/http"

  	myhttp "example.com/http"
  	client "example.com/client-go"
  )
\end{minted}

别名的命名规范与包名一致。

\chapter{变量}
\section{变量声明}
使用标准 \texttt{var} 关键字。请勿指定类型,除非它与表达式的类型不同。
\begin{itemize}[leftmargin=4em]
\item 错误用法

  \begin{minted}{go}
    var _s string = F()

    func F() string { return "A" }
  \end{minted}
\item 正确用法

  \begin{minted}{go}
    // 由于 F 已经明确了返回一个字符串类型,因此我们没有必要显式指定 _s 的类型
    var _s = F()

    func F() string { return "A" }
  \end{minted}
\end{itemize}

如果表达式的类型与所需的类型不完全匹配,请指定类型。如:
\begin{minted}[xleftmargin=3.5em]{go}
  type myError struct{}

  func (myError) Error() string { return "error" }

  func F() myError { return myError{} }

  var _e error = F()
\end{minted}

如果将变量明确设置为某个值,则应使用短变量声明形式 (\texttt{:=})。
\begin{itemize}[leftmargin=4em]
\item 错误用法

  \begin{minted}{go}
    var s = "foo"
  \end{minted}
\item 正确用法

  \begin{minted}{go}
    s := "foo"
  \end{minted}
\end{itemize}

但是,在某些情况下, \texttt{var} 使用关键字时默认值会更清晰。例如,声明空切片。
\begin{itemize}[leftmargin=4em]
\item 错误用法

  \begin{minted}{go}
    func f(list []int) {
    	filtered := []int{}
    	for _, v := range list {
    		if v > 10 {
    			filtered = append(filtered, v)
    		}
    	}
    }
  \end{minted}
\item 正确用法

  \begin{minted}{go}
    func f(list []int) {
    	var filtered []int
    	for _, v := range list {
    		if v > 10 {
    			filtered = append(filtered, v)
    		}
    	}
    }
  \end{minted}
\end{itemize}

\section{缩小变量作用域}
如果有可能,尽量缩小变量作用范围。除非它与减少嵌套的规则冲突。
\begin{itemize}[leftmargin=4em]
\item 错误用法

  \begin{minted}{go}
    err := ioutil.WriteFile(name, data, 0644)
    if err != nil {
    	return err
    }
  \end{minted}
\item 正确用法

  \begin{minted}{go}
    if err := ioutil.WriteFile(name, data, 0644); err != nil {
    	return err
    }
  \end{minted}
\end{itemize}

如果需要在 if 之外使用函数调用的结果,则不应尝试缩小范围。
\begin{itemize}[leftmargin=4em]
\item 错误用法

  \begin{minted}{go}
    if data, err := ioutil.ReadFile(name); err == nil {
    	err = cfg.Decode(data)
    	if err != nil {
    		return err
    	}

    	fmt.Println(cfg)
    	return nil
    } else {
    	return err
    }
  \end{minted}
\item 正确用法

  \begin{minted}{go}
    data, err := ioutil.ReadFile(name)
    if err != nil {
    	return err
    }

    if err := cfg.Decode(data); err != nil {
    	return err
    }

    fmt.Println(cfg)
    return nil
  \end{minted}
\end{itemize}

\section{使用原始字符串字面值}
\texttt{Go} 支持使用原始字符串字面值,也就是 '`' 来表示原生字符串,在需要转义的场景下,我们应该尽量使用这种方案来替换。

可以跨越多行并包含引号。使用这些字符串可以避免更难阅读的手工转义的字符串。

\begin{itemize}[leftmargin=4em]
\item 错误用法

  \begin{minted}{go}
    wantError := "unknown name:\"test\""
  \end{minted}
\item 正确用法

  \begin{minted}{go}
    wantError := `unknown error:"test"`
  \end{minted}
\end{itemize}

\section{避免可变全局变量}
使用选择依赖注入方式避免改变全局变量;既适用于函数指针又适用于其它值类型。
\begin{itemize}[leftmargin=4em]
\item 错误用法

  \begin{minted}{go}
    // sign.go
    var _timeNow = time.Now
    func sign(msg string) string {
    	now := _timeNow()
    	return signWithTime(msg, now)
    }


    // sign_test.go
    func TestSign(t *testing.T) {
    	oldTimeNow := _timeNow
    	_timeNow = func() time.Time {
    		return someFixedTime
    	}
    	defer func() { _timeNow = oldTimeNow }()
    	assert.Equal(t, want, sign(give))
    }
  \end{minted}
\item 正确用法

  \begin{minted}{go}
    // sign.go
    type signer struct {
    	now func() time.Time
    }
    func newSigner() *signer {
    	return &signer{
    		now: time.Now,
    	}
    }
    func (s *signer) Sign(msg string) string {
    	now := s.now()
    	return signWithTime(msg, now)
    }


    // sign_test.go
    func TestSigner(t *testing.T) {
    	s := newSigner()
    	s.now = func() time.Time {
    		return someFixedTime
    	}
    	assert.Equal(t, want, s.Sign(give))
    }
  \end{minted}
\end{itemize}

\chapter{Enum}
\textbf{枚举从 1 开始。}

在 \texttt{Go} 中引入枚举的标准方法是声明一个自定义类型和一个使用了 \texttt{iota} 的 \texttt{const} 组。
由于变量的默认值为 0,因此通常应以非零值开头枚举。
\begin{itemize}[leftmargin=4em]
\item 错误用法

  \begin{minted}{go}
    type Operation int

    const (
    	Add Operation = iota
    	Subtract
    	Multiply
    )

    // Add=0, Subtract=1, Multiply=2
  \end{minted}
\item 正确用法

  \begin{minted}{go}
    type Operation int

    const (
    	Add Operation = iota + 1
    	Subtract
    	Multiply
    )

    // Add=1, Subtract=2, Multiply=3
  \end{minted}
\end{itemize}

在某些情况下,使用零值是有意义的(枚举从零开始),例如,当零值是理想的默认行为时。
\begin{minted}[xleftmargin=3.5em]{go}
type LogOutput int

const (
	LogToStdout LogOutput = iota
	LogToFile
	LogToRemote
)

// LogToStdout=0, LogToFile=1, LogToRemote=2
\end{minted}

\chapter{Map}
对于空 \texttt{map} 请使用 \texttt{make(..)} 初始化, 并且 \texttt{map} 是通过编程方式填充的。
这使得 \texttt{map} 初始化在表现上不同于声明,并且它还可以方便地在 \texttt{make} 后添加大小提示。
\begin{itemize}[leftmargin=4em]
\item 错误用法

  \begin{minted}{go}
    // 声明和初始化看起来非常相似的
    var (
    	// m1 读写安全;
    	// m2 在写入时会 panic
    	m1 = map[T1]T2{}
    	m2 map[T1]T2
    )
  \end{minted}
\item 正确用法

  \begin{minted}{go}
    // 声明和初始化看起来差别非常大
    var (
    	// m1 读写安全;
    	// m2 在写入时会 panic
    	m1 = make(map[T1]T2)
    	m2 map[T1]T2
    )
  \end{minted}
\end{itemize}

尽可能在初始化时提供 \texttt{map} 容量大小。如果 \texttt{map} 包含固定的元素列表,
则使用 \texttt{map literals}(\texttt{map} 初始化列表)初始化映射。
\begin{itemize}[leftmargin=4em]
\item 错误用法

  \begin{minted}{go}
    m := make(map[T1]T2, 3)
    m[k1] = v1
    m[k2] = v2
    m[k3] = v3
  \end{minted}
\item 正确用法

  \begin{minted}{go}
    m := map[T1]T2{
      k1: v1,
      k2: v2,
      k3: v3,
    }
  \end{minted}
\end{itemize}

基本准则是:在初始化时使用 \texttt{map} 初始化列表来添加一组固定的元素。
否则使用 \texttt{make}(如果可以,尽量指定 \texttt{map} 容量)。

\section{指定容器容量}
尽可能指定容器容量,以便为容器预先分配内存。这将在添加元素时最小化后续分配(通过复制和调整容器大小)。

指定 \texttt{Map} 容量提示,尽可能在使用 \texttt{make()} 初始化的时候提供容量信息。
\begin{minted}[xleftmargin=3.5em]{go}
  make(map[T1]T2, hint)
\end{minted}

向 \texttt{make()} 提供容量提示会在初始化时尝试调整 \texttt{map} 的大小,这将减少在将元素添加到 \texttt{map} 时为 \texttt{map} 重新分配内存。
与 \texttt{slices} 不同, \texttt{map capacity} 提示并不保证完全的抢占式分配,而是用于估计所需的 \texttt{hashmap bucket} 的数量。 因此,在将元素添加到 \texttt{map} 时,甚至在指定 \texttt{map} 容量时,仍可能发生分配。
\begin{itemize}[leftmargin=4em]
\item 错误用法

  \begin{minted}{go}
    // m 是在没有大小提示的情况下创建的; 在运行时可能会有更多分配
    m := make(map[string]os.FileInfo)

    files, _ := ioutil.ReadDir("./files")
    for _, f := range files {
    	m[f.Name()] = f
    }
  \end{minted}
\item 正确用法

  \begin{minted}{go}
    // m 是有大小提示创建的;在运行时可能会有更少的分配
    files, _ := ioutil.ReadDir("./files")

    m := make(map[string]os.FileInfo, len(files))
    for _, f := range files {
    	m[f.Name()] = f
    }
  \end{minted}
\end{itemize}

在尽可能的情况下,在使用 \texttt{make()} 初始化切片时提供容量信息,特别是在追加切片时。
\begin{minted}[xleftmargin=3.5em]{go}
  make([]T, length, capacity)
\end{minted}

与 \texttt{maps} 不同, \texttt{slice capacity} 不是一个提示:
编译器将为提供给 \texttt{make()} 的 \texttt{slice} 的容量分配足够的内存,
这意味着后续的 \texttt{append()} 操作将导致零分配(直到 \texttt{slice} 的长度与容量匹配,
在此之后,任何 \texttt{append} 都可能调整大小以容纳其它元素)。

\chapter{Channel}
\textbf{channel 的 size 要么是 1,要么是无缓冲的。}

\texttt{channel} 通常 \texttt{size} 应为 \texttt{1} 或是无缓冲的。默认情况下,\texttt{channel} 是无缓冲的,其 \texttt{size} 为零。
任何其它尺寸都必须经过严格的审查,确定是否是通道边界,竞态条件,以及逻辑上下文导致需要 \texttt{size} 大于 \texttt{1} ,尽量从源头进行排查。
\begin{itemize}[leftmargin=4em]
\item 错误用法

  \begin{minted}{go}
    // 应该足以满足任何情况!
    c := make(chan int, 64)
  \end{minted}
\item 正确用法

  \begin{minted}{go}
    // 大小:1
    c := make(chan int, 1) // 或者
    // 无缓冲 channel,大小为 0
    c := make(chan int)
  \end{minted}
\end{itemize}

\section{禁止重复释放channel}
重复释放一般存在于异常流程判断中,如果恶意攻击者构造出异常条件使程序重复释放 \texttt{channel} ,则会触发运行时恐慌,从而造成 \texttt{DoS} 攻击。

\begin{itemize}[leftmargin=4em]
\item 错误用法

  下面代码中多次关掉channel会触发运行时错误。
  \begin{minted}{go}
    func foo(c chan int) {
    	defer close(c)
    	err := processBusiness()
    	if err != nil {
    		c <- 0
    		close(c) // 【错误】重复释放channel
    		return
    	}
    	c <- 1
    }
  \end{minted}
\item 正确用法

  使用defer延迟关闭channel,并且确保channel只释放一次。
  \begin{minted}{go}
    func foo(c chan int) {
    	defer close(c) // 【修改】使用defer延迟关闭channel
    	err := processBusiness()
    	if err != nil {
    		c <- 0
    		return
    	}
    	c <- 1
    }
  \end{minted}
\end{itemize}

\chapter{Slice}
\section{nil 是一个有效的 slice}
\texttt{nil} 是一个有效的长度为 \texttt{0} 的 \texttt{slice} ,这意味着,不应明确返回长度为零的切片。
应该返回 \texttt{nil} 来代替。
\begin{itemize}[leftmargin=4em]
\item 错误用法

  \begin{minted}{go}
    if x == "" {
    	return []int{}
    }
  \end{minted}
\item 正确用法

  \begin{minted}{go}
    if x == "" {
    	return nil
    }
  \end{minted}
\end{itemize}

要检查切片是否为空,请始终使用 \texttt{len(s) == 0} 。而非 \texttt{nil} 。
\begin{itemize}[leftmargin=4em]
\item 错误用法

  \begin{minted}{go}
    func isEmpty(s []string) bool {
    	return s == nil
    }
  \end{minted}
\item 正确用法

  \begin{minted}{go}
    func isEmpty(s []string) bool {
    	return len(s) == 0
    }
  \end{minted}
\end{itemize}

零值切片可立即使用,无需调用 \texttt{make} 创建。
\begin{itemize}[leftmargin=4em]
\item 错误用法

  \begin{minted}{go}
    nums := []int{}
    // or, nums := make([]int)

    if add1 {
    	nums = append(nums, 1)
    }

    if add2 {
    	nums = append(nums, 2)
    }
  \end{minted}
\item 正确用法

  \begin{minted}{go}
    var nums []int

    if add1 {
    	nums = append(nums, 1)
    }

    if add2 {
    	nums = append(nums, 2)
    }
  \end{minted}
\end{itemize}

记住,虽然 \texttt{nil} 切片是有效的切片,但它不等于长度为 \texttt{0} 的切片(一个为 \texttt{nil} ,另一个不是),
并且在不同的情况下(例如序列化),这两个切片的处理方式可能不同。

\section{追加时优先指定切片容量}
在尽可能的情况下,在初始化要追加的切片时为make()提供一个容量值。
\begin{itemize}[leftmargin=4em]
\item 错误用法

  \begin{minted}{go}
    for n := 0; n < b.N; n++ {
    	data := make([]int, 0)
    	for k := 0; k < size; k++{
    		data = append(data, k)
    	}
    }
  \end{minted}
\item 正确用法

  \begin{minted}{go}
    for n := 0; n < b.N; n++ {
    	data := make([]int, 0, size)
    	for k := 0; k < size; k++{
    		data = append(data, k)
    	}
    }
  \end{minted}
\end{itemize}

\chapter{Interface}
\section{interface 合理性验证}
在编译时验证接口的符合性,包括:
\begin{itemize}[leftmargin=4em]
\item 将实现特定接口的导出类型作为接口API 的一部分进行检查;
\item 实现同一接口的(导出和非导出)类型属于实现类型的集合;
\item 任何违反接口合理性检查的场景;都会终止编译,并通知给用户。
\end{itemize}

补充:上面3条是编译器对接口的检查机制, 使错误使用接口在编译期报错。 所以可以利用这个机制让部分问题在编译期暴露。

如果 \texttt{\*Handler} 与 \texttt{http.Handler} 的接口不匹配,
那么语句 \texttt{var \_ http.Handler = (\*Handler)(nil)} 将无法编译通过。

\begin{itemize}[leftmargin=4em]
\item 错误用法

  \begin{minted}{go}
    // 如果Handler没有实现http.Handler,会在运行时报错
    type Handler struct {
    // ...
    }
    func (h *Handler) ServeHTTP(
    	w http.ResponseWriter,
    	r *http.Request,
    ) {
    //...
    }
  \end{minted}
\item 正确用法

  \begin{minted}{go}
    type Handler struct {
    	// ...
    }
    // 用于触发编译期的接口的合理性检查机制
    // 如果Handler没有实现http.Handler,会在编译期报错
    var _ http.Handler = (*Handler)(nil)
    func (h *Handler) ServeHTTP(
    	w http.ResponseWriter,
    	r *http.Request,
    ) {
    	// ...
    }
  \end{minted}
\end{itemize}

赋值的右边应该是断言类型的零值。 对于指针类型(如 \texttt{\*Handler})、切片和映射,这是 \texttt{nil}; 对于结构类型,这是空结构。

\begin{minted}[xleftmargin=3.5em]{go}
  type LogHandler struct {
  	h   http.Handler
  	log *zap.Logger
  }
  var _ http.Handler = LogHandler{}
  func (h LogHandler) ServeHTTP(
  	w http.ResponseWriter,
  	r *http.Request,
  ) {
  	// ...
  }
\end{minted}

\section{指向 interface 的指针}
通常用不到指向接口类型的指针,应该将接口作为值进行传递,在这样的传递过程中,实质上传递的底层数据仍然可以是指针。

接口实质上在底层用两个字段表示:
\begin{itemize}[leftmargin=4em]
\item 一个指向某些特定类型信息的指针。可以将其视为 "type" ;
\item 数据指针。如果存储的数据是指针,则直接存储。如果存储的数据是一个值,则存储指向该值的指针。
\end{itemize}

如果希望接口方法修改基础数据,则必须使用指针传递(将对象指针赋值给接口变量)。
\begin{minted}[xleftmargin=3.5em]{go}
  type F interface {
  	f()
  }

  type S1 struct{}

  func (s S1) f() {}

  type S2 struct{}

  func (s *S2) f() {}

  // f1.f()无法修改底层数据
  // f2.f() 可以修改底层数据,给接口变量f2赋值时使用的是对象指针
  var f1 F = S1{}
  var f2 F = &S2{}
\end{minted}

\chapter{结构体}
\section{结构体嵌入}
嵌入类型(例如 \texttt{mutex} )应位于结构体内的字段列表的顶部,并且必须有一个空行将嵌入式字段与常规字段分隔开。
\begin{itemize}[leftmargin=4em]
\item 错误用法

  \begin{minted}{go}
    type Client struct {
    	version int
    	http.Client
    }
  \end{minted}
\item 正确用法

  \begin{minted}{go}
    type Client struct {
    	http.Client

    	version int
    }
  \end{minted}
\end{itemize}

内嵌应该提供切实的好处,比如以语义上合适的方式添加或增强功能。 它应该在对用户没有任何不利影响的情况下使用。

嵌入不应该:
\begin{itemize}[leftmargin=4em]
\item 纯粹是为了美观或方便;
\item 使外部类型更难构造或使用;
\item 影响外部类型的零值。如果外部类型有一个有用的零值,则在嵌入内部类型之后应该仍然有一个有用的零值;
\item 作为嵌入内部类型的副作用,从外部类型公开不相关的函数或字段;
\item 公开未导出的类型;
\item 影响外部类型的复制形式;
\item 更改外部类型的API或类型语义;
\item 嵌入内部类型的非规范形式;
\item 公开外部类型的实现详细信息;
\item 允许用户观察或控制类型内部;
\item 通过包装的方式改变内部函数的一般行为,这种包装方式会给用户带来一些意料之外情况。
\end{itemize}

简单地说,做到有意识和有目的嵌入。一种很好的测试体验是,“是否所有这些导出的内部方法、字段都将直接添加到外部类型”如果答案是 \texttt{some} 或 \texttt{no} ,不要嵌入内部类型,而是使用字段。
\begin{itemize}[leftmargin=4em]
\item 错误用法

  \begin{minted}{go}
    type A struct {
    	// Bad: A.Lock() and A.Unlock() 现在可用
    	// 不提供任何功能性好处,并允许用户控制有关A的内部细节。
    	sync.Mutex
    }

    type Book struct {
    	// Bad: 指针更改零值的有用性
    	io.ReadWriter
    	// other fields
    }
    // later
    var b Book
    b.Read(...)  // panic: nil pointer
    b.String()   // panic: nil pointer
    b.Write(...) // panic: nil pointer
    type Client struct {
    	sync.Mutex
    	sync.WaitGroup
    	bytes.Buffer
    	url.URL
    }
  \end{minted}
\item 正确用法

  \begin{minted}{go}
    type countingWriteCloser struct {
    	// Good: Write() 在外层提供用于特定目的,
    	// 并且委托工作到内部类型的Write()中。
    	io.WriteCloser
    	count int
    }
    func (w *countingWriteCloser) Write(bs []byte) (int, error) {
    	w.count += len(bs)
    	return w.WriteCloser.Write(bs)
    }
    type Book struct {
    	// Good: 有用的零值
    	bytes.Buffer
    	// other fields
    }
    // later
    var b Book
    b.Read(...)  // ok
    b.String()   // ok
    b.Write(...) // ok
    type Client struct {
    	mtx sync.Mutex
    	wg  sync.WaitGroup
    	buf bytes.Buffer
    	url url.URL
    }
  \end{minted}
\end{itemize}

\section{使用字段名初始化结构体}
初始化结构体时,应该指定字段名称。
省略结构中的零值字段,初始化具有字段名的结构时,除非提供有意义的上下文,否则忽略值为零的字段。
\begin{itemize}[leftmargin=4em]
\item 错误用法

  \begin{minted}{go}
    k := User{"John", "Doe", Admin: false}
  \end{minted}
\item 正确用法

  \begin{minted}{go}
    k := User{
    	FirstName: "John",
    	LastName: "Doe",
    }
  \end{minted}
\end{itemize}

这有助于通过省略该上下文中的默认值来减少阅读的障碍。只指定有意义的值。

在字段名提供有意义上下文的地方包含零值。例如,表驱动测试中的测试用例可以受益于字段的名称,即使它们是零值的。
\begin{minted}[xleftmargin=3.5em]{go}
  tests := []struct{
  	give string
  	want int
  }{
  	give: "0",
  	want: 0,
  	// ...
  }
\end{minted}

对零值结构使用 \texttt{var} ,如果在声明中省略了结构的所有字段,请使用 \texttt{var} 声明结构。
\begin{itemize}[leftmargin=4em]
\item 错误用法

  \begin{minted}{go}
    user := User{}
  \end{minted}
\item 正确用法

  \begin{minted}{go}
    var user User
  \end{minted}
\end{itemize}

这将零值结构与那些具有类似于为{初始化 \texttt{Maps}} 创建的、区别于非零值字段的结构区分开来。
初始化 \texttt{struct} 引用,在初始化结构引用时,请使用 \texttt{\&T\{\}} 代替 \texttt{new(T)} ,以使其与结构体初始化一致。
\begin{itemize}[leftmargin=4em]
\item 错误用法

  \begin{minted}{go}
    sval := T{Name: "foo"}

    // inconsistent
    sptr := new(T)
    sptr.Name = "bar"
  \end{minted}
\item 正确用法

  \begin{minted}{go}
    sval := T{Name: "foo"}

    sptr := &T{Name: "bar"}
  \end{minted}
\end{itemize}

\chapter{控制语句}
代码应通过尽可能先处理错误情况和特殊情况,并尽早返回或继续循环来减少嵌套。减少嵌套多个级别的代码的代码量。
\begin{itemize}[leftmargin=4em]
\item 错误用法

  \begin{minted}{go}
    for _, v := range data {
    	if v.F1 == 1 {
    		v = process(v)
    		if err := v.Call(); err == nil {
    			v.Send()
    		} else {
    			return err
    		}
    	} else {
    		log.Printf("Invalid v: %v", v)
    	}
    }
  \end{minted}
\item 正确用法

  \begin{minted}{go}
    for _, v := range data {
    	if v.F1 != 1 {
    		log.Printf("Invalid v: %v", v)
    		continue
    	}

    	v = process(v)
    	if err := v.Call(); err != nil {
    		return err
    	}
    	v.Send()
    }
  \end{minted}
\end{itemize}

去掉不必要的 \texttt{else} ,如果在 \texttt{if} 的两个分支中都设置了变量,则可以将其替换为单个 \texttt{if} 。
\begin{itemize}[leftmargin=4em]
\item 错误用法

  \begin{minted}{go}
    var a int
    if b {
    	a = 100
    } else {
    	a = 10
    }
  \end{minted}
\item 正确用法

  \begin{minted}{go}
    a := 10
    if b {
    	a = 100
    }
  \end{minted}
\end{itemize}

\chapter{Defer}
使用 \texttt{defer} 释放资源,诸如文件和锁。
\begin{itemize}[leftmargin=4em]
\item 错误用法

  \begin{minted}{go}
    p.Lock()
    if p.count < 10 {
    	p.Unlock()
    	return p.count
    }

    p.count++
    newCount := p.count
    p.Unlock()

    return newCount

    // 当有多个 return 分支时,很容易遗忘 unlock
  \end{minted}
\item 正确用法

  \begin{minted}{go}
    p.Lock()
    defer p.Unlock()

    if p.count < 10 {
    	return p.count
    }

    p.count++
    return p.count

    // 更可读
  \end{minted}
\end{itemize}

\texttt{defer} 的开销非常小,除非证明函数执行时间处于纳秒级的程度,才应避免这样做。
使用 \texttt{defer} 提升可读性是值得的,只有微不足道的成本。
尤其适用于那些不仅仅是简单内存访问的较大的方法,在这些方法中其它计算的资源消耗远超过 \texttt{defer} 。

\chapter{Error}
\texttt{Go} 中有多种声明错误(Error) 的选项:
\begin{itemize}[leftmargin=4em]
\item \texttt{errors.New} 对于简单静态字符串的错误;
\item \texttt{fmt.Errorf} 用于格式化的错误字符串;
\item 实现 \texttt{Error()} 方法的自定义类型;
\item 用 \texttt{"pkg/errors".Wrap} 的 \texttt{Wrapped errors} 。
\end{itemize}

返回错误时,考虑以下因素以确定最佳选择:
\begin{itemize}[leftmargin=4em]
\item 这是一个不需要额外信息的简单错误吗?如果是这样, \texttt{errors.New} 足够了;
\item 客户需要检测并处理此错误吗?如果是这样,则应使用自定义类型并实现该 \texttt{Error()} 方法;
\item 当前是否正在传播下游函数返回的错误?如果是这样,使用错误包装;
\item 否则 \texttt{fmt.Errorf} 就可以了。
\end{itemize}

如果客户端需要检测错误,并且已使用创建了一个简单的错误 \texttt{errors.New} ,请使用一个错误变量。
\begin{itemize}[leftmargin=4em]
\item 错误用法

  \begin{minted}{go}
    // package foo

    func Open() error {
    	return errors.New("could not open")
    }


    // package bar

    func use() {
    	if err := foo.Open(); err != nil {
    		if err.Error() == "could not open" {
    			// handle
    		} else {
    			panic("unknown error")
    		}
    	}
    }
  \end{minted}
\item 正确用法

  \begin{minted}{go}
    // package foo

    var ErrCouldNotOpen = errors.New("could not open")

    func Open() error {
    	return ErrCouldNotOpen
    }

    // package bar

    if err := foo.Open(); err != nil {
    	if errors.Is(err, foo.ErrCouldNotOpen) {
    		// handle
    	} else {
    		panic("unknown error")
    	}
    }
  \end{minted}
\end{itemize}

如果有可能需要客户端检测的错误,并且想向其中添加更多信息(例如,它不是静态字符串),则应使用自定义类型。
\begin{itemize}[leftmargin=4em]
\item 错误用法

  \begin{minted}{go}
    func open(file string) error {
    	return fmt.Errorf("file %q not found", file)
    }

    func use() {
    	if err := open("testfile.txt"); err != nil {
    		if strings.Contains(err.Error(), "not found") {
    			// handle
    		} else {
    			panic("unknown error")
    		}
    	}
    }
  \end{minted}
\item 正确用法

  \begin{minted}{go}
    type errNotFound struct {
    	file string
    }

    func (e errNotFound) Error() string {
    	return fmt.Sprintf("file %q not found", e.file)
    }

    func open(file string) error {
    	return errNotFound{file: file}
    }

    func use() {
    	if err := open("testfile.txt"); err != nil {
    		if _, ok := err.(errNotFound); ok {
    			// handle
    		} else {
    			panic("unknown error")
    		}
    	}
    }
  \end{minted}
\end{itemize}

直接导出自定义错误类型时要小心,因为它们已成为程序包公共 API 的一部分。最好公开匹配器功能以检查错误。
\begin{minted}[xleftmargin=3.5em]{go}
  // package foo

  type errNotFound struct {
  	file string
  }

  func (e errNotFound) Error() string {
  	return fmt.Sprintf("file %q not found", e.file)
  }

  func IsNotFoundError(err error) bool {
  	_, ok := err.(errNotFound)
  	return ok
  }

  func Open(file string) error {
  	return errNotFound{file: file}
  }


  // package bar

  if err := foo.Open("foo"); err != nil {
  	if foo.IsNotFoundError(err) {
  		// handle
  	} else {
  		panic("unknown error")
  	}
  }
\end{minted}

\section{Error Wrapping}
一个(函数/方法)调用失败时,有三种主要的传播错误方式:
\begin{itemize}[leftmargin=4em]
\item 如果没有要添加的其它上下文,并且要维护原始错误类型,则返回原始错误;
\item 添加上下文,使用 \texttt{"pkg/errors".Wrap} 以便错误消息提供更多上下文, \texttt{"pkg/errors".Cause} 可用于提取原始错误;
\item 如果调用者不需要检测或处理的特定错误情况,使用 \texttt{fmt.Errorf} 。
\end{itemize}

建议在可能的地方添加上下文,用来获得诸如“调用服务 foo:连接被拒绝”之类的更有用的错误,而不是诸如“连接被拒绝”之类的模糊错误。

在将上下文添加到返回的错误时,请避免使用“failed to”之类的短语以保持上下文简洁,
这些短语会陈述明显的内容,并随着错误在堆栈中的渗透而逐渐堆积:
\begin{itemize}[leftmargin=4em]
\item 错误用法

  \begin{minted}{go}
    s, err := store.New()
    if err != nil {
    	return fmt.Errorf("failed to create new store: %v", err)
    }
  \end{minted}
\item 正确用法

  \begin{minted}{go}
    s, err := store.New()
    if err != nil {
    	return fmt.Errorf("new store: %v", err)
    }
  \end{minted}
\end{itemize}

但是,一旦将错误发送到另一个系统,就应该明确消息是错误消息(例如使用err标记,或在日志中以”Failed”为前缀)。

\section{处理类型断言失败}
\texttt{type assertion} 的单个返回值形式针对不正确的类型将产生 \texttt{panic} 。因此,始终使用“comma ok”的惯用法。
\begin{itemize}[leftmargin=4em]
\item 错误用法

  \begin{minted}{go}
    t := i.(string)
  \end{minted}
\item 正确用法

  \begin{minted}{go}
    t, ok := i.(string)
    if !ok {
    	// 优雅地处理错误
    }
  \end{minted}
\end{itemize}

\chapter{Panic}
在生产环境中运行的代码必须避免出现 \texttt{panic} 。 \texttt{panic} 是 \texttt{cascading failures} 级联失败的主要根源 。
如果发生错误,该函数必须返回错误,并允许调用方决定如何处理它。
\begin{itemize}[leftmargin=4em]
\item 错误用法

  \begin{minted}{go}
    func run(args []string) {
    	if len(args) == 0 {
    		panic("an argument is required")
    	}
    	// ...
    }

    func main() {
    	run(os.Args[1:])
    }
  \end{minted}
\item 正确用法

  \begin{minted}{go}
    func run(args []string) error {
    	if len(args) == 0 {
    		return errors.New("an argument is required")
    	}
    	// ...
    	return nil
    }

    func main() {
    	if err := run(os.Args[1:]); err != nil {
    		fmt.Fprintln(os.Stderr, err)
    		os.Exit(1)
    	}
    }
  \end{minted}
\end{itemize}

\texttt{panic/recover} 不是错误处理策略。仅当发生不可恢复的事情(例如: \texttt{nil} 引用)时,程序才必须 \texttt{panic} 。
程序初始化是一个例外:程序启动时应使程序中止的不良情况可能会引起 \texttt{panic} 。
\begin{minted}[xleftmargin=3.5em,breaklines]{go}
var _statusTemplate = template.Must(template.New("name").Parse("_statusHTML"))
\end{minted}

即使在测试代码中,也优先使用 \texttt{t.Fatal} 或者 \texttt{t.FailNow} 而不是 \texttt{panic} 来确保失败被标记。
\begin{itemize}[leftmargin=4em]
\item 错误用法

  \begin{minted}{go}
    // func TestFoo(t *testing.T)

    f, err := ioutil.TempFile("", "test")
    if err != nil {
    	panic("failed to set up test")
    }
  \end{minted}
\item 正确用法

  \begin{minted}{go}
    // func TestFoo(t *testing.T)

    f, err := ioutil.TempFile("", "test")
    if err != nil {
    	t.Fatal("failed to set up test")
    }
  \end{minted}
\end{itemize}

\chapter{Printf-style}
如果你在函数外声明 \texttt{Printf-style} 函数的格式字符串,请将其设置为 \texttt{const} 常量。
这有助于 \texttt{go vet} 对格式字符串执行静态分析。
\begin{itemize}[leftmargin=4em]
\item 错误用法

  \begin{minted}{go}
    msg := "unexpected values %v, %v\n"
    fmt.Printf(msg, 1, 2)
  \end{minted}
\item 正确用法

  \begin{minted}{go}
    const msg = "unexpected values %v, %v\n"
    fmt.Printf(msg, 1, 2)
  \end{minted}
\end{itemize}

声明 \texttt{Printf-style} 函数时,确保 \texttt{go vet} 可以检测到它并检查格式字符串。
尽可能使用预定义的 \texttt{Printf-style} 函数名称, \texttt{go vet} 将默认进行检查。

如果不能使用预定义的名称,则以 \texttt{f} 结束选择的名称: \texttt{Wrapf} ,而不是 \texttt{Wrap} 。
\texttt{go vet} 可以要求检查特定的 \texttt{Printf} 样式名称,但名称必须以 \texttt{f} 结尾。

\chapter{函数}
\section{避免参数语义不明确}
函数调用中的意义不明确的参数可能会损害可读性。当参数名称的含义不明显时,请为参数添加 C 样式注释 (\texttt{/* ... */})
\begin{itemize}[leftmargin=4em]
\item 错误用法

  \begin{minted}{go}
    // func printInfo(name string, isLocal, done bool)

    printInfo("foo", true, true)
  \end{minted}
\item 正确用法

  \begin{minted}{go}
    // func printInfo(name string, isLocal, done bool)

    printInfo("foo", true /* isLocal */, true /* done */)
  \end{minted}
\end{itemize}

对于上面的示例代码,还有一种更好的处理方式是将上面的 \texttt{bool} 类型换成自定义类型。
将来,该参数可以支持不仅仅局限于两个状态(\texttt{true/false})。
\begin{minted}[xleftmargin=3.5em]{go}
type Region int

const (
	UnknownRegion Region = iota
	Local
)

type Status int

const (
	StatusReady Status= iota + 1
	StatusDone
	// Maybe we will have a StatusInProgress in the future.
)

func printInfo(name string, region Region, status Status)
\end{minted}

优先使用此种方式。

\section{最小职责原则}
函数实现时应遵循最小职责原则,尽量使函数的功能单一且简洁,避免多种功能揉合。
\begin{itemize}[leftmargin=4em]
\item 错误用法

  \begin{minted}{go}
    // Don't do this
    func main() {
    	fmt.Println(mulOfSums(1, 1))
    }

    func mulOfSums(a, b int) int {
    	return (a + b) * (a + b)
    }
  \end{minted}
\item 正确用法

  \begin{minted}{go}
    // Do this instead
    func main() {
    	fmt.Println(mul(add(1, 1), add(1, 1)))
    }

    func add(a, b int) int {
    	return a + b
    }

    func mul(a, b int) int {
    	return a * b
    }
  \end{minted}
\end{itemize}

\section{避免使用 init()}
尽可能避免使用 \texttt{init()} 。当 \texttt{init()} 是不可避免或可取的,代码应先尝试:
\begin{enumerate}[leftmargin=4em]
\item 无论程序环境或调用如何,都要完全确定;
\item 避免依赖于其它 \texttt{init()} 函数的顺序或副作用。虽然 \texttt{init()} 顺序是明确的,但代码可以更改, 因此 \texttt{init()} 函数之间的关系可能会使代码变得脆弱和容易出错;
\item 避免访问或操作全局或环境状态,如机器信息、环境变量、工作目录、程序参数/输入等;
\item 避免I/O,包括文件系统、网络和系统调用。
\end{enumerate}

不能满足这些要求的代码可能属于要作为 \texttt{main()} 调用的一部分(或程序生命周期中的其它地方), 或者作为 \texttt{main()} 本身的一部分写入。
特别是,打算由其它程序使用的库应该特别注意完全确定性, 而不是执行“init magic”。

\begin{itemize}[leftmargin=4em]
\item 错误用法

  \begin{minted}{go}
    type Foo struct {
    	// ...
    }
    var _defaultFoo Foo
    func init() {
    	_defaultFoo = Foo{
    		// ...
    	}
    }

    type Config struct {
    	// ...
    }
    var _config Config
    func init() {
    	// Bad: 基于当前目录
    	cwd, _ := os.Getwd()
    	// Bad: I/O
    	raw, _ := ioutil.ReadFile(
    		path.Join(cwd, "config", "config.yaml"),
    	)
    	yaml.Unmarshal(raw, &_config)
    }
  \end{minted}
\item 正确用法

  \begin{minted}{go}
    var _defaultFoo = Foo{
    	// ...
    }
    // or, 为了更好的可测试性:
    var _defaultFoo = defaultFoo()
    func defaultFoo() Foo {
    	return Foo{
    		// ...
    	}
    }
    type Config struct {
    	// ...
    }
    func loadConfig() Config {
    	cwd, err := os.Getwd()
    	// handle err
    	raw, err := ioutil.ReadFile(
    		path.Join(cwd, "config", "config.yaml"),
    	)
    	// handle err
    	var config Config
    	yaml.Unmarshal(raw, &config)
    	return config
    }
  \end{minted}
\end{itemize}

考虑到上述情况,在某些情况下, \texttt{init()} 可能更可取或是必要的,可能包括:
\begin{itemize}[leftmargin=4em]
\item 不能表示为单个赋值的复杂表达式;
\item 可插入的钩子,如 \texttt{database/sql} 、编码类型注册表等;
\item 对 \texttt{Google Cloud Functions} 和其它形式的确定性预计算的优化。
\end{itemize}

\section{主函数退出方式(Exit)}
Go 程序使用 \texttt{os.Exit} 或者 \texttt{log.Fatal\*} 立即退出 (使用 \texttt{panic} 不是退出程序的好方法)。

仅在 \texttt{main()} 中调用其中一个 \texttt{os.Exit} 或者 \texttt{log.Fatal\*} 。
所有其它函数应将错误通过返回值返回,并进行处理。
\begin{itemize}[leftmargin=4em]
\item 错误用法

  \begin{minted}{go}
    func main() {
    	body := readFile(path)
    	fmt.Println(body)
    }
    func readFile(path string) string {
    	f, err := os.Open(path)
    	if err != nil {
    		log.Fatal(err)
    	}
    	b, err := ioutil.ReadAll(f)
    	if err != nil {
    		log.Fatal(err)
    	}
    	return string(b)
    }
  \end{minted}
\item 正确用法

  \begin{minted}{go}
    func main() {
    	body, err := readFile(path)
    	if err != nil {
    		log.Fatal(err)
    	}
    	fmt.Println(body)
    }
    func readFile(path string) (string, error) {
    	f, err := os.Open(path)
    	if err != nil {
    		return "", err
    	}
    	b, err := ioutil.ReadAll(f)
    	if err != nil {
    		return "", err
    	}
    	return string(b), nil
    }
  \end{minted}
\end{itemize}

原则上,具有多种功能的程序退出存在一些问题:
\begin{itemize}[leftmargin=4em]
\item 不明显的控制流:任何函数都可以退出程序,因此很难对控制流进行推理;
\item 难以测试:退出程序的函数也将退出调用它的测试。这使得函数很难测试,并引入了跳过 \texttt{go test} 尚未运行的其它测试的风险;
\item 跳过清理:当函数退出程序时,会跳过已经进入 \texttt{defer} 队列里的函数调用。这增加了跳过重要清理任务的风险。
\end{itemize}

一次性退出原则,在 \texttt{main()} 函数中最多一次调用 \texttt{os.Exit} 或者 \texttt{log.Fatal} 。
如果有多个错误场景停止程序执行,请将该逻辑放在单独的函数下并从中返回错误。
这会缩短 \texttt{main()} 函数,并将所有关键业务逻辑放入一个单独的、可测试的函数中。

\begin{itemize}[leftmargin=4em]
\item 错误用法

  \begin{minted}{go}
    package main
    func main() {
    	args := os.Args[1:]
    	if len(args) != 1 {
    		log.Fatal("missing file")
    	}
    	name := args[0]
    	f, err := os.Open(name)
    	if err != nil {
    		log.Fatal(err)
    	}
    	defer f.Close()
    	// 如果我们调用log.Fatal 在这条线之后
    	// f.Close 将会被执行.
    	b, err := ioutil.ReadAll(f)
    	if err != nil {
    		log.Fatal(err)
    	}
    	// ...
    }
  \end{minted}
\item 正确用法

  \begin{minted}{go}
    package main
    func main() {
    	if err := run(); err != nil {
    		log.Fatal(err)
    	}
    }

    func run() error {
    	args := os.Args[1:]
    	if len(args) != 1 {
    		return errors.New("missing file")
    	}
    	name := args[0]
    	f, err := os.Open(name)
    	if err != nil {
    		return err
    	}
    	defer f.Close()
    	b, err := ioutil.ReadAll(f)
    	if err != nil {
    		return err
    	}
    	// ...
    }
  \end{minted}
\end{itemize}

\chapter{项目布局}
除了编码风格需要保持一致外,项目的布局也应该保持一致。项目目录应按以下要求进行组织:

\begin{itemize}[leftmargin=4em]
\item 通用目录

  \begin{itemize}
  \item \texttt{/cmd}

    项目的主干。

    每个应用程序的目录名应该与你想要的可执行文件的名称相匹配(例如, \texttt{/cmd/myapp} )。

    不要在这个目录中放置太多代码。如果你认为代码可以导入并在其它项目中使用,那么它应该位于 \texttt{/pkg} 目录中。
    如果代码不是可重用的,或者你不希望其它人重用它,请将该代码放到 \texttt{/internal} 目录中。你会惊讶于别人会怎么做,所以要明确你的意图!

    通常有一个小的 \texttt{main} 函数,从 \texttt{/internal} 和 \texttt{/pkg} 目录导入和调用代码,除此之外没有别的东西。

  \item \texttt{/internal}

    私有应用程序和库代码。这是你不希望其它人在其应用程序或库中导入代码。请注意,这个布局模式是由 Go 编译器本身执行的。
    注意,你并不局限于顶级 internal 目录。在项目树的任何级别上都可以有多个内部目录。

    你可以选择向 \texttt{internal} 包中添加一些额外的结构,以分隔共享和非共享的内部代码。
    这不是必需的(特别是对于较小的项目),但是最好有有可视化的线索来显示预期的包的用途。

    你的实际应用程序代码可以放在 \texttt{/internal/app} 目录下(例如 \texttt{/internal/app/myapp}),
    这些应用程序共享的代码可以放在 \texttt{/internal/pkg} 目录下(例如 \texttt{/internal/pkg/myprivlib})。

  \item \texttt{/pkg}

    外部应用程序可以使用的库代码(例如 \texttt{/pkg/mypubliclib})。其它项目会导入这些库,希望它们能正常工作,所以不建议在此目录存放别的东西。
    注意, \texttt{internal} 目录是确保私有包不可导入的更好方法,因为它是由 Go 强制执行的。

    \texttt{/pkg} 目录仍然是一种很好的方式,可以显式地表示该目录中的代码对于其它人来说是安全使用的好方法。
    由 Travis Jeffery  撰写的 \href{https://travisjeffery.com/b/2019/11/i-ll-take-pkg-over-internal/}{I'll take pkg over internal} 博客文章提供了 \texttt{pkg} 和 \texttt{internal} 目录的一个很好的概述,以及什么时候使用它们是有意义的。

    如果你的应用程序项目真的很小,并且额外的嵌套并不能增加多少价值,
    那就不要使用它。当它变得足够大时,你的根目录会变得非常繁琐时(尤其是当你有很多非 Go 应用组件时),请考虑一下。

  \item \texttt{/vendor}

    应用程序依赖项(手动管理或使用你喜欢的依赖项管理工具,如新的内置 \texttt{Go Modules} 功能)。
    \texttt{go mod vendor} 命令将为你创建 \texttt{/vendor} 目录。
    请注意,如果未使用默认情况下处于启用状态的 \texttt{Go 1.14} ,则可能需要在 \texttt{go build} 命令中添加 \texttt{-mod=vendor} 标志。

    如果你正在构建一个库,那么不要提交你的应用程序依赖项。

    七牛云有维护专门的代理模块功能 \href{https://github.com/goproxy/goproxy.cn/blob/master/README.zh-CN.md}{模块代理} 。

  \item \texttt{/configs}

    配置文件模板或默认配置。

    将你的 \texttt{confd} 或 \texttt{consul-template} 模板文件放在这里。

  \item \texttt{/init}

    System init(systemd,upstart,sysv)和 process manager/supervisor(runit,supervisor)配置。

  \item \texttt{/scripts}

    执行各种构建、安装、分析等操作的脚本。

    这些脚本保持了根级别的 Makefile 变得小而简单(例如, \href{https://github.com/hashicorp/terraform/blob/master/Makefile}{terraform Makefile} )。

  \item \texttt{/build}

    打包和持续集成。

    将你的云( AMI )、容器( Docker )、操作系统( deb、rpm、pkg )包配置和脚本放在 \texttt{/build/package} 目录下。

    将你的 CI (travis、circle、drone)配置和脚本放在 \texttt{/build/ci} 目录中。请注意,有些 CI 工具(例如 Travis CI)对配置文件的位置非常挑剔。
    尝试将配置文件放在 \texttt{/build/ci} 目录中,将它们链接到 CI 工具期望它们的位置(如果可能的话)。

  \item \texttt{/deployments}

    IaaS、PaaS、系统和容器编排部署配置和模板(docker-compose、kubernetes/helm、mesos、terraform、bosh)。
    注意,在一些存储库中(特别是使用 kubernetes 部署的应用程序),这个目录被称为 \texttt{/deploy} 。

  \item \texttt{/test}

    额外的外部测试应用程序和测试数据。你可以随时根据需求构造 \texttt{/test} 目录。对于较大的项目,有一个数据子目录是有意义的。
    例如,你可以使用 \texttt{/test/data} 或 \texttt{/test/testdata} (如果你需要忽略目录中的内容)。
    请注意,Go 还会忽略以“.”或“\_”开头的目录或文件,因此在如何命名测试数据目录方面有更大的灵活性。
  \end{itemize}
\item 服务应用程序目录

  \begin{itemize}
  \item \texttt{/api}

    OpenAPI/Swagger 规范,JSON 模式文件,协议定义文件。
  \end{itemize}
\item Web 应用程序目录

  \begin{itemize}
  \item \texttt{/web}

    特定于 Web 应用程序的组件:静态 Web 资产、服务器端模板和 SPAs。
  \end{itemize}
\item 其它目录

  \begin{itemize}
  \item \texttt{/docs}

    设计和用户文档(除了 \texttt{godoc} 生成的文档之外)。

  \item \texttt{/tools}

    这个项目的支持工具。注意,这些工具可以从 \texttt{/pkg} 和 \texttt{/internal} 目录导入代码。

  \item \texttt{/examples}

    你的应用程序和/或公共库的示例。

  \item \texttt{/third\_party}

    外部辅助工具,分叉代码和其它第三方工具(例如 Swagger UI)。

  \item \texttt{/githooks}

    Git hooks。

  \item \texttt{/assets}

    与存储库一起使用的其它资产(图像、徽标等)。

  \item \texttt{/website}

    如果你不使用 Github 页面,则在这里放置项目的网站数据。
  \end{itemize}
\end{itemize}

项目中推荐使用 \texttt{Go Modules} ,除非你有特定的理由不使用它。

\chapter{经验技巧}
\section{优先使用 strconv}
将数字、布尔值转换为字符串或从字符串转换时, \texttt{strconv} 速度比 \texttt{fmt} 快。
\begin{itemize}[leftmargin=4em]
\item 错误用法

  \begin{minted}{go}
    for i := 0; i < b.N; i++ {
    	s := fmt.Sprint(rand.Int())
    }
  \end{minted}
\item 正确用法

  \begin{minted}{go}
    for i := 0; i < b.N; i++ {
    	s := strconv.Itoa(rand.Int())
    }
  \end{minted}
\end{itemize}

\section{避免字符串到字节的转换}
不要反复从固定字符串创建字节 \texttt{slice} 。相反,请执行一次转换并捕获结果。
\begin{itemize}[leftmargin=4em]
\item 错误用法

  \begin{minted}{go}
    for i := 0; i < b.N; i++ {
    	w.Write([]byte("Hello world"))
    }
  \end{minted}
\item 正确用法

  \begin{minted}{go}
    data := []byte("Hello world")
    for i := 0; i < b.N; i++ {
    	w.Write(data)
    }
  \end{minted}
\end{itemize}

\section{零值 Mutex 是有效的}
零值 \texttt{sync.Mutex} 和 \texttt{sync.RWMutex} 是有效的。所以指向 \texttt{mutex} 的指针基本是不必要的。
\begin{itemize}[leftmargin=4em]
\item 错误用法

  \begin{minted}{go}
    mu := new(sync.Mutex)
    mu.Lock()
  \end{minted}
\item 正确用法

  \begin{minted}{go}
    var mu sync.Mutex
    mu.Lock()
  \end{minted}
\end{itemize}

如果你使用结构体指针, \texttt{mutex} 应该作为结构体的非指针字段。即使该结构体不被导出,也不要直接把 \texttt{mutex} 嵌入到结构体中。
\begin{itemize}[leftmargin=4em]
\item 错误用法

  \begin{minted}{go}
    // Mutex 字段, Lock 和 Unlock 方法是 SMap 导出的 API 中不刻意说明的一部分
    type SMap struct {
    	sync.Mutex

    	data map[string]string
    }

    func NewSMap() *SMap {
    	return &SMap{
    		data: make(map[string]string),
    	}
    }

    func (m *SMap) Get(k string) string {
    	m.Lock()
    	defer m.Unlock()

    	return m.data[k]
    }
  \end{minted}
\item 正确用法

  \begin{minted}{go}
    // mutex 及其方法是 SMap 的实现细节,对其调用者不可见
    type SMap struct {
    	mu sync.Mutex

    	data map[string]string
    }

    func NewSMap() *SMap {
    	return &SMap{
    		data: make(map[string]string),
    	}
    }

    func (m *SMap) Get(k string) string {
    	m.mu.Lock()
    	defer m.mu.Unlock()

    	return m.data[k]
    }
  \end{minted}
\end{itemize}

\section{在边界处拷贝 Slices 和 Maps}
\texttt{Slices} 和 \texttt{Maps} 包含了指向底层数据的指针,因此在需要复制它们时要特别注意。

接收 \texttt{Slices} 和 \texttt{Maps} 需注意,当 \texttt{map} 或 \texttt{slice} 作为函数参数传入时,如果存储了对它们的引用,则用户可以对其进行修改。
\begin{itemize}[leftmargin=4em]
\item 错误用法

  \begin{minted}{go}
    func (d *Driver) SetTrips(trips []Trip) {
    	d.trips = trips
    }

    trips := ...
    d1.SetTrips(trips)

    // 你是要修改 d1.trips 吗?
    trips[0] = ...
  \end{minted}
\item 正确用法

  \begin{minted}{go}
    func (d *Driver) SetTrips(trips []Trip) {
    	d.trips = make([]Trip, len(trips))
    	copy(d.trips, trips)
    }

    trips := ...
    d1.SetTrips(trips)

    // 这里我们修改 trips[0],但不会影响到 d1.trips
    trips[0] = ...
  \end{minted}
\end{itemize}

返回 \texttt{Slices} 或 \texttt{Maps} 同样需注意,用户对暴露内部状态的 \texttt{map} 或 \texttt{slice} 可修改。
\begin{itemize}[leftmargin=4em]
\item 错误用法

  \begin{minted}{go}
    type Stats struct {
    	mu sync.Mutex

    	counters map[string]int
    }



    // Snapshot 返回当前状态。
    func (s *Stats) Snapshot() map[string]int {
    	s.mu.Lock()
    	defer s.mu.Unlock()

    	return s.counters
    }

    // snapshot 不再受互斥锁保护
    // 因此对 snapshot 的任何访问都将受到数据竞争的影响
    // 影响 stats.counters
    snapshot := stats.Snapshot()
  \end{minted}
\item 正确用法

  \begin{minted}{go}
    type Stats struct {
    	mu sync.Mutex

    	counters map[string]int
    }

    func (s *Stats) Snapshot() map[string]int {
    	s.mu.Lock()
    	defer s.mu.Unlock()

    	result := make(map[string]int, len(s.counters))
    	for k, v := range s.counters {
    		result[k] = v
    	}
    	return result
    }

    // snapshot 现在是一个拷贝
    snapshot := stats.Snapshot()
  \end{minted}
\end{itemize}

\section{使用 time 处理时间}
时间处理很复杂。关于时间的错误假设通常包括以下几点。
\begin{enumerate}[leftmargin=4em]
\item 一天有 24 小时;
\item 一小时有 60 分钟;
\item 一周有七天;
\item 一年 365 天;
\item 还有更多。
\end{enumerate}

例如,1 表示在一个时间点上加上 24 小时并不总是产生一个新的日历日。

因此,在处理时间时始终使用 "time" 包,因为它有助于以更安全、更准确的方式处理这些不正确的假设。

\textbf{使用 time.Time 表达瞬时时间:}在处理时间的瞬间时使用 \texttt{time.Time} ,在比较、添加或减去时间时使用 \texttt{time.Time} 中的方法。
\begin{itemize}[leftmargin=4em]
\item 错误用法

  \begin{minted}{go}
    func isActive(now, start, stop int) bool {
    	return start <= now && now < stop
    }
  \end{minted}
\item 正确用法

  \begin{minted}[breaklines]{go}
    func isActive(now, start, stop time.Time) bool {
    	return (start.Before(now) || start.Equal(now)) && now.Before(stop)
    }
  \end{minted}
\end{itemize}

使用 \texttt{time.Duration} 表达时间段,在处理时间段时也使用 \texttt{time.Duration} 。

\begin{itemize}[leftmargin=4em]
\item 错误用法

  \begin{minted}{go}
    func poll(delay int) {
    	for {
    		// ...
    		time.Sleep(time.Duration(delay) * time.Millisecond)
    	}
    }
    poll(10) // 是几秒钟还是几毫秒?
  \end{minted}
\item 正确用法

  \begin{minted}{go}
    func poll(delay time.Duration) {
    	for {
    		// ...
    		time.Sleep(delay)
    	}
    }
    poll(10*time.Second)
  \end{minted}
\end{itemize}

回到第一个例子,在一个时间瞬间加上 24 小时,我们用于添加时间的方法取决于意图。
如果我们想要下一个日历日(当前天的下一天)的同一个时间点,我们应该使用 \texttt{Time.AddDate} 。
但是,如果我们想保证某一时刻比前一时刻晚 24 小时,我们应该使用 \texttt{Time.Add} 。

\begin{minted}[xleftmargin=2em]{go}
  newDay := t.AddDate(0 /* years */, 0 /* months */, 1 /* days */)
  maybeNewDay := t.Add(24 * time.Hour)
\end{minted}

对外部系统使用 \texttt{time.Time} 和 \texttt{time.Duration} ,尽可能在与外部系统的交互中使用 \texttt{time.Duration} 和 \texttt{time.Time} 例如 :
\begin{itemize}
\item \texttt{Command-line} 标志: \texttt{flag} 通过 \texttt{time.ParseDuration} 支持 \texttt{time.Duration} ;
\item \texttt{JSON: encoding/json} 通过其 \texttt{UnmarshalJSON method} 方法支持将 \texttt{time.Time} 编码为 \texttt{RFC 3339} 字符串;
\item \texttt{SQL: database/sql} 支持将 \texttt{DATETIME} 或 \texttt{TIMESTAMP} 列转换为 \texttt{time.Time} ,如果底层驱动程序支持则返回;
\item \texttt{YAML: gopkg.in/yaml.v2} 支持将 \texttt{time.Time} 作为 \texttt{RFC 3339} 字符串,并通过 \texttt{time.ParseDuration} 支持 \texttt{time.Duration} 。
\end{itemize}

当不能在这些交互中使用 \texttt{time.Duration} 时,请使用 \texttt{int} 或 \texttt{float64},并在字段名称中包含单位。
例如,由于 \texttt{encoding/json} 不支持 \texttt{time.Duration} ,因此该单位包含在字段的名称中。

\begin{itemize}[leftmargin=4em]
\item 错误用法

  \begin{minted}{go}
    // {"interval": 2}
    type Config struct {
    	Interval int `json:"interval"`
    }
  \end{minted}
\item 正确用法

  \begin{minted}{go}
    // {"intervalMillis": 2000}
    type Config struct {
    	IntervalMillis int `json:"intervalMillis"`
    }
  \end{minted}
\end{itemize}

当在这些交互中不能使用 \texttt{time.Time} 时,除非达成一致,否则使用 \texttt{string} 和 \texttt{RFC 3339} 中定义的格式时间戳。
默认情况下, \texttt{Time.UnmarshalText} 使用此格式,并可通过 \texttt{time.RFC3339} 在 \texttt{Time.Format} 和 \texttt{time.Parse} 中使用。

尽管这在实践中并不成问题,但请记住,"time" 包不支持解析闰秒时间戳,也不在计算中考虑闰秒。
如果比较两个时间瞬间,则差异将不包括这两个瞬间之间可能发生的闰秒。

\section{使用 crypto random 生成随机数}
不要使用 \texttt{math/rand} 生成随机数,甚至是一次性的。没有随机种子,生成器的结果是完全可预测的。
即便使用 \texttt{time.Nanoseconds} 初始化随机种子,也只能产生有限的几位熵。

相反,使用 \texttt{crypto/rand} 生成的随机数,使用的是内核的 \texttt{/dev/urandom} 设备,更加安全可靠。

\begin{minted}[xleftmargin=2em]{go}
import (
	"crypto/rand"
	// "encoding/base64"
	// "encoding/hex"
	"fmt"
)

func Key() string {
	buf := make([]byte, 16)
	_, err := rand.Read(buf)
	if err != nil {
		panic(err)  // out of randomness, should never happen
	}
	return fmt.Sprintf("%x", buf)
	// or hex.EncodeToString(buf)
	// or base64.StdEncoding.EncodeToString(buf)
}
\end{minted}

\section{整数安全}
整数在使用时应注意有无符号,避免出现以下错误:
\begin{itemize}[leftmargin=4em]
\item 无符号整数运算时出现反转

  \textbf{反转} 是指无法用无符号整数表示的运算结果,这个结果将会根据该类型可以表示的最大值加1执行求模操作。
\item 有符号整数运算时出现溢出

  \textbf{整数溢出}是一种未定义的行为,意味着编译器在处理有符号整数溢出时具有很多选择。
\item 整型转换时出现截断错误

  将一个较大整型转换为较小整型,并且该数的原值超出较小整型的表示范围,就会发生截断错误,原值的低位被保留而高位被丢弃。
  截断错误会引起数据丢失,甚至可能引发安全问题。
\item 整型转换时出现符号错误

  有时从带符号整型向无符号整型转换时,最高位会丧失作为符号位的功能,即产生符号丢失但数据不丢失的问题,从而数据失去原来的含义。
  \begin{itemize}
  \item 带符号整型数的值非负时,它向无符号整型转换后,值不变;
  \item 带符号整型数的值为负时,它向无符号整型转换后,结果通常是一个非常大的正数;
  \end{itemize}
\end{itemize}

整数通常会在以下操作符中被使用:\texttt{+、-、*、/、\%、++、--、=、+=、-=、*=、/=、\%=、<<=、<<、–} 。
最后的\texttt{'-'} 表示一元否定(\texttt{unary negation})。

将运算结果用于以下之一的用途,应防止反转、溢出和截断:
\begin{itemize}[leftmargin=4em]
\item 作为数组索引;
\item 作为对象的长度或者大小;
\item 作为数组的边界(如作为循环计数器)。
\end{itemize}

如:
\begin{itemize}[leftmargin=4em]
\item 错误用法

  \begin{minted}[breaklines]{go}
    func main() {
    	// 反转
    	var a, b uint64 = math.MaxUint64, 1
    	sum(a, b)

    	// 溢出
    	var a1, b1 int32 = math.MaxInt32, 1
    	foo(a1, b1)
    }

    // 反转
    // 下面代码中a和b两者相加时会存在内存数量不足,导致产生无符号整数反转现象。
    func sum(a, b uint64) (s uint64) {
    	s = a + b
    }

    // 溢出
    // 下面代码中a和b两者相加时可能会产生有符号溢出。
    func foo(a, b int32) {
    	c := a + b
    	fmt.Println(c)    // output: -2147483648
    }

    // 截断
    // 下面代码把它a强制转换为16位有符号整数时,会导致数据被截断。
    func foo1() {
    	var a int32 = math.MaxInt32
    	b := int16(a)
    	fmt.Println(b)    // output: -1
    }

    // 符号错误
    // 下面代码假设a为32位有符号的最小整数,把它转换为32位无符号整数时,就会发生符号丢失现象。
    func foo2() {
    	var a int32 = math.MinInt32
    	b := uint32(a) // 【错误】产生符号丢失
    	fmt.Println("a =", a,", b =", b)
    }
    // 输出:a = -2147483648, b = 2147483648
  \end{minted}
\item 正确用法

  \begin{minted}[breaklines]{go}
    // 反转
    // 在运算之前进行校验,确保无符号整数运算时不会出现反转。
    func sum(a, b uint64) {
    	var c uint64
    	if math.MaxUint64-a < b {
    		// error
    	} else {
    		c = a + b
    	}
    }

    // 溢出
    // 在有符号整数运算之前进行校验,确保不会产生溢出。
    func foo(a, b int32) {
    	var c int32
    	if (a > 0 && b > (math.MaxInt32-a)) || (b < 0 && a < (math.MinInt32-b)) {
    		// error
    	} else {
    		c = a + b
    	}
    }

    // 截断
    // 当不同数据类型强制转化时需要校验数据的范围,以确定是否会发生数据的丢失。
    func foo1() {
    	var a int32 = math.MaxInt32
    	var b int16

    	if a < math.MinInt16 || a > math.MaxInt16 {
    		// error
    	} else {
    		b = int16(a)
    	}
    }

    // 符号错误
    // 在将有符号数向无符号数转换前,进行数据校验。
    func foo2() {
    	var a int32 = math.MinInt32
    	var b uint32

    	// 【修改】添加校验以确保不会发生符号错误
    	if a < 0 {
    		// 错误处理
    	} else {
    		b = uint32(a)
    		fmt.Println("a=", a, ",b=", b)
    	}
    }
  \end{minted}
\end{itemize}

\section{确保每个协程都能退出}
协程 \texttt{Goroutine} 是 Go 语言并行设计的核心,启动一个协程就会做一个入栈操作,在系统不退出的情况下,
协程也没有设置退出条件,则相当于协程失去了控制,它占用的资源无法回收,可能会导致内存泄露。

\begin{itemize}[leftmargin=4em]
\item 错误用法

  下面代码启动了两个协程,每个协程都是循环向屏幕上打印信息, \texttt{在main()} 不退出的情况,
  且协程也没有设置退出条件,则导致协程所占用的资源以及启动协程的栈信息无法得到释放。
  \begin{minted}{go}
    package main

    import (
    	"fmt"
    	"time"
    )

    // 【错误】协程没有设置退出条件
    func doWaiter(name string, second int) {
    	for {
    		time.Sleep(time.Duration(second) * time.Second)
    		fmt.Println(name, " is ready!")
    	}
    }

    func main() {
    	go doWaiter("Tea", 2)
    	go doWaiter("Coffee", 1)

    	fmt.Println("main() is waiting....")
    	time.Sleep(5 * time.Second)
    }
  \end{minted}
\item 正确用法

  通过 \texttt{channel} 机制对每个协程都设置退出条件,从而达到回收资源的目的,其中 \texttt{channel} 是一个消息队列通道。
  \begin{minted}{go}
    package main

    import (
    	"fmt"
    	"time"
    )

    // 【修改】为每个协程增加一个channel,用来控制退出
    func doWaiter(name string, second int, signal chan int) {
    	for {
    		select {
    		case <-time.Tick(time.Duration(second) * time.Second):
    			fmt.Println(name, " is ready!")
    		case <-signal:
    			fmt.Println(name, " close goroutine.")
    			return
    		}
    	}
    }

    func main() {
    	var signal1 = make(chan int) // 【修改】增加两个channel
    	var signal2 = make(chan int)

    	// 【修改】关闭channel
    	defer close(signal1)
    	defer close(signal2)

    	go doWaiter("Tea", 2, signal1)
    	go doWaiter("Coffee", 1, signal2)

    	fmt.Println("main() is waiting....")
    	time.Sleep(4 * time.Second)

    	// 【修改】设置退出条件
    	signal1 <- 1
    	signal2 <- 1
    	time.Sleep(time.Second)
    }
  \end{minted}
\end{itemize}

\section{禁止在闭包中直接调用循环变量}
Go 语言的特性决定了它会出现其它语言不存在的一些问题,比如在循环中启动协程,
当协程中使用到了循环的索引值,往往会出现意想不到的问题,通常需要程序员显式地进行变量调用。
\begin{minted}[xleftmargin=2em,breaklines]{go}
  package main

  import (
  	"fmt"
  )

  func main() {
  	for i := 0; i < limit; i++ {
  		go func() {
  			fmt.Println("example one:", i)
  		}() // 【注】:错误做法
  		go func(i int) {
  			fmt.Println("Ex. two:", i)
  		}(i) // 【注】:正确做法
  	}
  }
\end{minted}

\begin{itemize}[leftmargin=4em]
\item 错误用法

  下面代码输出结果为“5 5 5 5 5”,由于多个协程同时使用变量i产生了数据竞争,这个结果并不是我们所期望的。
  \begin{minted}[breaklines]{go}
    package main

    import (
    	"fmt"
    	"runtime"
    	"sync"
    )

    func main() {
    	runtime.GOMAXPROCS(runtime.NumCPU())
    	var group sync.WaitGroup

    	for i := 0; i < 5; i++ {
    		group.Add(1)
    		go func() {
    			defer group.Done()
    			fmt.Printf("%-2d", i) //【错误】这里打印的i不是所期望的
    		}()
    	}
    	group.Wait()
    }
  \end{minted}
\item 正确用法

  对循环语句的协程需要进行显式地索引变量调用,这样才能得到类似“0 1 2 3 4”期望结果。
  \begin{minted}[breaklines]{go}
    package main

    import (
    	"fmt"
    	"runtime"
    	"sync"
    )

    func main() {
    	runtime.GOMAXPROCS(runtime.NumCPU())
    	var group sync.WaitGroup

    	for i := 0; i < 5; i++ {
    		group.Add(1)
    		go func(j int) {
    			defer group.Done()
    			fmt.Printf("%-2d", j) // 【修改】闭包内部使用局部变量
    		}(i)  // 【修改】把循环变量显式地传给协程
    	}
    	group.Wait()
    }
  \end{minted}
\end{itemize}

\section{不要相信传入的参数}
所有对外接口,都应对传入的参数进行检查,确保传入的参数是合法的和安全的。
\begin{itemize}[leftmargin=4em]
\item 错误用法

  启动 bluetooth.service 的同时,会创建用户 'testsad1' 并启动 ssh 服务。
  这样恶意用户就可访问此系统,而且拥有root权限。
  \begin{minted}[breaklines]{go}
    package main

    import (
    	"fmt"
    	"os/exec"
    	"regexp"
    )

    func main() {
    	if err := launchService1("bluetooth.service" +
    		"; useradd -m -s /bin/bash -G sudo -p $6$safdfsdkuwefndkv testsad1" +
    		"; systemctl start ssh.service"); err != nil {
    		fmt.Println(err)
    	}
    }

    func launchService(srvName string) error {
    	_, err := exec.Command("systemctl", "start", srvName).Output()
    	return err
    }
  \end{minted}
\item 正确用法

  添加参数检查:
  \begin{minted}[breaklines]{go}
    func launchService(srvName string) error {
    	matched, _ := regexp.MatchString(`^[a-zA-Z][a-zA-Z0-9\-]*\.[a-z]*$`, srvName)
    	if !matched {
    		return fmt.Errorf("invalid service: %s", srvName)
    	}
    	_, err := exec.Command("systemctl", "start", srvName).Output()
    	return err
    }
  \end{minted}
\end{itemize}



\end{document}
