9. 格式
------------

每个人都可能有自己的代码风格和格式, 但如果一个项目中的所有人都遵循同一风格的话, 这个项目就能更顺利地进行. 每个人未必能同意下述的每一处格式规则, 而且其中的不少规则需要一定时间的适应, 但整个项目服从统一的编程风格是很重要的, 只有这样才能让所有人轻松地阅读和理解代码.

为了帮助你正确的格式化代码, 我们写了一个 `emacs 配置文件 <https://raw.githubusercontent.com/google/styleguide/gh-pages/google-c-style.el>`_.

.. _line-length:

9.1. 行长度
~~~~~~~~~~~~~~~~~~~~

**总述**

每一行代码字符数不超过 80.

我们也认识到这条规则是有争议的, 但很多已有代码都遵照这一规则, 因此我们感觉一致性更重要.

**优点**

提倡该原则的人认为强迫他们调整编辑器窗口大小是很野蛮的行为. 很多人同时并排开几个代码窗口, 根本没有多余的空间拉伸窗口. 大家都把窗口最大尺寸加以限定, 并且 80 列宽是传统标准. 那么为什么要改变呢?

**缺点**

反对该原则的人则认为更宽的代码行更易阅读. 80 列的限制是上个世纪 60 年代的大型机的古板缺陷; 现代设备具有更宽的显示屏, 可以很轻松地显示更多代码.

**结论**

80 个字符是最大值.

如果无法在不伤害易读性的条件下进行断行, 那么注释行可以超过 80 个字符, 这样可以方便复制粘贴. 例如, 带有命令示例或 URL 的行可以超过 80 个字符.

包含长路径的 ``#include`` 语句可以超出80列.

:ref:`头文件保护 <define-guard>` 可以无视该原则.

9.2. 非 ASCII 字符
~~~~~~~~~~~~~~~~~~~~~~~~~~~~~~~~

**总述**

尽量不使用非 ASCII 字符, 使用时必须使用 UTF-8 编码.

**说明**

即使是英文, 也不应将用户界面的文本硬编码到源代码中, 因此非 ASCII 字符应当很少被用到. 特殊情况下可以适当包含此类字符. 例如, 代码分析外部数据文件时, 可以适当硬编码数据文件中作为分隔符的非 ASCII 字符串; 更常见的是 (不需要本地化的) 单元测试代码可能包含非 ASCII 字符串. 此类情况下, 应使用 UTF-8 编码, 因为很多工具都可以理解和处理 UTF-8 编码.

十六进制编码也可以, 能增强可读性的情况下尤其鼓励 —— 比如 ``"\xEF\xBB\xBF"``, 或者更简洁地写作 ``u8"\uFEFF"``, 在 Unicode 中是 *零宽度 无间断* 的间隔符号, 如果不用十六进制直接放在 UTF-8 格式的源文件中, 是看不到的.

(Yang.Y 注: ``"\xEF\xBB\xBF"`` 通常用作 UTF-8 with BOM 编码标记)

使用 ``u8`` 前缀把带 ``uXXXX`` 转义序列的字符串字面值编码成 UTF-8. 不要用在本身就带 UTF-8 字符的字符串字面值上, 因为如果编译器不把源代码识别成 UTF-8, 输出就会出错.

别用 C++11 的 ``char16_t`` 和 ``char32_t``, 它们和 UTF-8 文本没有关系, ``wchar_t`` 同理, 除非你写的代码要调用 Windows API, 后者广泛使用了 ``wchar_t``.

9.3. 空格还是制表位
~~~~~~~~~~~~~~~~~~~~~~~~~~~~~~~~

**总述**

只使用空格, 每次缩进 2 个空格.

**说明**

我们使用空格缩进. 不要在代码中使用制表符. 你应该设置编辑器将制表符转为空格.

9.4. 函数声明与定义
~~~~~~~~~~~~~~~~~~~~~~~~~~~~~~~~

**总述**

返回类型和函数名在同一行, 参数也尽量放在同一行, 如果放不下就对形参分行, 分行方式与 :ref:`函数调用 <function-calls>` 一致.

**说明**

函数看上去像这样:

.. code-block:: c++

    ReturnType ClassName::FunctionName(Type par_name1, Type par_name2) {
      DoSomething();
      ...
    }

如果同一行文本太多, 放不下所有参数:

.. code-block:: c++

    ReturnType ClassName::ReallyLongFunctionName(Type par_name1, Type par_name2,
                                                 Type par_name3) {
      DoSomething();
      ...
    }

甚至连第一个参数都放不下:

.. code-block:: c++

    ReturnType LongClassName::ReallyReallyReallyLongFunctionName(
        Type par_name1,  // 4 space indent
        Type par_name2,
        Type par_name3) {
      DoSomething();  // 2 space indent
      ...
    }

注意以下几点:

- 使用好的参数名.

- 只有在参数未被使用或者其用途非常明显时, 才能省略参数名.

- 如果返回类型和函数名在一行放不下, 分行.

- 如果返回类型与函数声明或定义分行了, 不要缩进.

- 左圆括号总是和函数名在同一行.

- 函数名和左圆括号间永远没有空格.

- 圆括号与参数间没有空格.

- 左大括号总在最后一个参数同一行的末尾处, 不另起新行.

- 右大括号总是单独位于函数最后一行, 或者与左大括号同一行.

- 右圆括号和左大括号间总是有一个空格.

- 所有形参应尽可能对齐.

- 缺省缩进为 2 个空格.

- 换行后的参数保持 4 个空格的缩进.

未被使用的参数, 或者根据上下文很容易看出其用途的参数, 可以省略参数名:

.. code-block:: c++

    class Foo {
     public:
      Foo(Foo&&);
      Foo(const Foo&);
      Foo& operator=(Foo&&);
      Foo& operator=(const Foo&);
    };

未被使用的参数如果其用途不明显的话, 在函数定义处将参数名注释起来:

.. code-block:: c++

    class Shape {
     public:
      virtual void Rotate(double radians) = 0;
    };

    class Circle : public Shape {
     public:
      void Rotate(double radians) override;
    };

    void Circle::Rotate(double /*radians*/) {}

.. code-block:: c++

    // 差 - 如果将来有人要实现, 很难猜出变量的作用.
    void Circle::Rotate(double) {}

属性, 和展开为属性的宏, 写在函数声明或定义的最前面, 即返回类型之前:

.. code-block:: c++

    MUST_USE_RESULT bool IsOK();

9.5. Lambda 表达式
~~~~~~~~~~~~~~~~~~~~~~~~~~~~~~~~~~

**总述**

Lambda 表达式对形参和函数体的格式化和其他函数一致; 捕获列表同理, 表项用逗号隔开.

**说明**

若用引用捕获, 在变量名和 ``&`` 之间不留空格.

.. code-block:: c++

    int x = 0;
    auto add_to_x = [&x](int n) { x += n; };

短 lambda 就写得和内联函数一样.

.. code-block:: c++

    std::set<int> blacklist = {7, 8, 9};
    std::vector<int> digits = {3, 9, 1, 8, 4, 7, 1};
    digits.erase(std::remove_if(digits.begin(), digits.end(), [&blacklist](int i) {
                   return blacklist.find(i) != blacklist.end();
                 }),
                 digits.end());

.. _function-calls:

9.6. 函数调用
~~~~~~~~~~~~~~~~~~~~~~

**总述**

要么一行写完函数调用, 要么在圆括号里对参数分行, 要么参数另起一行且缩进四格. 如果没有其它顾虑的话, 尽可能精简行数, 比如把多个参数适当地放在同一行里.

**说明**

函数调用遵循如下形式:

.. code-block:: c++

    bool retval = DoSomething(argument1, argument2, argument3);

如果同一行放不下, 可断为多行, 后面每一行都和第一个实参对齐, 左圆括号后和右圆括号前不要留空格:

.. code-block:: c++

    bool retval = DoSomething(averyveryveryverylongargument1,
                              argument2, argument3);

参数也可以放在次行, 缩进四格:

.. code-block:: c++

    if (...) {
      ...
      ...
      if (...) {
        DoSomething(
            argument1, argument2,  // 4 空格缩进
            argument3, argument4);
      }

把多个参数放在同一行以减少函数调用所需的行数, 除非影响到可读性. 有人认为把每个参数都独立成行, 不仅更好读, 而且方便编辑参数. 不过, 比起所谓的参数编辑, 我们更看重可读性, 且后者比较好办:

如果一些参数本身就是略复杂的表达式, 且降低了可读性, 那么可以直接创建临时变量描述该表达式, 并传递给函数:

.. code-block:: c++

    int my_heuristic = scores[x] * y + bases[x];
    bool retval = DoSomething(my_heuristic, x, y, z);

或者放着不管, 补充上注释:

.. code-block:: c++

    bool retval = DoSomething(scores[x] * y + bases[x],  // Score heuristic.
                              x, y, z);

如果某参数独立成行, 对可读性更有帮助的话, 那也可以如此做. 参数的格式处理应当以可读性而非其他作为最重要的原则.

此外, 如果一系列参数本身就有一定的结构, 可以酌情地按其结构来决定参数格式:

.. code-block:: c++

    // 通过 3x3 矩阵转换 widget.
    my_widget.Transform(x1, x2, x3,
                        y1, y2, y3,
                        z1, z2, z3);

.. _braced-initializer-list-format:

9.7. 列表初始化格式
~~~~~~~~~~~~~~~~~~~~~~~~~~~~~~~~

**总述**

您平时怎么格式化函数调用, 就怎么格式化 :ref:`列表初始化 <braced-initializer-list>`.

**说明**

如果列表初始化伴随着名字, 比如类型或变量名, 格式化时将将名字视作函数调用名, `{}` 视作函数调用的括号. 如果没有名字, 就视作名字长度为零.

.. code-block:: c++

    // 一行列表初始化示范.
    return {foo, bar};
    functioncall({foo, bar});
    pair<int, int> p{foo, bar};

    // 当不得不断行时.
    SomeFunction(
        {"assume a zero-length name before {"},  // 假设在 { 前有长度为零的名字.
        some_other_function_parameter);
    SomeType variable{
        some, other, values,
        {"assume a zero-length name before {"},  // 假设在 { 前有长度为零的名字.
        SomeOtherType{
            "Very long string requiring the surrounding breaks.",  // 非常长的字符串, 前后都需要断行.
            some, other values},
        SomeOtherType{"Slightly shorter string",  // 稍短的字符串.
                      some, other, values}};
    SomeType variable{
        "This is too long to fit all in one line"};  // 字符串过长, 因此无法放在同一行.
    MyType m = {  // 注意了, 您可以在 { 前断行.
        superlongvariablename1,
        superlongvariablename2,
        {short, interior, list},
        {interiorwrappinglist,
         interiorwrappinglist2}};

9.8. 条件语句
~~~~~~~~~~~~~~~~~~~~~~

**总述**

倾向于不在圆括号内使用空格. 关键字 ``if`` 和 ``else`` 另起一行.

**说明**

对基本条件语句有两种可以接受的格式. 一种在圆括号和条件之间有空格, 另一种没有.

最常见的是没有空格的格式. 哪一种都可以, 最重要的是 *保持一致*. 如果你是在修改一个文件, 参考当前已有格式. 如果是写新的代码, 参考目录下或项目中其它文件. 还在犹豫的话, 就不要加空格了.

.. code-block:: c++

    if (condition) {  // 圆括号里没有空格.
      ...  // 2 空格缩进.
    } else if (...) {  // else 与 if 的右括号同一行.
      ...
    } else {
      ...
    }

如果你更喜欢在圆括号内部加空格:

.. code-block:: c++

    if ( condition ) {  // 圆括号与空格紧邻 - 不常见
      ...  // 2 空格缩进.
    } else {  // else 与 if 的右括号同一行.
      ...
    }

注意所有情况下 ``if`` 和左圆括号间都有个空格. 右圆括号和左大括号之间也要有个空格:

.. code-block:: c++

    if(condition)     // 差 - IF 后面没空格.
    if (condition){   // 差 - { 前面没空格.
    if(condition){    // 变本加厉地差.

.. code-block:: c++

    if (condition) {  // 好 - IF 和 { 都与空格紧邻.

如果能增强可读性, 简短的条件语句允许写在同一行. 只有当语句简单并且没有使用 ``else`` 子句时使用:

.. code-block:: c++

    if (x == kFoo) return new Foo();
    if (x == kBar) return new Bar();

如果语句有 ``else`` 分支则不允许:

.. code-block:: c++

    // 不允许 - 当有 ELSE 分支时 IF 块却写在同一行
    if (x) DoThis();
    else DoThat();

通常, 单行语句不需要使用大括号, 如果你喜欢用也没问题; 复杂的条件或循环语句用大括号可读性会更好. 也有一些项目要求 ``if`` 必须总是使用大括号:

.. code-block:: c++

    if (condition)
      DoSomething();  // 2 空格缩进.

    if (condition) {
      DoSomething();  // 2 空格缩进.
    }

但如果语句中某个 ``if-else`` 分支使用了大括号的话, 其它分支也必须使用:

.. code-block:: c++

    // 不可以这样子 - IF 有大括号 ELSE 却没有.
    if (condition) {
      foo;
    } else
      bar;

    // 不可以这样子 - ELSE 有大括号 IF 却没有.
    if (condition)
      foo;
    else {
      bar;
    }


.. code-block:: c++

    // 只要其中一个分支用了大括号, 两个分支都要用上大括号.
    if (condition) {
      foo;
    } else {
      bar;
    }

9.9. 循环和开关选择语句
~~~~~~~~~~~~~~~~~~~~~~~~~~~~~~~~~~~~~~

**总述**

``switch`` 语句可以使用大括号分段, 以表明 cases 之间不是连在一起的. 在单语句循环里, 括号可用可不用. 空循环体应使用 ``{}`` 或 ``continue``.

**说明**

``switch`` 语句中的 ``case`` 块可以使用大括号也可以不用, 取决于你的个人喜好. 如果用的话, 要按照下文所述的方法.

如果有不满足 ``case`` 条件的枚举值, ``switch`` 应该总是包含一个 ``default`` 匹配 (如果有输入值没有 case 去处理, 编译器将给出 warning). 如果 ``default`` 应该永远执行不到, 简单的加条 ``assert``:

.. code-block:: c++

    switch (var) {
      case 0: {  // 2 空格缩进
        ...      // 4 空格缩进
        break;
      }
      case 1: {
        ...
        break;
      }
      default: {
        assert(false);
      }
    }

在单语句循环里, 括号可用可不用:

.. code-block:: c++

    for (int i = 0; i < kSomeNumber; ++i)
      printf("I love you\n");

    for (int i = 0; i < kSomeNumber; ++i) {
      printf("I take it back\n");
    }

空循环体应使用 ``{}`` 或 ``continue``, 而不是一个简单的分号.

.. code-block:: c++

    while (condition) {
      // 反复循环直到条件失效.
    }
    for (int i = 0; i < kSomeNumber; ++i) {}  // 可 - 空循环体.
    while (condition) continue;  // 可 - contunue 表明没有逻辑.

.. code-block:: c++

    while (condition);  // 差 - 看起来仅仅只是 while/loop 的部分之一.

9.10. 指针和引用表达式
~~~~~~~~~~~~~~~~~~~~~~~~~~~~~~~~~~~~~~

**总述**

句点或箭头前后不要有空格. 指针/地址操作符 (``*, &``) 之后不能有空格.

**说明**

下面是指针和引用表达式的正确使用范例:

.. code-block:: c++

    x = *p;
    p = &x;
    x = r.y;
    x = r->y;

注意:

- 在访问成员时, 句点或箭头前后没有空格.

- 指针操作符 ``*`` 或 ``&`` 后没有空格.

在声明指针变量或参数时, 星号与类型或变量名紧挨都可以:

.. code-block:: c++

    // 好, 空格前置.
    char *c;
    const string &str;

    // 好, 空格后置.
    char* c;
    const string& str;

.. code-block:: c++

    int x, *y;  // 不允许 - 在多重声明中不能使用 & 或 *
    char * c;  // 差 - * 两边都有空格
    const string & str;  // 差 - & 两边都有空格.

在单个文件内要保持风格一致, 所以, 如果是修改现有文件, 要遵照该文件的风格.

9.11. 布尔表达式
~~~~~~~~~~~~~~~~~~~~~~~~~~~~

**总述**

如果一个布尔表达式超过 :ref:`标准行宽 <line-length>`, 断行方式要统一一下.

**说明**

下例中, 逻辑与 (``&&``) 操作符总位于行尾:

.. code-block:: c++

    if (this_one_thing > this_other_thing &&
        a_third_thing == a_fourth_thing &&
        yet_another && last_one) {
      ...
    }

注意, 上例的逻辑与 (``&&``) 操作符均位于行尾. 这个格式在 Google 里很常见, 虽然把所有操作符放在开头也可以. 可以考虑额外插入圆括号, 合理使用的话对增强可读性是很有帮助的. 此外, 直接用符号形式的操作符, 比如 ``&&`` 和 ``~``, 不要用词语形式的 ``and`` 和 ``compl``.

9.12. 函数返回值
~~~~~~~~~~~~~~~~~~~~~~~~~~~~

**总述**

不要在 ``return`` 表达式里加上非必须的圆括号.

**说明**

只有在写 ``x = expr`` 要加上括号的时候才在 ``return expr;`` 里使用括号.

.. code-block:: c++

    return result;                  // 返回值很简单, 没有圆括号.
    // 可以用圆括号把复杂表达式圈起来, 改善可读性.
    return (some_long_condition &&
            another_condition);

.. code-block:: c++

    return (value);                // 毕竟您从来不会写 var = (value);
    return(result);                // return 可不是函数!

9.13. 变量及数组初始化
~~~~~~~~~~~~~~~~~~~~~~~~~~~~~~~~~~~~~~

**总述**

用 ``=``, ``()`` 和 ``{}`` 均可.

**说明**

您可以用 ``=``, ``()`` 和 ``{}``, 以下的例子都是正确的:

.. code-block:: c++

    int x = 3;
    int x(3);
    int x{3};
    string name("Some Name");
    string name = "Some Name";
    string name{"Some Name"};

请务必小心列表初始化 ``{...}`` 用 ``std::initializer_list`` 构造函数初始化出的类型. 非空列表初始化就会优先调用 ``std::initializer_list``, 不过空列表初始化除外, 后者原则上会调用默认构造函数. 为了强制禁用 ``std::initializer_list`` 构造函数, 请改用括号.

.. code-block:: c++

    vector<int> v(100, 1);  // 内容为 100 个 1 的向量.
    vector<int> v{100, 1};  // 内容为 100 和 1 的向量.

此外, 列表初始化不允许整型类型的四舍五入, 这可以用来避免一些类型上的编程失误. 

.. code-block:: c++

    int pi(3.14);  // 好 - pi == 3.
    int pi{3.14};  // 编译错误: 缩窄转换.

9.14. 预处理指令
~~~~~~~~~~~~~~~~~~~~~~~~~~~~

**总述**

预处理指令不要缩进, 从行首开始.

**说明**

即使预处理指令位于缩进代码块中, 指令也应从行首开始.

.. code-block:: c++

    // 好 - 指令从行首开始
      if (lopsided_score) {
    #if DISASTER_PENDING      // 正确 - 从行首开始
        DropEverything();
    # if NOTIFY               // 非必要 - # 后跟空格
        NotifyClient();
    # endif
    #endif
        BackToNormal();
      }

.. code-block:: c++

    // 差 - 指令缩进
      if (lopsided_score) {
        #if DISASTER_PENDING  // 差 - "#if" 应该放在行开头
        DropEverything();
        #endif                // 差 - "#endif" 不要缩进
        BackToNormal();
      }

9.15. 类格式
~~~~~~~~~~~~~~~~~~~~~~

**总述**

访问控制块的声明依次序是 ``public:``, ``protected:``, ``private:``, 每个都缩进 1 个空格.

**说明**

类声明 (下面的代码中缺少注释, 参考 :ref:`类注释 <class-comments>`) 的基本格式如下:

.. code-block:: c++

    class MyClass : public OtherClass {
     public:      // 注意有一个空格的缩进
      MyClass();  // 标准的两空格缩进
      explicit MyClass(int var);
      ~MyClass() {}

      void SomeFunction();
      void SomeFunctionThatDoesNothing() {
      }

      void set_some_var(int var) { some_var_ = var; }
      int some_var() const { return some_var_; }

     private:
      bool SomeInternalFunction();

      int some_var_;
      int some_other_var_;
    };

注意事项:

- 所有基类名应在 80 列限制下尽量与子类名放在同一行.

- 关键词 ``public:``, ``protected:``, ``private:`` 要缩进 1 个空格.

- 除第一个关键词 (一般是 ``public``) 外, 其他关键词前要空一行. 如果类比较小的话也可以不空.

- 这些关键词后不要保留空行.

- ``public`` 放在最前面, 然后是 ``protected``, 最后是 ``private``.

- 关于声明顺序的规则请参考 :ref:`声明顺序 <declaration-order>` 一节.

9.16. 构造函数初始值列表
~~~~~~~~~~~~~~~~~~~~~~~~~~~~

**总述**

构造函数初始化列表放在同一行或按四格缩进并排多行.

**说明**

下面两种初始值列表方式都可以接受:

.. code-block:: c++

    // 如果所有变量能放在同一行:
    MyClass::MyClass(int var) : some_var_(var) {
      DoSomething();
    }

    // 如果不能放在同一行,
    // 必须置于冒号后, 并缩进 4 个空格
    MyClass::MyClass(int var)
        : some_var_(var), some_other_var_(var + 1) {
      DoSomething();
    }

    // 如果初始化列表需要置于多行, 将每一个成员放在单独的一行
    // 并逐行对齐
    MyClass::MyClass(int var)
        : some_var_(var),             // 4 space indent
          some_other_var_(var + 1) {  // lined up
      DoSomething();
    }

    // 右大括号 } 可以和左大括号 { 放在同一行
    // 如果这样做合适的话
    MyClass::MyClass(int var)
        : some_var_(var) {}

9.17. 命名空间格式化
~~~~~~~~~~~~~~~~~~~~~~~~~~~~~~~~~~

**总述**

命名空间内容不缩进.

**说明**

:ref:`命名空间 <namespaces>` 不要增加额外的缩进层次, 例如:

.. code-block:: c++

    namespace {

    void foo() {  // 正确. 命名空间内没有额外的缩进.
      ...
    }

    }  // namespace

不要在命名空间内缩进:

.. code-block:: c++

    namespace {

      // 错, 缩进多余了.
      void foo() {
        ...
      }

    }  // namespace

声明嵌套命名空间时, 每个命名空间都独立成行.

.. code-block:: c++

    namespace foo {
    namespace bar {

9.19. 水平留白
~~~~~~~~~~~~~~~~~~~~~~~~

**总述**

水平留白的使用根据在代码中的位置决定. 永远不要在行尾添加没意义的留白.

**说明**

通用
=============================

.. code-block:: c++

    void f(bool b) {  // 左大括号前总是有空格.
      ...
    int i = 0;  // 分号前不加空格.
    // 列表初始化中大括号内的空格是可选的.
    // 如果加了空格, 那么两边都要加上.
    int x[] = { 0 };
    int x[] = {0};

    // 继承与初始化列表中的冒号前后恒有空格.
    class Foo : public Bar {
     public:
      // 对于单行函数的实现, 在大括号内加上空格
      // 然后是函数实现
      Foo(int b) : Bar(), baz_(b) {}  // 大括号里面是空的话, 不加空格.
      void Reset() { baz_ = 0; }  // 用空格把大括号与实现分开.
      ...

添加冗余的留白会给其他人编辑时造成额外负担. 因此, 行尾不要留空格. 如果确定一行代码已经修改完毕, 将多余的空格去掉; 或者在专门清理空格时去掉(尤其是在没有其他人在处理这件事的时候). (Yang.Y 注: 现在大部分代码编辑器稍加设置后, 都支持自动删除行首/行尾空格, 如果不支持, 考虑换一款编辑器或 IDE)

循环和条件语句
=============================

.. code-block:: c++

    if (b) {          // if 条件语句和循环语句关键字后均有空格.
    } else {          // else 前后有空格.
    }
    while (test) {}   // 圆括号内部不紧邻空格.
    switch (i) {
    for (int i = 0; i < 5; ++i) {
    switch ( i ) {    // 循环和条件语句的圆括号里可以与空格紧邻.
    if ( test ) {     // 圆括号, 但这很少见. 总之要一致.
    for ( int i = 0; i < 5; ++i ) {
    for ( ; i < 5 ; ++i) {  // 循环里内 ; 后恒有空格, ;  前可以加个空格.
    switch (i) {
      case 1:         // switch case 的冒号前无空格.
        ...
      case 2: break;  // 如果冒号有代码, 加个空格.

操作符
=============================

.. code-block:: c++

    // 赋值运算符前后总是有空格.
    x = 0;

    // 其它二元操作符也前后恒有空格, 不过对于表达式的子式可以不加空格.
    // 圆括号内部没有紧邻空格.
    v = w * x + y / z;
    v = w*x + y/z;
    v = w * (x + z);

    // 在参数和一元操作符之间不加空格.
    x = -5;
    ++x;
    if (x && !y)
      ...

模板和转换
=============================

.. code-block:: c++

    // 尖括号(< and >) 不与空格紧邻, < 前没有空格, > 和 ( 之间也没有.
    vector<string> x;
    y = static_cast<char*>(x);

    // 在类型与指针操作符之间留空格也可以, 但要保持一致.
    vector<char *> x;

9.19. 垂直留白
~~~~~~~~~~~~~~~~~~~~~~~~

**总述**

垂直留白越少越好.

**说明**

这不仅仅是规则而是原则问题了: 不在万不得已, 不要使用空行. 尤其是: 两个函数定义之间的空行不要超过 2 行, 函数体首尾不要留空行, 函数体中也不要随意添加空行.

基本原则是: 同一屏可以显示的代码越多, 越容易理解程序的控制流. 当然, 过于密集的代码块和过于疏松的代码块同样难看, 这取决于你的判断. 但通常是垂直留白越少越好.

下面的规则可以让加入的空行更有效:

- 函数体内开头或结尾的空行可读性微乎其微.

- 在多重 if-else 块里加空行或许有点可读性.

译者 (YuleFox) 笔记
~~~~~~~~~~~~~~~~~~~~~~~~~~~~~~~~~~~~

#. 对于代码格式, 因人, 系统而异各有优缺点, 但同一个项目中遵循同一标准还是有必要的;
#. 行宽原则上不超过 80 列, 把 22 寸的显示屏都占完, 怎么也说不过去;
#. 尽量不使用非 ASCII 字符, 如果使用的话, 参考 UTF-8 格式 (尤其是 UNIX/Linux 下, Windows 下可以考虑宽字符), 尽量不将字符串常量耦合到代码中, 比如独立出资源文件, 这不仅仅是风格问题了;
#. UNIX/Linux 下无条件使用空格, MSVC 的话使用 Tab 也无可厚非;
#. 函数参数, 逻辑条件, 初始化列表: 要么所有参数和函数名放在同一行, 要么所有参数并排分行;
#. 除函数定义的左大括号可以置于行首外, 包括函数/类/结构体/枚举声明, 各种语句的左大括号置于行尾, 所有右大括号独立成行;
#. ``.``/``->`` 操作符前后不留空格, ``*``/``&`` 不要前后都留, 一个就可, 靠左靠右依各人喜好;
#. 预处理指令/命名空间不使用额外缩进, 类/结构体/枚举/函数/语句使用缩进;
#. 初始化用 ``=`` 还是 ``()`` 依个人喜好, 统一就好;
#. ``return`` 不要加 ``()``;
#. 水平/垂直留白不要滥用, 怎么易读怎么来.
#. 关于 UNIX/Linux 风格为什么要把左大括号置于行尾 (``.cc`` 文件的函数实现处, 左大括号位于行首), 我的理解是代码看上去比较简约, 想想行首除了函数体被一对大括号封在一起之外, 只有右大括号的代码看上去确实也舒服; Windows 风格将左大括号置于行首的优点是匹配情况一目了然.

译者(acgtyrant)笔记
~~~~~~~~~~~~~~~~~~~~~~~~~~~~~~~~~~~~~~

#. 80 行限制事实上有助于避免代码可读性失控, 比如超多重嵌套块, 超多重函数调用等等. 
#. Linux 上设置好了 Locale 就几乎一劳永逸设置好所有开发环境的编码, 不像奇葩的 Windows.
#. Google 强调有一对 if-else 时, 不论有没有嵌套, 都要有大括号. Apple 正好 `有栽过跟头 <http://coolshell.cn/articles/11112.html>`_ .
#. 其实我主张指针/地址操作符与变量名紧邻, ``int* a, b`` vs ``int *a, b``, 新手会误以为前者的 ``b`` 是 ``int *`` 变量, 但后者就不一样了, 高下立判. 
#. 在这风格指南里我才刚知道 C++ 原来还有所谓的 `Alternative operator representations <http://en.cppreference.com/w/cpp/language/operator_alternative>`_, 大概没人用吧. 
#. 注意构造函数初始值列表(Constructer Initializer List)与列表初始化(Initializer List)是两码事, 我就差点混淆了它们的翻译. 
#. 事实上, 如果您熟悉英语本身的书写规则, 就会发现该风格指南在格式上的规定与英语语法相当一脉相承. 比如普通标点符号和单词后面还有文本的话, 总会留一个空格; 特殊符号与单词之间就不用留了, 比如 ``if (true)`` 中的圆括号与 ``true``.
#. 本风格指南没有明确规定 void 函数里要不要用 return 语句, 不过就 Google 开源项目 leveldb 并没有写; 此外从 `Is a blank return statement at the end of a function whos return type is void necessary? <http://stackoverflow.com/questions/9316717/is-a-blank-return-statement-at-the-end-of-a-function-whos-return-type-is-void-ne>`_ 来看, ``return;`` 比 ``return ;`` 更约定俗成(事实上 cpplint 会对后者报错, 指出分号前有多余的空格), 且可用来提前跳出函数栈. 
